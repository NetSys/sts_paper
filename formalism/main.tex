\documentclass{article}
\usepackage{indentfirst}
\usepackage{graphicx,xspace,subfigure,outlines,framed,subfigure,paralist,multirow,times,amsmath,amssymb}
\usepackage[ruled,vlined]{algorithm2e}
\usepackage[footnotesize]{caption}
%\usepackage{times}
\usepackage{xcolor}
\usepackage{xspace}
\usepackage{pifont}
\usepackage{pseudocode}
\usepackage{changepage}

\definecolor{MyDarkBlue}{rgb}{0,0.08,0.45}
\usepackage[pdftex]{hyperref}
\hypersetup{
  colorlinks,%
  citecolor=MyDarkBlue,%
  filecolor=MyDarkBlue,%
  linkcolor=MyDarkBlue,%
  urlcolor=MyDarkBlue
}
\usepackage[sort]{cite}

% Macros for generating graphical event sequences:
\usepackage{tikz}
% External event. Takes label as parameter.
\newcommand{\external}[1]{
\tikz[baseline=-0.5ex]\draw[black,fill=white] (0,0)
node[rounded corners,fill=red!60,draw,inner sep=0pt,minimum size=0.4cm] {$e_{#1}$};}%
% Internal event. Takes label as parameter.
\newcommand{\internal}[1]{
\tikz[baseline=-0.5ex]\draw[black,fill=white] (0,0)
node[rounded corners,fill=green!80,draw,inner sep=0pt,minimum size=0.4cm] {$i_{#1}$};}%

\makeatletter
% Functional foreach construct
% #1 - Function to call on each comma-separated item in #3
% #2 - Parameter to pass to function in #1 as first parameter
% #3 - Comma-separated list of items to pass as second parameter to function #1
\def\foreach#1#2#3{%
  \@test@foreach{#1}{#2}#3,\@end@token%
}

% Internal helper function - Eats one input
\def\@swallow#1{}

% Internal helper function - Checks the next character after #1 and #2 and
% continues loop iteration if \@end@token is not found
\def\@test@foreach#1#2{%
  \@ifnextchar\@end@token%
    {\@swallow}%
    {\@foreach{#1}{#2}}%
}

% Internal helper function - Calls #1{#2}{#3} and recurses
% The magic of splitting the third parameter occurs in the pattern matching of the \def
\def\@foreach#1#2#3,#4\@end@token{%
  #1{#2}{#3}%
  \@test@foreach{#1}{#2}#4\@end@token%
}

\makeatother

% Helper method, prepends a rightarrow before a node, throws away first
% argument
% TODO(cs): figure out how to make this a private method.
\def\connecthelper#1#2{#2$\mathtt{\rightarrow\!\!}$}
% Prepend a rightarrow before a node
\def\connect#1{\connecthelper{}{#1}}

% Chain together a list of nodes. Takes two parameters:
% #1: a comma-delimited list of the of everything in the list *except* the
% last element.
% #2: the *tail* of the chain
% TODO(cs): figure out how to delimit the first comma so that tail doesn't
% need to be specified separately.
\def\chain#1#2{\foreach{\connecthelper}{}{#1}$\mathtt{\;\cdot\!\!\cdot\!\!\cdot\!}$#2}

\newcommand{\tbd}[1]{{\bf [[TBD: {#1}]]}}
\newcommand{\ie}{{\it i.e.}}
\newcommand{\eg}{{\it e.g.}}
\newcommand{\cf}{{\it cf.}}
\newcommand{\etc}{{\it etc.}}
\newcommand{\viz}{{\it viz.}}
\newcommand{\apriori}{{\it a priori}}
\newcommand{\eat}[1]{}

% Notes:
\newcommand{\num}[1]{{\color{red}\bf {#1}}}
\newcommand{\panda}[1]{{\color{ForestGreen}\bf TK: {#1}}}
\newcommand{\andi}[1]{{\color{blue}\bf AW: {#1}}}
\newcommand{\colin}[1]{{\color{red}\bf CS: {#1}}}
\newcommand{\scott}[1]{{\color{purple}\bf SS: {#1}}}
\newcommand{\barath}[1]{{\color{red}\bf BR: {#1}}}

%\newcommand{\teemu}[1]{}
%\newcommand{\andi}[1]{}
%\newcommand{\sam}[1]{}
%\newcommand{\colin}[1]{}
%\newcommand{\scott}[1]{}
%\newcommand{\barath}[1]{}

% Delta-debugging symbols:
\newcommand{\PASS}{\text{\ding{52}}\xspace}
\newcommand{\DFAIL}{\text{\ding{56}}\xspace}
\newcommand{\cpass}{{T_{\scriptscriptstyle \PASS}}}
\newcommand{\cfail}{{T_{\scriptscriptstyle \DFAIL}}}
\newcommand{\dpass}{{T'_{\scriptscriptstyle \PASS}}}
\newcommand{\dfail}{{T'_{\scriptscriptstyle \DFAIL}}}
\newcommand{\done}{{T_{\scriptscriptstyle 1}}}
\newcommand{\dtwo}{{T_{\scriptscriptstyle 2}}}
\newcommand{\test}{\textit{replay}\xspace}
\newcommand{\ddmin}{\textit{ddmin}\xspace}

%\newenvironment{condition}[1][Condition]{\begin{trivlist}
%\item[\hskip \labelsep {\bfseries #1}]}{\end{trivlist}}

\newtheorem{theorem}{Theorem}[section]
\newtheorem{condition}[theorem]{Condition}

\sloppy
\begin{document}

\title{Formalizing Minimal Causal Sequences}

\author{Colin Scott \and Aurojit Panda \and Scott Shenker}
   \date{}
   \maketitle
   \thispagestyle{empty}

\section{Overview}
\label{sec:intro}
The SDN platform's $raison\text{ }d'\hat{e}tre$ is to 
hide complexity from control applications. To this end, modern platforms perform
replication, resource arbitration, failure recovery, and network 
virtualization on the control application's behalf. 

While these measures are effective in simplifying control applications,
they do not remove any complexity from the overall system. Rather, they merely move the complexity
from control applications into the underlying SDN platform.

As in any software system, additional complexity increases the probability of
bugs. And unfortunately for the network operator, finding bugs in the platform requires
access to precisely the details hidden from the control application.
When operators encounters erratic behavior in their network, the error's root
cause may lie in their own policy specification, or in the SDN platform
itself. To deal with the latter case, they must trace through
multiple layers of abstraction: virtualization logic, distribution logic, and
network devices.

As it stands, the SDN platform provides meager support for troubleshooting.
The predominant troubleshooting method is log analysis: manually
specifying log statements at relevant points throughout the system,
collecting; gathering; and ordering distributed log files; and analyzing the
results {\it post-hoc} when a error is encountered in production. Besides its
apparent tediousness, this approach is lacking in several ways: logs events
are enormous in number, impossible to aggregate into a single serial
execution of the system, and often at the wrong level of granularity to be of
use.

Recent work has contributed much-needed improvements to the highest (control
application) and lowest (dataplane forwarding tables) levels of abstraction, 
but no principled troubleshooting mechanism exists yet for the SDN platform.
NICE applies concolic execution and model checking to SDN control
applications, thereby automating the testing process and catching bugs before
they are deployed~\cite{nice}. Aneater~\cite{anteater} and HSA~\cite{hsa}
introduce mechanisms for checking static invariants in the dataplane.

It would be unthinkable to introduce a new programming language without a
debugger. Similarly, we think it highly undesirable to deploy SDN-based
networking without a viable troubleshooting paradigm. 

Correctness of the SDN platform can be stated concisely: high-level policies
should correspond with low-level configuration. We observe that the structure
of the SDN platform, graphs at every layer, enables a straightforward
algorithm to check this invariant. Our algorithm, which we term correspondence checking,
enumerates all inconsistencies at any point in time and isolates the
root cause of an inconsistency to a particular component of the system.

In eventually-consistent systems such as software-defined
networks however, transient inconsistencies between network policy and actual network
behavior are an inevitable state-of-affairs.
In such an environment, it does not suffice for troubleshooting tools to
simply enumerate inconsistencies; they should also aid the developer
in identifying which are related to serious problems, and which are
harmlessly ephemeral. To this end we present \simulator.
\Simulator allows troubleshooters 
to sift out pernicious inconsistencies by tracking the life cycle of problems 
both forward and backward in time.

We have implemented prototypes
of correspondence checking and \simulator. Our code is publicly available
at~\cite{github}.

The rest of this paper is organized as follows. In \S\ref{sec:overview},
we present an overview of the SDN stack and its failure modes.
In \S\ref{sec:approach} we present correspondence checking and
\simulator in detail. In \S\ref{sec:evaluation} we present
two use-cases and a preliminary performance evaluation
Finally, in \S\ref{sec:related_work} we discuss related work,
and in \S\ref{sec:conclusion} we conclude.


\section{Problem Statement}
\label{sec:problem_statement}
We represent the forwarding state of the network
at a particular time as a configuration $c$, which contains all the forwarding
entries in the network
as well as the liveness of the various network elements.
The control software is a system % (consisting of one or more controller processes)
that takes a sequence of
external network events $E = e_1,e_2,\dots,e_m$ (such as link failures) as input,
and produces a sequence of network configurations
$C = c_1,c_2,\dots,c_n$. Note that the network configuration $c$ does not
include the internal state of the control software.

An invariant is a predicate $P$ over forwarding state (a safety
condition, such as having no loops or blackholes). We say that a configuration
$c$ violates the invariant if $P(c)$ does not
hold, denoted as $\overline{P}(c)$.

\subsection{Log Input}

We are given a log $L$ of a system execution generated
by a centralized QA test orchestrator. The log $L$ contains external
events $E_L = e_1,e_2,\dots,e_m$, and
timestamps $T_L = \left\{ (e_k, t_k) \right\}$ of when the external events
occurred, recorded from the test orchestrator's clock.
A replay of log $L$ involves replaying the external events along with a
particular timing $T$,
which need not be identical to the original timings $T_L$ captured in the
log. We
denote a replay attempt by $replay(E_L,T)$.
The output of $replay$ is a sequence of forwarding state configurations
$C_R = \hat{c}_1,\hat{c}_2,\dots,\hat{c}_n$. In the ideal case $replay(E_L,T_L)$ reproduces the same
sequence of network configurations as occurred in the original execution, but as we discuss later
this does not always hold.

If the configuration sequence $C_L = c_1,c_2,\dots,c_n$ associated with the log $(E_L, T_L)$ violated predicate $P$
(\ie~$\exists_{c_i \in C_L}. \overline{P}(c_i)$)
then we say $replay(E_L,T) = C_R$ reproduces that invariant violation if
$\exists_{\hat{c}_i \in C_R}. \overline{P}(\hat{c}_i)$.

\subsection{Internal Events}

As stated, the log $(E_L, T_L)$ does not include
information about events that are internal to the control software, including
(a) message delivery events, either between controllers (\eg~database
synchronization messages) or
between controllers and switches (\eg~OpenFlow commands), and (b) state transitions
within controllers (\eg~a backup node deciding to become master).

Internal events are important for ensuring that $replay$ reproduces the
original invariant violation. In particular, if the following (conservative) condition holds~\cite{tel2000introduction}, $replay(E_L, \cdot)$ will be
guaranteed to reproduce the original violation:

%\begin{adjustwidth}{1cm}{}
\begin{condition}
\label{happens_before}
Each input $e$ is injected only after all other events, including internal events, that precede it in the
happens-before~\cite{Lamport:1978:TCO:359545.359563} relation ($\{i \mid i \rightarrow e\}$) from the
original execution have occurred.
\end{condition}
%\end{adjustwidth}

\noindent Note that $replay(E_L, T_L)$ does not necessarily meet this condition, since the control software may behave
differently during replay due to non-determinism,\footnote{As we explain
later, the control software may also behave differently as a result of
pruning inputs from the original run.} and simply maintaining the timings may not suffice
to account for those difference in behavior.

While it is not practically feasible for us to to observe and record all state
transitions within control software, we can feasibly record message delivery
events. We can therefore augment the original log $(E_L, T_L)$ with a schedule
$\tau_L = e_1\rightarrow~i_1\rightarrow~\dots~e_2~\rightarrow~\dots$, where
each $i$ is a message delivery event observed in the original
execution.\footnote{We further assume that we can arbitrarily reorder or
drop message delivery events (through interposition) during $replay$.}

With the schedule $\tau_L$, we schedule input events and message delivery
events during $replay$ so that we enforce an order
of events that is consistent with $\tau_L$. While this does not imply that
Condition~\ref{happens_before} holds (since we do not observe all internal
events), enforcing an order consistent with
$\tau_L$ is a necessary condition for ensuring that
Condition~\ref{happens_before} holds.

\subsection{Minimization}

The goal of our work is, when given a log $(E_L, T_L, \tau_L)$ that exhibited an
invariant violation, to find a small sequence of events that reproduces that
invariant violation. Formally, we define a minimal causal sequence (MCS)
to be a subsequence $E_M$
of $E_L$ and a timing $T_M$ such
that $replay(E_M,T_M)$ reproduces the invariant violation, but for all proper
subsequences $E_N$ of $E_M$
there is no timing $T$ s.t. $replay(E_N,T)$ reproduces the violation.
That is, an MCS is a sequence and timing of external events that reproduces the violation,
where one cannot find a subsequence of the MCS that reproduces the violation.
Note that an MCS is not necessarily {\em globally} minimal, in that there could be smaller
subsequences that reproduce this violation, but are not a subsequence of this MCS.

In the process of minimizing, we employ delta debugging~\cite{Zeller:1999:YMP:318773.318946}
to iteratively compute
subsequences $E_S\subseteq E_L$, and then replay each $E_S$ and check for an
invariant violation. Assume each $E_S$ is such
that all logically dependent events occur, \eg~links go down before coming up,
and hosts migrate correctly. Each subsequence of events $E_S$ corresponds with
timings for those events $T_S\subseteq T_L$.

As explained above, $replay(E_S, T_S)$ does not necessarily reproduce the
invariant violation (even if $E_S$ is a superset of an MCS), since it may
violate Condition~\ref{happens_before}. Rather than replaying with $T_S$, we attempt to produce a
schedule $\tau_S$ such that all events in $\tau_L$ are arranged in $\tau_S$
such that: \\

If $a, b\in \tau_S$ and $a, b\in \tau_L$ then $a \rightarrow b$ in $\tau_S$ if
and only if $a \rightarrow b$ in $\tau_L$. \\

\noindent This just says that $\tau_S$ follows the happens-before order reflected in the
original log.\footnote{But not necessarily the full happens-before order from
the original execution, since we do not have visibility into all internal
events.}

Constructing $\tau_S$ involves three issues: coping with syntactic differences in internal
events across replay runs (\S\ref{sec:functional_equivs}),
handling internal events from the original
execution that may not occur after pruning (dealt with in~\cite{sts}, Section 2),
and dealing with new internal events that were not
observed at all in the original execution (\S\ref{sec:unexpected}).

% ---------------------------------------------------------------
\eat{
\textbf{Inputs}
\begin{align*}
    E_l &= \left\{ e_1, e_2, \ldots, e_n  \right\} && \text{External log where $e_j$ is an external event: link up,
down, etc.}\\
    I_l &= \left\{ i_1, i_2, \ldots, i_n \right\} && \text{Internal log where $i_j$ is an internal events: \newline received
or sent messages, etc.}\\
    T_l &= \left\{ (e_k, t) \right\} &&\text{Time at which external event $e_k$ happens}\\
    \tau_l &= e_1\rightarrow i_1\rightarrow \ldots e_2\rightarrow i_m \ldots && \text{Schedule: Causal order of events}
\end{align*}

\textbf{Given functions}
\begin{align*}
    Replay&: E\times T  &&\text{Replay function used to simulate system and discover final state}
\end{align*}

\textbf{Assumptions}
The replay of a particular set of external events and times results in a particular set of internal events (which we can
observe) which can be potentially reordered causally. In particular:
\begin{align*}
    Replay(E_l, T) &\implies I_l
\end{align*}

That is when replaying the original schedule with the original timing we get the original set of internal events.


\begin{align*}
    Replay(E, T) &\implies I\\
    I \nsubseteq I_l\\
    I\cap I_l \neq \emptyset
\end{align*}

}


\section{Functional Equivalence}
\label{sec:functional_equivs}
Internal events may differ syntactically (\eg~sequence numbers
of control packets may all differ) when replaying a subsequence of the original log.
Nonetheless, we observe that often these altered internal events are {\em functionally
equivalent}, in the sense that they
have the same effect on the state of the system with respect to triggering the
invariant violation (despite syntactic differences).

Formally, $\dots$.


\section{Coping With Unexpected Events}
\label{sec:unexpected}
Another possible change induced by pruning inputs is the occurrence of new
internal events that were not observed in the original log.
New events are an indication that the control software's state machine has
diverged from the path it took in the original run. New events therefore present multiple
possibilities for where
we should inject the subsequent input. Consider the following case:
if $i_2$ and $i_3$ are internal events observed
during replay that are both in the same equivalence class as a single event $i_1$ from the
original run, we could inject the subsequent input after $i_2$ or after $i_3$.

\colin{Measurement question: what if we just immediately gave up on any
subsequences that exhibited new events? How much minimization would we achieve?}

% TODO: figure this figure out
%\begin{wrapfigure}{c}{1.3\linewidth}
%  \centering
%  \includegraphics[width=\linewidth,height=0.8in]{../diagrams/state_machines/event_sequence.pdf}
%\end{wrapfigure}

In the general case it is always possible to construct two state machines that lead
to differing outcomes: one that only leads to the invariant violation when
we inject the next input
\emph{before} a new internal event, and another only when we inject \emph{after} a new internal
event. In other words, to be guaranteed to traverse any existing suffix that leads
to the invariant violation, we must recursively branch, trying both
possibilities for every new internal event. This implies an exponential number of
possibilities to be explored in the worst case.\footnote{
In our system~\cite{sts}, exponential search over these possibilities is not a practical option.
The heuristic our system uses when waiting for expected internal
events is to proceed normally if there are new internal events,
always injecting the next input when its last expected predecessor
either occurs or times out.}

\subsection{Modeling the Control Software}

If we assume a model of the control software's state machine, we can potentially
avoid exponential blowup. Assume we have a predicate $\Phi(\tau_P, a, b)$
that, given a prefix $\tau_P$ of events scheduled so far, and a pair of events $a$, $b$
(either external or internal), returns true whenever $\tau_P || a \rightarrow
b$ leads the software's state machine to a state $s$ s.t. some buggy state
$\hat{s}$ is reachable from $s$ along a sequence of labeled state transitions
$\alpha$, where the labels in $\alpha$
that correspond to external events are a subsequence of the remaining external inputs $E_S
\backslash \left\{ e \in \tau_P \right\}$. \colin{May want to remove the
requirement that $\alpha$ be a subsequence of external events.}
Here, a buggy state refers to
a state that produces a network configuration $c$ that violates a given
invariant.

For all prefixes $\tau_P || a\rightarrow b$ of $\tau_L$, if $\tau_L$ is a
superset of an MCS, then $\Phi(\tau_P, a, b)$ returns true.

For any new internal event $a$ and pending external event $b$, if
$\Phi(\tau_P, a, b)$ returns true, then we know that we can allow $a$ through
before $b$. In this way we can interleave new
internal events with our schedule of expected events $\tau_S$, and be
guaranteed to leave open the possibility of finding a divergent suffix through the
state machine that leads to an invariant violation.

\subsection{Obtaining a model}

Two ideas for obtaining $\Phi$: \\

\noindent{\bf Extrapolate from complete model.} There was a paper in
PLDI~\cite{vericon} where the authors
wrote their own controller and verified its entire state space. We could issue
queries on the state space to compute $\Phi$, and then extrapolate $\Phi$ for
other controllers assuming their state machines are sufficiently similar. \\

\noindent{\bf Infer partial models.} In general we want to
continue assuming that the control software is a black box.
Even with that constraint, we could infer a partial model of the control
software by observing its outputs across many executions, \ie~build a model
from logs. We might use Synoptic~\cite{beschastnikh2011leveraging} to compute
the model.


\section{Coping with Non-determinism}
\label{sec:non_determinism}
\eat{ % Is the motivation not having visilbity into internal state transitions, or is non-determinism?
The schedule reflected in our log
$\tau_L = e_1\rightarrow~i_1\rightarrow~\dots~e_2~\rightarrow~\dots$ does not
include internal state transitions (within the control software).

Due to non-determinism,
replay may exhibit non-determinism, \ie~we are not be guaranteed to
reproduce the invariant violation even if we maintain the happens-before
relation for all events we can observe.
}

Completely coping with non-determinism involves two issues:
\begin{enumerate}
\item Obtaining visibility into internal state transitions.
\item Having control over internal state transitions during replay.
\end{enumerate}

Let's assume that we have deterministic replay logging enabled during runtime. This gives us
visibility.

Let's further assume that the OS under the control software provides us with primitives for enforcing determinism,
as in dOS~\cite{bergan2010deterministic} or Determinator~\cite{aviram2012efficient}.

With these primitives, how would we completely cope with non-determinism?

Not clear that we can completely cope with it. Consider "control flow
volitility": random number
generators, decisions based on clock readings, hash tables that use
non-trivial hashing functions, and control flow depends on order to make decisions.


\section{What Systems This Works Well On}
\label{sec:systems}
The characteristics of $\Phi$ would help us answer what other systems besides
SDN control software our technique would work well on.

Two hypotheses for existing constraints on $\Phi$: quiescence, and centralization.


\section{Partial Program Flow Analysis}
\label{sec:program_flow}
Suppose we ran program flow analysis on our mock network (to predict internal
events it will trigger), but not on the control software? This would still
allow us to remain agnostic to the language of the control software, but it might help us get better replay success.


% TODO: Include acknowledgements section
\renewcommand{\baselinestretch}{0.97}
    \bibliographystyle{abbrv}
    \bibliography{bib}
    \normalsize


\end{document}


%\input{appendix}
%\theendnotes

\end{document}
