By choosing a subsequence of inputs to replay, \ie~pruning some of the original inputs,
we may cause the messages sent by the control software to differ syntactically
from those in the original run. For example, consider the sequence numbers of control
messages: if we prune an input at the beginning of the execution, then the
control software's sequence number counter may increment one less time than it
did in the original execution, and all
control packets sent by it will therefore carry a different sequence number
than they did in the original trace. These syntactic differences complicate
our goal of ensuring an event ordering that is consistent with $\tau_L$.

We observe that often these altered internal events are {\em functionally
equivalent}, in the sense that they
have the same effect on the state of the system with respect to triggering the
invariant violation (despite syntactic differences).

Formally, $\dots$.
