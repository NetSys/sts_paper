We represent the forwarding state of the network
at a particular time as a configuration $c$, which contains all the forwarding
entries in the network
as well as the liveness of the various network elements.
The control software is a system % (consisting of one or more controller processes)
that takes a sequence of
external network events $E = e_1,e_2,\dots,e_m$ (such as link failures) as input,
and produces a sequence of network configurations
$C = c_1,c_2,\dots,c_n$. Note that the network configuration $c$ does not
include the internal state of the control software.

An invariant is a predicate $P$ over forwarding state (a safety
condition, such as having no loops or blackholes). We say that a configuration
$c$ violates the invariant if $P(c)$ does not
hold, denoted as $\overline{P}(c)$.

\subsection{Log Input}

We are given a log $L$ of a system execution generated
by a centralized QA test orchestrator. The log $L$ contains external
events $E_L = e_1,e_2,\dots,e_m$, and
timestamps $T_L = \left\{ (e_k, t_k) \right\}$ of when the external events
occurred, recorded from the test orchestrator's clock.
A replay of log $L$ involves replaying the external events along with a
particular timing $T$,
which need not be identical to the original timings $T_L$ captured in the
log. We
denote a replay attempt by $replay(E_L,T)$.
The output of $replay$ is a sequence of forwarding state configurations
$C_R = \hat{c}_1,\hat{c}_2,\dots,\hat{c}_n$. In the ideal case $replay(E_L,T_L)$ reproduces the same
sequence of network configurations as occurred in the original execution, but as we discuss later
this does not always hold.

If the configuration sequence $C_L = c_1,c_2,\dots,c_n$ associated with the log $(E_L, T_L)$ violated predicate $P$
(\ie~$\exists_{c_i \in C_L}. \overline{P}(c_i)$)
then we say $replay(E_L,T) = C_R$ reproduces that invariant violation if
$\exists_{\hat{c}_i \in C_R}. \overline{P}(\hat{c}_i)$.

\subsection{Internal Events}

As stated, the log $(E_L, T_L)$ does not include
information about events that are internal to the control software, including
(a) message delivery events, either between controllers (\eg~database
synchronization messages) or
between controllers and switches (\eg~OpenFlow commands), and (b) state transitions
within controllers (\eg~a backup node deciding to become master).

Internal events are important for ensuring that $replay$ reproduces the
original invariant violation. In particular, if the following (conservative) condition holds~\cite{tel2000introduction}, $replay(E_L, \cdot)$ will be
guaranteed to reproduce the original violation:

%\begin{adjustwidth}{1cm}{}
\begin{condition}
\label{happens_before}
Each input $e$ is injected only after all other events, including internal events, that precede it in the
happens-before~\cite{Lamport:1978:TCO:359545.359563} relation ($\{i \mid i \rightarrow e\}$) from the
original execution have occurred.
\end{condition}
%\end{adjustwidth}

\noindent Note that $replay(E_L, T_L)$ does not necessarily meet this condition, since the control software may behave
differently during replay due to non-determinism,\footnote{As we explain
later, the control software may also behave differently as a result of
pruning inputs from the original run.} and simply maintaining the timings may not suffice
to account for those difference in behavior.

While it is not practically feasible for us to to observe and record all state
transitions within control software, we can feasibly record message delivery
events. We can therefore augment the original log $(E_L, T_L)$ with a schedule
$\tau_L = e_1\rightarrow~i_1\rightarrow~\dots~e_2~\rightarrow~\dots$, where
each $i$ is a message delivery event observed in the original
execution.\footnote{We further assume that we can arbitrarily reorder or
drop message delivery events (through interposition) during $replay$.}

With the schedule $\tau_L$, we schedule input events and message delivery
events during $replay$ so that we enforce an order
of events that is consistent with $\tau_L$. While this does not imply that
Condition~\ref{happens_before} holds (since we do not observe all internal
events), enforcing an order consistent with
$\tau_L$ is a necessary condition for ensuring that
Condition~\ref{happens_before} holds.

\subsection{Minimization}

The goal of our work is, when given a log $(E_L, T_L, \tau_L)$ that exhibited an
invariant violation, to find a small sequence of events that reproduces that
invariant violation. Formally, we define a minimal causal sequence (MCS)
to be a subsequence $E_M$
of $E_L$ and a timing $T_M$ such
that $replay(E_M,T_M)$ reproduces the invariant violation, but for all proper
subsequences $E_N$ of $E_M$
there is no timing $T$ s.t. $replay(E_N,T)$ reproduces the violation.
That is, an MCS is a sequence and timing of external events that reproduces the violation,
where one cannot find a subsequence of the MCS that reproduces the violation.
Note that an MCS is not necessarily {\em globally} minimal, in that there could be smaller
subsequences that reproduce this violation, but are not a subsequence of this MCS.

In the process of minimizing, we employ delta debugging~\cite{Zeller:1999:YMP:318773.318946}
to iteratively compute
subsequences $E_S\subseteq E_L$, and then replay each $E_S$ and check for an
invariant violation. Assume each $E_S$ is such
that all logically dependent events occur, \eg~links go down before coming up,
and hosts migrate correctly. Each subsequence of events $E_S$ corresponds with
timings for those events $T_S\subseteq T_L$.

As explained above, $replay(E_S, T_S)$ does not necessarily reproduce the
invariant violation (even if $E_S$ is a superset of an MCS), since it may
violate Condition~\ref{happens_before}. Rather than replaying with $T_S$, we attempt to produce a
schedule $\tau_S$ such that all events in $\tau_L$ are arranged in $\tau_S$
such that: \\

If $a, b\in \tau_S$ and $a, b\in \tau_L$ then $a \rightarrow b$ in $\tau_S$ if
and only if $a \rightarrow b$ in $\tau_L$. \\

\noindent This just says that $\tau_S$ follows the happens-before order reflected in the
original log.\footnote{But not necessarily the full happens-before order from
the original execution, since we do not have visibility into all internal
events.}

Constructing $\tau_S$ involves three issues: coping with syntactic differences in internal
events across replay runs (\S\ref{sec:functional_equivs}),
handling internal events from the original
execution that may not occur after pruning (dealt with in~\cite{sts}, Section 2),
and dealing with new internal events that were not
observed at all in the original execution (\S\ref{sec:unexpected}).

% ---------------------------------------------------------------
\eat{
\textbf{Inputs}
\begin{align*}
    E_l &= \left\{ e_1, e_2, \ldots, e_n  \right\} && \text{External log where $e_j$ is an external event: link up,
down, etc.}\\
    I_l &= \left\{ i_1, i_2, \ldots, i_n \right\} && \text{Internal log where $i_j$ is an internal events: \newline received
or sent messages, etc.}\\
    T_l &= \left\{ (e_k, t) \right\} &&\text{Time at which external event $e_k$ happens}\\
    \tau_l &= e_1\rightarrow i_1\rightarrow \ldots e_2\rightarrow i_m \ldots && \text{Schedule: Causal order of events}
\end{align*}

\textbf{Given functions}
\begin{align*}
    Replay&: E\times T  &&\text{Replay function used to simulate system and discover final state}
\end{align*}

\textbf{Assumptions}
The replay of a particular set of external events and times results in a particular set of internal events (which we can
observe) which can be potentially reordered causally. In particular:
\begin{align*}
    Replay(E_l, T) &\implies I_l
\end{align*}

That is when replaying the original schedule with the original timing we get the original set of internal events.


\begin{align*}
    Replay(E, T) &\implies I\\
    I \nsubseteq I_l\\
    I\cap I_l \neq \emptyset
\end{align*}

}
