In this section we present two mechanisms to facilitate the troubleshooting
process. Correspondence checking allows troubleshooters to isolate
the cause of policy violations to a particular layer. \Simulator{}
allows troubleshooters to isolate relevant events throughout the system
execution. We provide the details of these techniques below. 

\subsection{Correspondence Checking}

\colin{reviewer b: Correspondence checking as defined in this paper captures a
rather weak condition that allows implementations that intuitively seem
incorrect.} \colin{I think we should demphasize correspondence checking. After
describing it, we should treat it as a ``black 
box'' policy-violation detector that serves as an important component in
casual analysis. Also, 
apparently we need to do a better job of calling out that correspondence
checking 
is indeed a weak condition. Finding real bugs will help with this objection, I
believe.}

Platform correctness can be expressed as a simple invariant:
the policy specified by the application layer should correspond to the
configuration of the
physical network. We check this invariant by applying the virtual packet
algebra pioneered in headerspace analysis~\cite{hsa}. 

Formally, each layer of the SDN stack can be represented as a graph,
$G = (V, E)$. Packets are series of bits, $h \in \{0,1\}^L = H$,
where $L$ is the maximum number of bits in the header. Upon receiving a packet,
forwarding elements apply a transformation function, potentially modifying
packets before forwarding them on\footnote{Multicast forwarding can expressed
by extending the range to sets of output tuples}:
\begin{align*}
T: (H \times E) \rightarrow (H \times E_{\emptyset})
\end{align*}

\colin{reviewer B: what is $E_{\emptyset}$? it's not defined}

We use $`\Psi`$ to denote the collection of all transfer functions present in
the network at a particular point in time. In this model, network traversal is simply a composition of transformation
functions. For example, if a header $h$ enters the network through edge
$e$, its state after $k$ hops will be:
\begin{align*}
\Phi^k(h,e) = \Psi(\Psi(\dots \Psi(h,e)\dots))
\end{align*}

The externally visible behavior of the network can be expressed as the
transitive closure of $\Phi$:
\begin{align*}
\Omega: (H \times E_{access}) \rightarrow (H \times E_{\emptyset}) \\
\Omega(h,e) = \Phi^{\infty}(h,e)
\end{align*}
Here, $E_{access}$ denotes access links adjacent to end-hosts.

In SDN, it should always be the case that:
\begin{align*}
\Omega^{view} \sim \Omega^{physical}
\end{align*}
Informally, this means that any packet injected at an access link in $G^{virtual}$ should arrive at
the same final location as the corresponding (encapsulated) packet injected at the
corresponding access link in $G^{physical}$. Note that hosts are represented
in all graphs, although there may not be a one-to-one mapping between the
internal vertices of $G^{virtual}$ and $G^{physical}$.
\colin{reviewer B: the equivalence relation ~ is only defined informally\dots To
get an equivalence that does not suffer from this problem, you might consult
the literature on bisimulation equivalence (Robin Milner's small book on CCS
and the Pi-calculus is an excellent starting point; Xavier Leroy's work on the
CompCert verified compiler is a nice example of this in action). A
bisimulation would require that every action at the virtual level can be
simulated by an equivalent action at the physical level, and vice versa.}

To check correspondence in SDN, we begin by taking a causally consistent
snapshot~\cite{Chandy:1985:DSD:214451.214456} of the physical network.
\colin{reviewer A: how you take it? is the virtual layer included? is the
application layer included?} \colin{reviewer C: Does each layer have to output
a transformation
function in a specific format? Does this approach require a
standardization of the interfaces between layers in terms of some
low-level rules? It isn't clear to me that many layers would
interface with a forwarding-table like set of rules} The routing
tables of forwarding elements can then be translated into transformation functions.
Finally, we feed a symbolic packet $x^L$ to each access link of the
network.\footnote{The rules for process wildcard bits $x^n$ are defined in
the HSA paper~\cite{hsa}} The end result is a propagation graph representing all possible paths taken by a packet injected
at the access link.

The leaves of the propagation graph represent $\Omega$. We
verify correspondence in SDN by generating propagation graphs for all SDN layers,
and comparing the leaves. Any mismatch in leaves of the propagation graphs
represent policy violations between control applications and network
configuration.

\subsection{\SIMULATOR{}}

\colin{Amin Vahdat:
I understand another key benefit of SDN/OpenFlow is being able to play with a
lot of "what if" scenarios to enable you to fine-tune the network before going
live.

Exactly. So one of the key benefits we have is a very nice emulation and
simulation environment where the exact same control software that would be
running on servers might be controlling a combination of real and emulated
switching devices. And then we can inject a number of failure scenarios under
controlled circumstances to really accelerate our test work.}

Correspondence checking only captures a snapshot of the network state.
We have developed \simulator{} to allow troubleshooters to explore
policy violations over time. We base our replay on a consistent
trace of low level failure and topology change events, as enabled,
\eg{}, by OFRewind~\cite{ofrewind}. \colin{reviewer A: how is this defined?}
The events from the trace are fed into a simulator, which allows the
troubleshooter to invoke correspondence
checking at any point in time. The simulator focuses on corner-case events,
and models the failure modes in sufficient detail to reproduce the error, while
allowing for complete control of the timing, ordering, and constuction of the events.
\colin{reviewer A: As for the simulation-based replay analysis, it isn't clear what you
are simulating/emulating and what are the events included in the trace. Please
clarify.} \colin{reviewer D:  This is a nice result, and it
would be nice to see in more detail how this is accomplished.  Is there
an analysis performed on the control program to decide which aspects
of the failure modes of the system are relevant to a problem?
Unfortunately, insufficient information is presented about the model
to indicate that this is true.}


%and does not enable the troubleshooter to classify the
%gravity of the detected inconsistencies, i.e., whether the detected 
%inconsistencies are temporary, i.e., byproducts of the forwarding reconverging 
%after a link failure, or whether they are and persistent.
%Also, it cannot aid in isolating the precise \emph{even sequence} that triggered
%the faulty behavior.

If users were to run \simulator{} on raw event logs, they would encounter a
large number of failure events and transient policy violations. We describe
how users can apply \simulator{} to differentiate transient from persistent
policy violations below.

\textbf{Identifying persistent policy violations}. When a policy-violation is detected,
the simulators forks off a simulation branch that investigates the future system behavior
in a case where no further external events are played out. If the detected
policy-violation
is resolved in isolation within a customizable number of simulation time steps, it is considered
a temporary problem. If it is not, this is a strong indication that the system has indeed
reached a persistent policy-violation that needs to be investigated.

\textbf{Checking related problems by fuzzing.} Input traces can be \emph{fuzzed}, i.e.,
randomly perturbed, to expose the system to similar error conditions, and confirm
that a proposed solution is not just a point-fix. \colin{reviewer A: Fuzzing
can be applied across a multitude of dimensions. What kind of fuzzing do you
suggest using?}

\textbf{Investigating pathological environment conditions.} The simulator allows for investigation
of pathological environment conditions difficult to achieve in a real world test bed
(\eg{}, correlated failure rates, extremely long delays etc.). This enables
investigation of situations that have a high potential for triggering errors.

\textbf{Interactive exploration.} Troubleshooters can also interactively bisect
the trace or modify specific events to further pinpoint the cause for a failure.
This is useful as soon as a suspect event sequence has been identified.

\subsection{Discussion}
Correspondence checking and \simulator{} serve to isolate the platform layer and
event sequence responsible for a given error. To identify the root cause of
the failure in the code, they can be complemented by classical debugging
techniques,\ie{} log messages and source code debugging. These are much more
effective when applied to investigate a specific event sequence. Once a
potential fix has been developed, it can be validated by repeating the
problematic replay. Fuzzing helps to validate whether there may be
related error events that the patch may have left open.


