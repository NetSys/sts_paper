% TODO: cite ndb + Nick's keynote

%-- Program Slicing --

The delta debugging algorithm~\cite{Zeller:2002:SIF:506201.506206} seeks to solve
a problem that is exactly analogous to ours on a single machine: given input that causes a test case
to fail, what is the minimum subset of the input that still produces the failure?
We apply the same reasoning to a distributed system.

%-- Deterministic Replay (OFRewind) --

Deterministic replay techniques such as OFRewind~\cite{ofrewind}
are designed to allow developers to interactively prune
the inputs that lead up to errant behavior. We present an algorithm that
automates this process.

%-- Model checking (NICE): --

NICE~\cite{nice} combines model checking with concolic execution
to enumerate all possible code paths taken by control software (NOX)
and identify concrete inputs (\eg{} control message orderings) that cause
the network to enter invalid configurations. Unlike NICE, by analyzing
bugs {\em post-hoc} from live runs of the system our approach applies
to large software systems without suffering from state explosion.

%-- Invariant Checking? --

\eat{
Invariant checking tools such as Anteater~\cite{anteater} and HSA~\cite{hsa}
detect problems in the dataplane. We leverage invariant checking tools
to distinguish inputs that are necessary for reproducing a given invariant violation.
}

%-- Root cause analysis? --

Root cause analysis techniques~\cite{577079} seek to identify the minimum set of failed
components (\eg{} link failures) needed to explain a collection of alarms. Rather than
focusing on individual component failures, we seek to minimize inputs that affect the behavior
of the overall distributed system.

%-- Distributed Systems debuggers --

Pip~\cite{pip} is a framework for instrumenting general-purpose distributed systems
with code to record, display, and check invariants on causal paths throughout
live executions. \Simulator{} observes the causal behavior of the
distributed system in a simulated environment, enabling us to iteratively prune extraneous input events.

%-- Simulators? --
%
%Several other network simulators exist for testing SDN controllers. Mininet is a
%platform for emulating OpenFlow switches and hosts within a single
% VM~\cite{Lantz:2010:NLR:1868447.1868466}. The ns-series of network simulators
%provides a general framework for testing new protocols, topologies,
%and traffic mixes~\cite{ns3}. We found that these existing simulators did
%not provide sufficient support for the corner-cases situations which are the
%focus of our work, such as failures and VM migration.

%-- Distributed Systems --
%
%Many of our ideas originate from the literature on troubleshooting general
%distributed systems. WiDS checker introduced the notion of recording
%production executions to be later replayed and verified in a controlled simulation.
% Finally, end-to-end tracing
%frameworks such as X-Trace~\cite{Fonseca:2007:XPN:1973430.1973450} and
%Pinpoint~\cite{Chen02pinpoint:problem} provide a framework for tracing requests throughout
%a distributed system in order to infer correctness errors between layers and
%across components. Our work solves a more constrained problem; we leverage
%the structure of the SDN stack to enable a simple notion of platform
%correctness. In addition, these systems assume that invariants should hold at
%all times; we observe that in an eventually-consistent system such as SDN,
%transient correctness violations are inevitable. We built \simulator{} to help troubleshooters
%differentiate ephemeral from persistent errors.

% If we manage to run multiple applications by Monday, we should cite papers
% on consistency and cross-layer debugging:
%X-Trace~\cite{xtrace}
% Vector Clocks
% Onix
% Virtualization definitely won't happen by Monday. But, papers include
% Martin's presto '10 paper 'Virtualizaing the Network Forwarding Plane'

