% Just realized: b/c of anonymity, the PC can't chastise us for
% running our system on our own code -- we can't tell them that it's our code!
\Simulator{} is certainly not without limitations. We discuss several of them
here.

\noindent{\bf Isn't a global log difficult to obtain?} Some failure events may appear
multiple times in the log (\eg{} replica servers detecting that a master went down),
yet we only want to replay the original event.
If it is not possible to distinguish the original
failure event from the resulting events (\eg{} in the case of a disk failure
where the original crash message is not recoverable), developers may need to
manually decipher the original event, or apply root cause analysis tools for
detecting the originating event from correlated alarms~\cite{577079}.
We note that if the simulator is unable to reproduce the policy-violation,
the simulator may be able to `fuzz' different event
orderings in an attempt to retrigger it.

\eat{ % CS: cutting, since I have since realized that snapshots are purely an
% optimization. It might take a really long time, but you can always just
% start from the beginning of the log
\noindent{\bf Snapshots}
As networks typically run for long periods of time, and not all devices are 'turned on'
at the same time, our simulation needs a \emph{snapshot} of the network state as a
starting point. If supported by the network, we can rely on prior for for taking
causally consistent snapshots to achieve this goal. Otherwise, it may be possible
to take a series of snapshots during periods of quiet operation (e.g., at night)
and check for consistency.
}

\noindent{\bf Are all simulated failures really indistinguishable
from actual failures?}
Maybe not, but since the simulator
is built in software, the fidelity of our simulated failures
can be made arbitrarily sophisticated, subject to scalability
limitations of fine-grained simulations.

\noindent{\bf Is it reasonable to assume that external inputs events are
causally independent?} This depends on what the developer is trying to debug.
Consider a drunk network operator who (i) trips on an ethernet cable, and
consequently (ii) spills his drink on a switch (busting the hardware). Our
system views (i) and (ii) as causally independent, but in reality (i) caused
(ii). The developer's goal is not to debug drunk operators though.\footnote{We
leave that up to the operator's manager}

\noindent{\bf Will this approach work on all control platforms?}
\Simulator{} requires some controller-specific modifications, including
awareness of the API to specify policy changes, awareness of the format of log
messages, and causal annotation instrumentation. Correspondence checking also
assumes that it has access to a physical view. If no physical view is
available, alternate detectors could be used.

\noindent{\bf Will control platforms ever become stable enough to not require
          \simulator{}?}
We certainly hope so! The field is from that point today though, and there
will always be some latent bugs.

\colin{In log size analysis, mention two optimizations. log truncation (causally
consistent snapshots) and pruning branches of the happens-before}

%\noindent{\bf How is this specific to SDN?}
%
%Yes!

\eat{
Second, we hope to gather error logs from real production deployments which
will help us populate this repository; this may require providing novel kinds
of anonymization, so that large datacenter operators would be willing to share
their problems (since they want their SDN code to work) without revealing the
details of their network.  This may require a infrastructural counterpart to
minimally-causal events; the smallest number of infrastructure components that
can reproduce the same bug.
}

\eat{
\colin{TODO: add note about not being able to tell difference between
endogenous and exogenous events, which might hide the true cause of a bug.
This is OK though, since we assume that the platform should always have a
correct option, but it doesn't take it. That is, the root cause of the crash
is not what we're interested in (software vs. hardware), it's how the platform
reacts to that crash that matters (should be robust, no matter what the
cause of the crash)}

\andi{Don't quite know what you mean by endogenous vs. exogenous}

\colin{Note that we also assume an out-of-band management network between
switches and controllers, and that the management network always provides 
connectivity. We could add the management network into our model if we want, I
suppose.}
}

\eat{ % The production environment logs this anyway? Add this point in
      % if we have space.
\noindent{\bf Aren't the proposed production logs going to be huge?}

In contrast to general record-and-replay
mechanisms, the amount of recorded state needed for
high-fidelity replay is tractable\andi{Check with our newly, well defined strong assumptions: 
we need full internal/external events. Is this tractable}. With proactive flow installation,
updates are pushed to routing tables over a relatively long time scale; periodic
FIB snapshots along with a log of link state events, control server
downtime, host mobility information, and policy-changes suffice for our purposes.
Assuming a maximum of 256K routing or ACL entries per switch~\cite{cisco7000}, and 
36 bytes per entry, each FIB will contain a maximum of 9216
kilobytes, uncompressed. A fat tree network of 27,648 hosts
includes 2,880
switches~\cite{Al-Fares:2008:SCD:1402958.1402967}.
Therefore a snapshot of the FIBs of the entire network would naively take up roughly
26 GB. Note however, that the data is likely to be compressible quite well, due to do its
structural and temporal properties. Assuming 8.5 error events per minute per
datacenter~\cite{Greenberg:2009:VSF:1592568.1592576}, 1,000,000
VM placement changes per day per datacenter~\cite{Soundararajan:2010:CBS:1899928.1899941},
and a small rate of human-specified policy changes, the log of the external inputs
should grow at a rate of \textasciitilde 750 entries per minute.
}

%To account for host mobility, assume that each server hosts 10 VMs,
%and 1\% of VMs are created, suspended, or migrated every minute. Then 10,000 host mobility events must be
%logged per minute, also a reasonable storage cost.

\eat{
\colin{Notes from Rean Griffith:
\begin{itemize}
\item total vms in a typical datacenter: 1000
\item migration frequency (migrations/minute): 20 per hour
\item VM spin ups/downs: 150 power ons per hour (see our OSR 2010 paper for
power off estimates)
\item Do we log VM migrations and how does that log grow (I wasn't able to
get any estimates on log-growth data)
\end{itemize}

We had an OSR 2010 paper that provided numbers scaled by the number of
VMs in an installation:
Challenges in building scalable virtualized datacenter management
(http://dl.acm.org/citation.cfm?id=1899941)
}
}

%As a point of reference, border routers' working RIB size is
%$\textasciitilde$130MB~\cite{Karpilovsky:2006:UFR:1368436.1368439}.
\eat{

TODO: Replace this analysis.
It stinks. PORTLAND is not the right way to evaluate this due to the lacking number of rules.

\noindent{\bf Correspondence Checking Runtime.} 
Computing the propagation
graph for correspondence checking is equivalent to enumerating
all possible paths in the network, which scales with the diameter
of the network and the number of routing entries per switch.
The propagation graph for each host can be
computed in parallel however, so the computation is bottlenecked by the serial runtime
of computing a single host's propagation graph.

We show the serial runtime of correspondence checking in
Figure~\ref{fig:hsa_runtime}. For this analysis we generated fat tree topologies
between 2 and 48 pods wide, with pre-installed PORTLAND~\cite{NiranjanMysore:2009:PSF:1592568.1592575}
routing tables in each switch. Each data point is the minimum of three
runs on a single Intel Xeon 2.80GHz core. Note that the number of PORTLAND routing entries per switch scales with the number
of pods in the fat-tree. We excluded the time to convert
flow tables to HSA transfer functions, since transfer functions can be maintained
offline.

As the figure depicts, even for large networks
(27,648 hosts) the serial runtime of correspondence checking is reasonable for
interactive use. The number of serial tasks to be executed
is the number of hosts in the network squared, disregarding ECMP load balancing.

\begin{figure}[t]
    %\hspace{-10pt}
    \includegraphics[width=3.25in]{../graphs/hsa_overhead_graph/graph.pdf}
    \caption[]{\label{fig:hsa_runtime} Serial runtime of correspondence
    checking on PORTLAND fat tree networks. Each datapoint consists of
    $x^3/4$ hosts and $5x^2/4$ switches (\eg{} 48 pods means 27,468 hosts
    attached to 2,880 switches)}
\end{figure}
}

\eat{ This evaluation stinks. Be gone!

\noindent{\bf Simulator Scalability.} As our approach depends on the frequently
repeating simulations, we now evaluate the setup time incurred by the simulator
system when handling large network topologies. For this experiment, shown in
Figure~\ref{fig:scalability}, we generate fat tree topologies between 2 and 48
pods wide, where all switches in the network connected to a single controller.
The controller sends each switch an OpenFlow $FLOW\_MOD$ and subsequent
$BARRIER\_REQUEST$ message, and waits for the corresponding $BARRIER\_REPLY$. We
then measure the time to between the first $FLOW\_MOD$ sent and the last
$BARRIER\_REPLY$ received. As expected, the runtime was roughly linear with the
number of switches in the network. The figure also shows that the processing
time for large networks (5 seconds per simulator round) was well within the
bounds for interactive use.

\begin{figure}[t]
    %\hspace{-10pt}
    \includegraphics[width=3.25in]{../graphs/scalability_graph/scale.pdf}
    \caption[]{\label{fig:scalability} Time to send and process messages
    between controller and simulated switches. Each datapoint consists of
    $x^3/4$ hosts and $5x^2/4$ switches (\eg{} 48 pods means 27,468 hosts
    attached to 2,880 switches)}
\end{figure}

We also tested the extreme limits of the simulator's scalability, pushing up
the number of switches until something broke. We encountered what appears to be
a limitation of the Linux TCP/IP stack: TCP connection attempts began failing
beyond 26,680 sockets. Note that 26,680 switches is an order-of-magnitude larger than
the today's biggest networks.
}
