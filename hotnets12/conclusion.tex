%\colin{With respect to related work, we should note that we could apply our algorithm to
%other distributed systems (not just SDN controllers). It has always bothered
%me somewhat that this work is so strongly married to SDN.}

\eat{ SDN is widely heralded as the ``future of networking'', and its purpose is
to make it easy to manage networks. It does so by moving control plane
functionality out of network devices, and into a tightly-coupled cluster of
servers that provide a simple programmatic interface through which policies can
be specified. The challenges encountered in a tightly-coupled environment
(master-backup failover, state replication, sharding) are notably different from
traditional, fully distributed control planes, and thus present new
opportunities and challenges to the networking community. If we, the networking
community, have any desire to remain relevant, we must begin trying to
understand this set of challenges. }

SDN refactors the network control plane by moving the decision logic from the
forwarding devices to custom controller software running on separate servers.
These SDN controllers have evolved to large-scale, complex systems that are
layered both horizontally for separation of concerns and distributed vertically
for fault-tolerance and scalability. The distributed and/or replicated
controllers and the forwarding hardware (switches) form a large scale
distributed system with complex failure modes. With such complexity come complex
bugs.

On the other hand, SDN also presents an opportunity to attack the underdeveloped
field of network troubleshooting in a more systematic way. In this paper we
present an approach and early prototype of a system that solves an important
piece of the puzzle: By automatically identifying fault-inducing input from
execution logs, we allow developers to focus their efforts on
debugging the problematic code itself rather than reproducing the error in the
first place. We have demonstrated that our solution works on real, production
control software, despite noted limitations. We hope that our work inspires the
community to better understand the principles and mechanisms needed to construct
and operate the network control planes of the future.

% ============== ORIGINAL ===============
%We have applied this system to several available SDN platforms, and were able to find or reproduce bugs in all the platforms we investigated.
%Our paper seeks to improve the process of troubleshooting errors in the SDN
%platform. We focus on large, production networks,
%where controllers are distributed across multiple servers
%and the platform provides a virtualization abstraction to one or more tenants.
%We present two mechanisms, cross-layer correspondence checking
%and \simulator{}, designed to help troubleshooters automatically identify
%and isolate inconsistencies between high-level policy and low-level
%configuration. Our intent throughout this work is to convince
%the troubleshooting community to think outside of the NOX!
