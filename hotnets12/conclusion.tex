%\colin{With respect to related work, we should note that we could apply our algorithm to
%other distributed systems (not just SDN controllers). It has always bothered
%me somewhat that this work is so strongly married to SDN.}

SDN is widely heralded as the ``future of networking'', and its purpose is to make
it easy to manage networks. It does so by moving control plane functionality out of
network devices, and into a tightly-coupled cluster of servers that provide a simple
programmatic interface through which policies can be specified. As we have
learned in this work, the challenges encountered
in a tightly-coupled environment (master-backup failover, state replication, sharding) are
notably different from traditional, fully distributed control planes.

In this paper we have taken a stab at one facet of these challenges:
troubleshooting errors in control software. We developed a technique for automatically
identifying fault-inducing input from execution logs, with the
goal of allowing developers to focus their efforts on debugging the problematic
code itself rather than reproducing the error in the first place. We have
demonstrated that our solution works on real, production control software,
despite several noted limitations. We hope that our work inspires the community to
better understand the principles and mechanisms needed to construct and
operate the network control planes of the future.

% ============== ORIGINAL ===============
%We have applied this system to several available SDN platforms, and were able to find or reproduce bugs in all the platforms we investigated.
%Our paper seeks to improve the process of troubleshooting errors in the SDN
%platform. We focus on large, production networks,
%where controllers are distributed across multiple servers
%and the platform provides a virtualization abstraction to one or more tenants.
%We present two mechanisms, cross-layer correspondence checking
%and \simulator{}, designed to help troubleshooters automatically identify
%and isolate inconsistencies between high-level policy and low-level
%configuration. Our intent throughout this work is to convince
%the troubleshooting community to think outside of the NOX!
