Software-defined networking (SDN) proposes to simplify network management by
providing a simple logically-centralized API upon which network management
programs can be written.  However, the software used to support this API is
anything but simple: the SDN control plane (consisting of the network
operating system and higher layers) is a complicated distributed system that
must react quickly and correctly to link/controller failures, host migrations,
policy-configuration changes, and other events.
All complicated distributed systems are prone to bugs, and from our first-hand familiarity with several important open-source (NOX, POX, Floodlight, Frenetic, ONOS) and three major commercial closed-source controllers, we can attest that SDN control platforms are no exception.
% All complicated distributed systems are prone to bugs and from our first-hand familiarity with several important open-source (NOX, POX, Floodlight, Frenetic, ONOS) and closed-source controllers (Google, VMware, BigSwitch), we can attest that SDN control platforms are no exception. 

When faced with a bug, it is difficult to identify which events (\ie, which
set of link failures, controller failures, policy changes) are triggering the
bug; this act of ``troubleshooting'' (which precedes the act of debugging the
underlying code) is extremely time-consuming, as developers spend many hours poring
over multigigabyte execution traces.\footnote{A recent study suggested that software developers in general spend roughly half (49\% according to one
study~\cite{msoft_concurrency}) of their time troubleshooting, and spend
considerable time on bugs that are difficult to trigger (the same study found
that 70\% of the reported concurrency bugs take days to months to fix).} Bugs
are a fact of life; our aim is to make the troubleshooting process easier.

To that end, we have built a troubleshooting system for the SDN control plane.
Given a large execution trace, our tool attempts to automatically eliminate
trace events that are not causally related to the bug, producing a ``Minimal
Causal Sequence'' (MCS) of triggering events. Once the original trace has
been reduced to an MCS (or an approximation thereof), the developer embarks
on the debugging process. Our intent is that the greatly reduced size of the trace makes it easier for the developer to figure out which code path might be responsible for the underlying bug.

Our goal of minimizing test cases is in the spirit of delta debugging~\cite{Zeller:1999:YMP:318773.318946}, but our
problem is harder in two dimensions: our system is distributed in space (\ie,
spread across many machines) and the input is distributed in time (\ie, the
input is not a single file, but a sequence of events spread over time). This
completely changes the nature of the required troubleshooting infrastructure,
and we are not aware of any other system that tackles these challenges (though
we discuss the vast amount of related work in \S\ref{sec:related_work}).

Our troubleshooting system, which we call {\projectname} for \projectmeaning,
consists of 21,000 lines of Python, and is designed to
integrate with existing QA test infrastructure (discussed in
\S\ref{sec:architecture}). We
have used \projectname~to analyze bugs in \num{five} of the major open-source SDN
controllers, as well as one closed-source controller. We first evaluate
our approach by inserting synthetic bugs (\ie, we insert known
bugs, generate test traces that trigger those bugs, and then examine the MCSes that
result), and then demonstrate its viability by analyzing \num{six} newly discovered bugs.
