%\begin{aquote}{Steven Moffat}
%``People assume that time is a strict progression of cause to effect,
%but {\em actually}, from a non-linear, non-subjective viewpoint--it's more like a
%big ball of wibbly wobbly... time-y wimey... stuff.''
%\end{aquote}

Would World War \rom{1} still have occurred if Archduke Ferdinand had not been
shot? %Would the Ebola virus have spread to humans if Mabalo Lokela had not
%first contracted it?
That is, was the prior event intrinsic to the
historical outcome, or
was it extraneous? Unfortunately we can never know such
historical counterfactuals.

When troubleshooting computer systems, we often need to answer similar questions,
\eg~``Was this routing loop triggered when the controller discovered the link failure?''
And unlike human history, it is often possible to retroactively find explanations for
`how we got into this mess'. In this paper we address the problem of programmatically
finding such explanations in the context of software-defined networking.

%Software bug diagnosis, a highly expensive and time consuming process, would benefit
%greatly from technology for answering such questions. A 2006 survey found
%that developers at Microsoft spend 49\% of their
%time troubleshooting bugs~\cite{msoft_concurrency}. The same study found that 70\% of the reported concurrency bugs
%take days to months to fix, and 74\% of
%respondents considered reproducibility to be hard or very hard.

Based on anecdotal evidence from colleagues and acquaintances in the industry, it seems clear that
developers of SDN controller software spend much of their time
troubleshooting bugs. This should be no surprise, since software developers in
general spend roughly half (49\% according to one
study~\cite{msoft_concurrency}) of their time troubleshooting, and spend
considerable time on bugs that are difficult to trigger
(the same study found that 70\% of the reported concurrency bugs
take days to months to fix).
More fundamentally though, modern SDN controller software is highly complex,
distributing state between replicated
servers~\cite{floodlight},
providing isolation and resource arbitration between multiple
tenants~\cite{Casado:2010:VNF:1921151.1921162}, and
globally optimizing network utilization~\cite{urs}.
Most of this complexity comes from
fundamentally difficult distributed systems challenges such as asynchrony and
partial failure. Even Google's Urs H$\ddot{\mathrm{o}}$elzle
attests that~\cite{urs} ``[coordination between replicated controllers] is going to
cause some angst, and justifiably, in the industry.''
%As SDN control software
%is used to manage larger and more sophisticated networks,
%we expect its complexity to increase.
% Scott: needs more SDN war stories from VMW, BSN

The troubleshooting process, which starts by analyzing long logs gathered from either operational monitoring or QA testing
(we focus on the latter), is hindered by the large number of hardware failures,
policy changes, host migrations, and other inputs to SDN control software.
As one data point, Microsoft Research
reports 8.5 network error events per minute per
datacenter~\cite{Greenberg:2009:VSF:1592568.1592576}.
Troubleshooters find little immediate use from logs containing many inputs
prior to a fault,
since they are forced to manually filter extraneous inputs
before they can start fruitfully investigating the root cause.
It is no surprise that when asked to describe their
ideal tool, most network admins said ``automated troubleshooting''~\cite{Zeng:Survey}.

%And despite the prevalence of failures, maintaining uptime is of critical importance; according to one
%survey, ``meeting customer expectations about reliability'' is the second highest
%priority for networking professionals~\cite{market_report}.
% Software bugs cost $59 billion in 2002: bit.ly/V2Z7Al

Before continuing, we should clarify what we mean by `troubleshooting' and `bugs' in the SDN context.
SDN networks are designed to support high-level policies, such as inter-tenant
isolation (\ie~ACL enforcement). A bug creates situations
where the network violates one or more of these high-level policies; that is, even though the control plane
has been told to implement a particular policy, the resulting configuration (\ie~flow entries in the switches)
does not do so properly. %We call this an {\em invalid} configuration.
These violations could arise from a hardware failure or a policy misconfiguration, and there is active work on
developing tools to detect these problems \cite{hsa,Zeng:2012:ATP:2413176.2413205}. However, such violations
could also arise from bugs in the SDN control plane software; our focus is on
this case.

Bugs may be triggered by uncommon sequences of inputs, such as a simultaneous link failure or controller reboot.
The act of {\em troubleshooting} involves identifying which set of inputs is most directly responsible for triggering the bug.
{\em Debugging} is then the act of tracking down the error in the code itself, given a
set of triggering inputs.
The smaller the set of triggering inputs, the easier debugging will be.

Our focus here is on troubleshooting. When we observe an invalid
configuration,
which is prima facie evidence for a bug, our goal is
to automatically filter out inputs to the SDN software
that are not relevant to triggering the bug, leaving a small sequence of inputs
that is directly responsible.
This would go a long way towards achieving ``automated troubleshooting.''

If you consider a software-defined network as a distributed state machine,
with individual processes sending messages between themselves, one straightforward approach is
to account for potential causality: if an external input does not induce any messages before
the occurrence of the invalid configuration, it
cannot possibly have affected the outcome~\cite{Lamport:1978:TCO:359545.359563}.
Unfortunately, pruning only those inputs without a potential causal relation to
the invalid configuration does not
significantly reduce the number of inputs.

Our approach is to prune inputs from the original run, replay the remaining
inputs to the control software using simulated network devices, and
check whether the network re-enters the invalid configuration.
Our approach is identical in spirit to delta
debugging~\cite{Zeller:1999:YMP:318773.318946}---an algorithm for
minimizing test cases that are inserted at a single
point in time to a single program---but
involves a substantially different set of technical challenges since networks are
distributed asynchronous systems rather than a single program.

The main difficulty in pruning historical inputs is
coping with divergent histories. Traditional replay
techniques~\cite{Dunlap:2002:REI:844128.844148,Geels:2006:RDD:1267359.1267386}
reproduce errors reliably by precisely recording the low-level I/O operations of
software under test. Pruning inputs, however, may cause the execution to
subtly change (\eg~the sequence numbers of packets may all differ), and some
state changes that occurred in the original
run may not occur. Without the exact same low-level I/O operations,
deterministic replay techniques cannot proceed in a sensible manner.

We deal with divergent histories by recording and replaying at the application layer,
where we have access to the syntax and semantics of messages passed
throughout the distributed system. In this way we can recognize functionally
equivalent messages and maintain causal dependencies throughout replay despite
altered histories.

Painstaking manual analysis of logs is the
{\em de facto} method of troubleshooting production SDN control software
today. As far as we know, our work
is the first to programmatically isolate fault-inducing inputs to a distributed
system. Record and replay techniques such as
OFRewind~\cite{ofrewind} and liblog~\cite{Geels:2006:RDD:1267359.1267386}
allow you to manually step through the original execution and verify whether a
set of inputs triggered a bug,
but the original run is often so large that the
the set of potentially triggering inputs verges on unmanageable.
Tracing tools such as ndb~\cite{handigol2012debugger} provide
a historical view into dataplane (mis)behavior. In contrast, our technique provides
information about precisely what caused the network to
enter an invalid configuration in the first place.

We have applied \simulator~to four open source SDN control
platforms: Floodlight~\cite{bigswitch}, ONOS~\cite{ONOS}, POX~\cite{pox}, and NOX~\cite{nox}.
We describe several case studies of bugs we found in these platforms. Of the \num{five} bugs we encountered in a five day investigation, \simulator~reduced the size of the input trace to 2 events in the best case
and 18 events in the worst case. We compare the time taken to troubleshoot
the bugs with and without minimized traces to show the value in \simulator.

% Main result from our work: the amount of effort it takes to codify domain knowledge is significantly less than the effort
% it would take to step through a normal replay debugger by hand for each bug.

%% Cut?
%The rest of this paper is organized as follows. In \S\ref{sec:overview},
%we present a brief overview of the SDN control plane and its failure modes.
%In \S\ref{sec:approach} we present our approach in detail.
%We discuss use cases for in \S\ref{sec:architecture},
%and describe case studies and a performance evaluation in
%\S\ref{sec:evaluation}. We end by discussing the limitations of our
%approach (\S\ref{sec:discussion}) and related work (\S\ref{sec:related_work}).

% ------------------------------------------------------------------- %
%    OLD OSDI TEXT:
\eat{

% Feedback from Ali Ghodsi:
%  - Why is S/W easier to test than H/W? It's not really this point that's
%    important either: it's the centralized control plane that makes it easier
%    to test. [rebuttal: cite Urs Hoelzle]
%  - SDN is not more challenging! (read: SDN doesn't not offer a challenge)
%    Not problematic that the difficulty hasn't chanfed -- just make it clear
%    that we aren't getting farther away from the distributed nature of
%    networks.
%  - Need to NAIL the motivation:
%      - Distributed systems are fundamentally hard to troubleshoot, and
%      they're going to remain that way forever.
%      - SDN provides an opportunity to do better than traceroute! In
%      traditional networks, configuration state is spread all over the place.
%      - Not trivial in SDN (aren't running away from distributed nature)
%        It presents a new challenge: how to verify high-level policies have
%        actually been implemented correctly? => enter correspondence checking.
%      - What's the key insight here?
%  - Say what W^3 stands for?
%  - Too much boiler plate
%  - How much do the readers learn from reading the introduction
%  - "It is important to place this work in context and scope the problem we are
%   attacking" is really lecturing the reader  ...  of course context and scope of
%   any research problem is important.

\colin{``Meeting Customer Expectations about Reliability'' is the second highest
priority for networking professionals, according to a market
report~\cite{market_report}}

Given their critical role in enterprises, one might expect that networks
would come with a well-developed suite of troubleshooting tools. However,
the unfortunate truth is that {\tt traceroute}, developed in 1987~\cite{traceroute},
remains the network administrator's most sophisticated diagnostic mechanism. This reflects
the lack of structure in network control planes, which are an ad hoc mixture of
distributed protocols and manual configuration that directly manipulate forwarding tables. It is hard to tell what is broken
when desired behavior is only implicitly expressed in the routing entries themselves, and that is the case with today's networks.

In this respect, the emergence of Software-Defined
Networks (SDN) provides both an opportunity and a challenge. Moving control logic out of hardware and into software
enables enables concise policy specifications and significantly more sophisticated testing and troubleshooting tools. Moreover, since SDN
is still in its infancy (compared to traditional networking approaches), we as a community have an opportunity to make diagnostic tools
a more integral part of the overall design process. Although SDN's goal is to simplify the
{\em management} of networks, the challenge is that the SDN software stack itself is a complex distributed system,
operating in asynchronous, heterogeneous, and failure-prone environments.
In this paper we present our approach to troubleshooting bugs in SDN control
software, a tool called \projectname{}.

\projectname{} simplifies the troubleshooting process by programmatically localizing the root cause
of network problems along three dimensions: {\it what} network problems exist at a
given point in time, {\it where} in the control software a problem first developed, and
{\it when} the triggering event occurred. To accomplish this
\projectname{} employs two techniques:

\noindent{\bf Correspondence Checking}. We observe that the structure of the
SDN software stack (as we discuss in the next section) enables a straightforward algorithm for
checking that control applications' policies are implemented correctly in
the physical network. Our algorithm enumerates all correctness violations
(\ie{}, any instance where the high-level policies are not properly
implemented by the SDN control software) present in the network at a given point in
time, providing a crisp determination of the range of inputs
and the system component(s) responsible for the fault.

\noindent{\bf \Simulator{}.} \projectname{} can replay the execution of the
system against
a stream of network events (e.g., link failures). This allows us to (i)
distinguish between correctness violations that are harmless and quickly heal and
those that are persistent, and (ii) identify the minimal sequence of events that triggered
the correctness violation.

% OSDI "Tips":
%  - Make very clear that we implemented it!
%  - Make very clear how this is an improvement over related work!
%  - What has the system's use showed about the practical implications of the
%    ideas?
%  - What have you learned from the work?
%  - How generally applicable are these ideas?
%  - What were the alternatives considered at various points?
%  - Keep forward references (to later sections) to a minimum

In combination, correspondence checking and \simulator{} programmatically
localize software-faults in the SDN software stack.
With \projectname{}, operators and developers are free to focus their efforts
on debugging the code itself, without needing to
diagnose the symptoms in the first place.
% This represents an improvement over the state-of-the-art SDN troubleshooting
% practices

It is important to place this work in context and scope the problem we are
attacking. First, \projectname{} is a troubleshooting tool, not a debugger; by
this we mean that \projectname{} helps identify and localize network
problems (what, where, and when?), but it does not help identify exactly which
line of code causes the error (why?). Second, \projectname{} is focused on the
system software of the SDN stack (described in the next section). While progress has been made in troubleshooting control
applications that run on top of the SDN platform~\cite{nice} and in troubleshooting the forwarding tables in the physical switches~\cite{anteater,hsa}, we
are not aware of previous troubleshooting work that focuses on the SDN
platform itself; to the best of our knowledge, painstaking analysis of detailed logs is the current state-of-the-art in SDN platform troubleshooting.

In evaluating \projectname{} on
three popular SDN platforms---Frenetic, Floodlight and POX---we found or reproduced several bugs:
isolation breaches and faulty failover logic between distributed controllers.
We also demonstrated the feasibility of deploying
\projectname{} on production networks,
finding that our tools can enumerate all correctness violations in
simulated networks exceeding 25,000 hosts in under 5 seconds.

The rest of this paper is organized as follows. In \S\ref{sec:overview},
we present an overview of the SDN stack and its failure modes.
In \S\ref{sec:approach} we present correspondence checking and
\simulator{} in detail. We discuss the design of \projectname{} in
\S\ref{sec:architecture}. In \S\ref{sec:evaluation} we present
three bugs found by \projectname{}, as well as a performance evaluation.
Finally, in \S\ref{sec:related_work} we discuss related work,
and in \S\ref{sec:conclusion} we conclude.

\colin{Reviewer OE seemed to think it was negative that we focus on
corner-case scenarios. We need to provide a stronger argument for why focus on
these}

\eat{
 % Feedback on version 2.0:
 % not clear what you mean by "SDN platform", "correctness violation"
\subsection{\colin{Feedback from Kay et al.}}
\begin{itemize}
\item Not sure that OSDI PC members will understand the subtleties of this statement: ``it is hard to tell what is broken when the goals are only implicitly expressed in the flow entries themselves.''
\item I think we should push a little harder on emphasizing that the bugs are really nasty, and fundamental to the distributed nature of the system. "may have bugs" seems a little weak.
\item We should make it very clear how our tools are an improvement over the status quo (log analysis). In particular, we automatically localize the (i) layer, and (ii) minimal causal set of events responsible for an error.
\item \simulator{} does two things: find the minimal causal set, {\bf and}
differentiates benign from pernicious. Only one of these points was clear, not
both.
\item The Google quote seems to indicate that the problem has already been solved!
\item Along a similar vein, the point about ``the SDN platform will eventually
become stable'' makes it seems like these problems aren't fundamental. But
these problems are fundamental! How will the community view this paper in 10
years?
\item The reader doesn't walk away from this problem thinking ``This problem
seems really heard!''. What are the specific bugs this framework addresses?
Also, make it clear why this problem is important! If Pepsi knows that
Coca-cola might be able to snoop on their traffic, we've lost huge \$\$\$.
\item Don't frame ourselves just against log analysis. Make
the shortcomings of the status quo very clear, and be very precise about how
we address those shortcoming. (Why aren't traditional distributed systems
techniques apropro? Why is static checking of a single layer insufficient? How
is \simulator{} different than traditional replay debugging?)
\item Would be cool to have numbers on how many log events developers have to
step through today.
\item Would also be cool to have numbers on how many benign correctness violations
there are at a given point (say, in POX).
\item ``Policy-violation'' wasn't well-defined.
\item The introduction (prose) was too dense. Frame the problem in terms that
the reader understands. Get them thinking about the problem right off the bat.
Then lead them through our reasoning of how we came to the solution. Present
a straw-man solution, and show why that doesn't work. This way
the reader is much more engaged -- the reader should ``nod along''.
\item What exactly are we simulating? Hardware faults? Misconfigurations?
(answer: all of the above)
\item How exactly does correspondence checking provide a crisp determination of the scope of a correctness violation in the
network as well as the range of inputs that produce it? (answer: you start off
with a customer complaint, ``A can't ping B''. Correspondence checking tells
you all possible inputs that would experience the blackhole, and what parts of
the network are effected)
\item ``Stack'' connotes TCP/IP. Maybe use ``Platform'' instead?
\item ``Fault'' connotes hardware failure. Maybe use ``Software Fault''
\item Cite Frenetic, POX, Floodlight?
\item What does \projectname{} stand for?
\end{itemize}
}

}
