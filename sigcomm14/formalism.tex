We represent the forwarding state of the network
at a particular time as a configuration $c$, which contains all the forwarding
entries in the network
as well as the liveness of the various network elements.
The control software is a system consisting of one or more controller processes
that takes a sequence of external network events
$E = $\chain{\external{1},\external{2}}{\external{m}}
(\eg~link failures) as inputs,
and produces a sequence of network configurations
$C = c_1,c_2,\dots,c_n$. % Note that the network configuration $c$ does not
%include the internal state of the control software.

An invariant is a predicate $P$ over forwarding state (a safety
condition, \eg~loop-freedom). We say that configuration
$c$ violates the invariant if $P(c)$ is false, denoted $\overline{P}(c)$.

We are given a log $L$ generated
by a centralized QA test orchestrator.\footnote{We discuss how these logs are generated in \S\ref{sec:systems_challenges}.\label{fn:log_gen}}
The log $L$ contains a sequence of events
\setlength{\belowdisplayskip}{0.3pt} \setlength{\belowdisplayshortskip}{0.3pt}
\setlength{\abovedisplayskip}{0pt} \setlength{\abovedisplayshortskip}{0pt}
\begin{align*}
\tau_L = \text{\doublechain{\external{1},\internal{1},\internal{2},\external{2}}{\external{m}}{\internal{p}}}
\end{align*}
which includes external events
$E_L = $\eventlist{\external{1},\external{2}}{\external{m}}
injected by
the orchestrator, and internal events
$I_L = $\eventlist{\internal{1},\internal{2}}{\internal{p}}
triggered by the control software (\eg~OpenFlow messages).
The events $E_L$ include timestamps $\{ (\text{\external{k}}, t_k) \}$ from the
orchestrator's clock. % of when the test orchestrator injected them.

A replay of log $L$ involves replaying the external events $E_L$, possibly
taking into account the occurrence of internal events $I_L$.
We denote a replay attempt by $replay(\tau)$.
The output of $replay$ is a sequence of forwarding configurations
$C_R = \hat{c}_1,\hat{c}_2,\dots,\hat{c}_n$. Ideally $replay(\tau_L)$
reproduces the original sequence of configurations, but as we discuss later
this does not always hold.

If the configuration sequence $C_L = c_1,c_2,\dots,c_n$ associated with the
log $L$ violated predicate $P$
(\ie~$\exists_{c_i \in C_L}. \overline{P}(c_i)$)
then we say $replay(\cdot) = C_R$ reproduces that violation
if $C_R$ contains an equivalent faulty configuration
(\ie~$\exists_{\hat{c}_i \in C_R}. \overline{P}(\hat{c}_i)$).

The goal of our work is, when given a log $L$ that exhibited an
invariant violation,\footref{fn:log_gen} to find a small, replayable sequence of events that reproduces that
invariant violation. Formally, we define a minimal causal sequence (MCS)
to be a sequence $\tau_M$ where the external events $E_M \in \tau_M$ are a
subsequence of $E_L$ such
that $replay(\tau_M)$ reproduces the invariant violation, but for all proper
subsequences $E_N$ of $E_M$
there is no sequence $\tau_N$ such that $replay(\tau_N)$ reproduces the violation.
Note that an MCS is not necessarily {\em globally} minimal, in that there could be smaller
subsequences of $E_L$ that reproduce this violation, but are not a subsequence of this MCS.

We find approximate MCSes by deciding
which external events to eliminate and, more importantly, when to inject external
events. We describe this process in the next section.
%The key component of this system is a \tester~that can execute $replay()$.
%Because we do not have access to operational QA or production logs, our focus is on
%using the \tester~to generate random inputs (shown in
%Table~\ref{tab:inputs}), detecting bugs in control software,
%and then finding MCSes that trigger them.
