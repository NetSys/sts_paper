The ecosystem of troubleshooting tools for computer networks is desperately
sparse; relatively few innovations have taken place since the invention of
{\tt traceroute} in 1987~\cite{traceroute}. The emergence of software-defined
networking (SDN), however, is now enabling a greatly improved testing and debugging
cycle. Consider a recent quote from Google's Urs H\"oelzle~\cite{urs}:
% TODO: the article is not by Urs -- need to get real quote from ONS. Also,
% just paraphrase instead of quote?

\begin{quote}
One of the strongest benefits with SDN for Google has been the robust
testing capabilities available with SDN. H\"olzle reported that the
transition from traditional to OpenFlow networks was nearly seamless and
they encountered far fewer challenges than expected. Much of what made
this possible was the ability to very easily simulate backbone-scale
network environments virtually, including the ability to mirror
production event streams in testing environments, which allowed Google
to identify and fix all the bugs in advance.
\end{quote}

% Could boil this down to one sentence...
Software-defined networks are nonetheless highly complex distributed systems.
Modern platforms perform replication, resource arbitration, failure recovery, and network 
virtualization at datacenter scale. In this environment the SDN platform must be robust to
hardware failures, message drops, churn, and a plethora of other issues
inherent to distributed systems.

As in any software system, complexity increases the probability of
bugs. SDN platform developers' hope is that the platform will become
stable eventually, such that operators can safely assume correctness of lower layers.
Nevertheless, the evolution of SDN is far too nascent to provide stability
today. When network operators encounter erratic behavior in their network
they are currently forced to manually trace through
multiple layers of abstraction: policy-specification, virtualization logic,
distribution logic, and network devices. 

\colin{troubleshooting != debugging. Troubleshooting is about clarifying
symptoms. Debugging assumes a well-defined symptom, and finds the root cause.}

Besides {\it ad-hoc} measurement tools,
the predominant troubleshooting mechanism in SDN is
log analysis: manually specifying log statements at relevant points throughout the system;
collecting and aligning distributed log files; and analyzing the
results {\it post-hoc} when a error is encountered in production. Log analysis
is an important component of the troubleshooting process, but it is deficient
as a stand-alone solution: logs events
are enormous in number, often at the wrong level of granularity,
and yield a partial ordering of the system execution at best.

Recent work has contributed troubleshooting techniques focused on the highest (control
application) and lowest (dataplane forwarding tables) layers of the SDN stack.
NICE applies concolic execution and model checking to SDN control
applications, thereby automating the testing process and catching bugs before
they are deployed~\cite{nice}. Anteater~\cite{anteater} and HSA~\cite{hsa}
introduce mechanisms for checking static invariants in the dataplane.
Nonetheless, no automated troubleshooting mechanism exists for problems that span
multiple layers of the SDN stack, specifically those caused by bugs in the platform
itself.

%The utility of new programming models is heavily dependent on
%the usability of their troubleshooting mechanisms.
%Similarly, we think it highly undesirable to deploy SDN-based
%networking without a viable troubleshooting paradigm. 

% poor transition. Also, our work is not brought out until the second page...
In this paper we present \projectname{}, a framework for troubleshooting
network policy violations caused by bugs in the
SDN platform. \projectname{} localizes the root cause
of policy-violations along two dimensions:
component, and time.

\noindent{\bf Correspondence Checking}. We observe that the structure of the
SDN platform (graphs at every layer) enables a straightforward algorithm for
checking that control applications' policies are implemented correctly in
the physical network. Our algorithm, `correspondence checking',
enumerates all policy-violations present in the network at a given point in
time, providing a crisp determination of the scope of a policy-violation in the
network as well as the range of inputs that produce it. Moreover,
by checking correspondence between intermediate layers of the
SDN platform, \projectname{} is able to localize the particular component responsible 
for a given policy-violation.


% Need more buildup to this component
\noindent{\bf \Simulator{}.} Like any distributed
system, transient policy-violations due to failures and delays are 
common, especially at large scale. We present \simulator{}, a semi-automated
technique to differentiate pernicious and benign policy-violations.
By iteratively replaying the execution of
the system and pruning out events that are not causally related to pernicious
policy-violations, \simulator{} localizes the minimal causal set of network events
leading up to a policy-violation. 

In evaluating \projectname{} on
three controller platforms---Frenetic, Floodlight, and POX---we find \colin{N} bugs,
including isolation breaches,
faulty failover logic, and consistency problems between replicated
controllers. We further demonstrate the feasibility of deploying
correspondence checking and \simulator{} on production networks,
finding that correspondence checking can enumerate all policy-violations in a
fat-tree network of 100,000 hosts in under \colin{N} seconds.
% Need to say something about scalability of simulator as well

The rest of this paper is organized as follows. In \S\ref{sec:overview},
we present an overview of the SDN stack and its failure modes.
In \S\ref{sec:approach} we present correspondence checking and
\simulator{} in detail. In \S\ref{sec:evaluation} we present
a use-case and a preliminary performance evaluation.
Finally, in \S\ref{sec:related_work} we discuss related work,
and in \S\ref{sec:conclusion} we conclude.

\subsection{Feedback from Kay et al.}

\begin{itemize}
\item \simulator{} does two things: find the minimal causal set, {\bf and}
differentiates benign from pernicious. Only one of these points was clear, not
both.
\item The Google quote seems to indicate that the problem has already been solved!
\item Along a similar vein, the point about ``the SDN platform will eventually
become stable'' makes it seems like these problems aren't fundamental. But
these problems are fundamental! How will the community view this paper in 10
years?
\item The reader doesn't walk away from this problem thinking ``This problem
seems really heard!''. What are the specific bugs this framework addresses?
Also, make it clear why this problem is important! If Pepsi knows that
Coca-cola might be able to snoop on their traffic, we've lost huge \$\$\$.
\item Don't frame ourselves just against log analysis. Make 
the shortcomings of the status quo very clear, and be very precise about how
we address those shortcoming. (Why aren't traditional distributed systems
techniques apropro? Why is static checking of a single layer insufficient? How
is \simulator{} different than traditional replay debugging?)
\item Would be cool to have numbers on how many log events developers have to
step through today.
\item Would also be cool to have numbers on how many benign policy-violations
there are at a given point (say, in POX).
\item ``Policy-violation'' wasn't well-defined.
\item The introduction (prose) was too dense. Frame the problem in terms that
the reader understands. Get them thinking about the problem right off the bat.
Then lead them through our reasoning of how we came to the solution. Present
a straw-man solution, and show why that doesn't work. This way
the reader is much more engaged -- the reader should ``nod along''.
\item Too many big words, \eg{} ``nascent''
\item What exactly are we simulating? Hardware faults? Misconfigurations?
(answer: all of the above)
\item How exactly does correspondence checking provide a crisp determination of the scope of a policy-violation in the
network as well as the range of inputs that produce it? (answer: you start off
with a customer complaint, ``A can't ping B''. Correspondence checking tells
you all possible inputs that would experience the blackhole, and what parts of
the network are effected)
\item ``Stack'' connotes TCP/IP. Maybe use ``Platform'' instead?
\item ``Fault'' connotes hardware failure. Maybe use ``Software Fault''
\item Cite Frenetic, POX, Floodlight?
\item What does \projectname{} stand for?
\end{itemize}
