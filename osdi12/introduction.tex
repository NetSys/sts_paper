Given their critical role in enterprises, one might expect that networks
would come with a well-developed suite of troubleshooting tools. However,
the unfortunate truth is that {\tt traceroute}, developed in 1987~\cite{traceroute},
remains the network administrator's most sophisticated diagnostic mechanism. This reflects
the lack of structure in network control planes, which are an ad hoc mixture of
distributed protocols and manual configuration that directly manipulate the forwarding tables; it is hard to tell what is broken
when desired behavior is only implicitly expressed in the routing entries themselves.

In this respect, the emergence of Software-Defined
Networks (SDN) provides both an opportunity and a challenge. SDN moves control logic out of hardware and into software, where
flexibility enables significantly more sophisticated testing and troubleshooting tools. Moreover, since SDN
is still in its infancy (compared to the more traditional networking approaches which have been
used for decades), we (as a community) have an opportunity to make diagnostic tools
a more integral part of the overall SDN design process. The challenge is that although SDN's goal is to simplify the
{\em management} of networks, the SDN platform itself is a complex distributed system, operating in asynchronous, heterogeneous, and failure-prone environments.
In this paper we present one approach to troubleshooting the SDN platform, a tool called \projectname{}.

It is important to place this work in context and scope the problem we are attacking.  First, \projectname{} is a troubleshooting tool, not a debugger; by this we mean that \projectname{} helps identify and localize the network problems, but it does not help  identify exactly which line of code causes the error. Second, \projectname{} is focused on the SDN platform itself. While progress has been made in troubleshooting control
applications that run on top of the SDN platform \cite{nice} and in troubleshooting the forwarding tables in the physical switches \cite{anteater,hsa}, we  
are not aware of previous troubleshooting work that focuses on the SDN platform; to the best of our knowledge, painstaking analysis of detailed logs is the current state-of-the-art in SDN platform troubleshooting. 

\projectname{} localizes the root cause
of network problems along two dimensions: where in the SDN stack, and when the triggering event occurred. To accomplish this, 
\projectname{} employs two basic techniques:

\noindent{\bf Correspondence Checking}. We observe that the structure of the
SDN platform (as we discuss in the next section) enables a straightforward algorithm for
checking that control applications' policies are implemented correctly in
the physical network. Our algorithm enumerates all policy-violations (i.e., any instance where the policies are not properly implemented in the network) present in the network at a given point in
time, providing a crisp determination of the scope of a policy-violation in the
network and the particular SDN layer responsible 
for it.

\noindent{\bf \Simulator{}.} \projectname{} can replay the execution of the SDN platform against
a stream of network events (e.g., link failures). This allows us to (i) distinguish between policy-violations that are harmless and quickly heal and those that are persistent and/or pernicious, and (ii) identify the minimal set of events that triggered
the policy violation. \\

In combination, correspondence checking and \simulator{} automatically localize software-faults in the SDN platform.
With \projectname{}, operators and developers are free to focus their efforts on debugging the code itself, without the need to 
painstakingly diagnose the symptoms in the first place.

In evaluating \projectname{} on
three controller platforms---Frenetic, Floodlight, and POX---we find \colin{N} bugs,
including isolation breaches,
faulty failover logic, and consistency problems between replicated
controllers. We further demonstrate the feasibility of deploying
correspondence checking and \simulator{} on production networks,
finding that correspondence checking can enumerate all policy-violations in a
fat-tree network of 100,000 hosts in under \colin{N} seconds.
% Need to say something about scalability of simulator as well

The rest of this paper is organized as follows. In \S\ref{sec:overview},
we present an overview of the SDN stack and its failure modes.
In \S\ref{sec:approach} we present correspondence checking and
\simulator{} in detail. In \S\ref{sec:evaluation} we present
a use-case and a preliminary performance evaluation.
Finally, in \S\ref{sec:related_work} we discuss related work,
and in \S\ref{sec:conclusion} we conclude.

\eat{
\subsection{\colin{Feedback from Kay et al.}}
\begin{itemize}
\item Not sure that OSDI PC members will understand the subtleties of this statement: ``it is hard to tell what is broken when the goals are only implicitly expressed in the flow entries themselves.''
\item I think we should push a little harder on emphasizing that the bugs are really nasty, and fundamental to the distributed nature of the system. "may have bugs" seems a little weak.
\item We should make it very clear how our tools are an improvement over the status quo (log analysis). In particular, we automatically localize the (i) layer, and (ii) minimal causal set of events responsible for an error.
\item \simulator{} does two things: find the minimal causal set, {\bf and}
differentiates benign from pernicious. Only one of these points was clear, not
both.
\item The Google quote seems to indicate that the problem has already been solved!
\item Along a similar vein, the point about ``the SDN platform will eventually
become stable'' makes it seems like these problems aren't fundamental. But
these problems are fundamental! How will the community view this paper in 10
years?
\item The reader doesn't walk away from this problem thinking ``This problem
seems really heard!''. What are the specific bugs this framework addresses?
Also, make it clear why this problem is important! If Pepsi knows that
Coca-cola might be able to snoop on their traffic, we've lost huge \$\$\$.
\item Don't frame ourselves just against log analysis. Make 
the shortcomings of the status quo very clear, and be very precise about how
we address those shortcoming. (Why aren't traditional distributed systems
techniques apropro? Why is static checking of a single layer insufficient? How
is \simulator{} different than traditional replay debugging?)
\item Would be cool to have numbers on how many log events developers have to
step through today.
\item Would also be cool to have numbers on how many benign policy-violations
there are at a given point (say, in POX).
\item ``Policy-violation'' wasn't well-defined.
\item The introduction (prose) was too dense. Frame the problem in terms that
the reader understands. Get them thinking about the problem right off the bat.
Then lead them through our reasoning of how we came to the solution. Present
a straw-man solution, and show why that doesn't work. This way
the reader is much more engaged -- the reader should ``nod along''.
\item What exactly are we simulating? Hardware faults? Misconfigurations?
(answer: all of the above)
\item How exactly does correspondence checking provide a crisp determination of the scope of a policy-violation in the
network as well as the range of inputs that produce it? (answer: you start off
with a customer complaint, ``A can't ping B''. Correspondence checking tells
you all possible inputs that would experience the blackhole, and what parts of
the network are effected)
\item ``Stack'' connotes TCP/IP. Maybe use ``Platform'' instead?
\item ``Fault'' connotes hardware failure. Maybe use ``Software Fault''
\item Cite Frenetic, POX, Floodlight?
\item What does \projectname{} stand for?
\end{itemize}
}
