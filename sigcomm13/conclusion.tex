SDN is widely heralded as the ``future of networking'', and its purpose is to make
it easy to manage networks. It does so by moving control plane functionality out of
network devices, and into a tightly-coupled cluster of servers that provide a simple
programmatic interface through which policies can be specified. As we have
learned in this work, the challenges of maintaining virtualized and distributed
views in a failure-prone environment are
notably different from the challenges encountered in traditional,
fully distributed control planes.

In this paper we have taken a stab at one facet of these challenges:
troubleshooting errors in control software. We developed a technique for automatically
identifying fault-inducing input from execution logs, with the
goal of allowing developers to focus their efforts on debugging the problematic
code itself rather than reproducing the error in the first place. We have
have applied this system to several available SDN platforms, and were able to find or reproduce bugs in all the platforms we investigated,
despite several noted limitations. We hope that our work inspires the community to
better understand the principles and mechanisms needed to construct and
operate the network control planes of the future.
