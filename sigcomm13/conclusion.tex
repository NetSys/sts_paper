SDN is widely heralded as the ``future of networking'', and its purpose is to make
it easy to manage networks. Achieving this end forces platform developers to directly
confront asynchrony, partial failure, and other problems that are inherent to all distributed
systems and notoriously difficult to get right.

In this paper we developed a technique for automatically
identifying a minimal sequence of inputs responsible for triggering a given bug.
We have have applied this system to several available SDN platforms, and
were able to find or reproduce bugs in all the platforms we investigated.
%Now we just need to provide a mechanism for the next question: would that date
%have panned out if I hadn't spilt the wine?

% Two ideas that have been put on the backburner:
% - distinguishing persistent violations from transient
% - using correspondence checking to localize the layer where the bug first
%   manifests

%We chose SDN as our domain becuase X,Y,Z. We envision a new paradigm where
%domain knowledge is applied to debugging in all systems.

% ------------------------------------------- %
%             OLD TEXT
\eat{
It does so by moving control plane functionality out of
network devices, and into a tightly-coupled cluster of servers that provide a simple
programmatic interface through which policies can be specified. As we have
learned in this work, the challenges of maintaining virtualized and distributed
views in a failure-prone environment are
notably different from the challenges encountered in traditional,
fully distributed control planes.

}
