%\begin{aquote}{Steven Moffat}
%``People assume that time is a strict progression of cause to effect,
%but {\em actually}, from a non-linear, non-subjective viewpoint--it's more like a
%big ball of wibbly wobbly... time-y wimey... stuff.''
%\end{aquote}

Would World War \rom{1} still have occurred if Archduke Ferdinand had not been
shot? Would the Ebola virus have spread to humans if Mabalo Lokela had not
first contracted it?
That is, were those prior events intrinsic to the
historical outcome, or
were they extraneous? Unfortunately we can never know such
historical counterfactuals for sure.

When troubleshooting computer systems, we often need to answer similar questions,
\eg~``Was this routing loop triggered when the controller discovered the link failure?''
And unlike human history, it is often possible to retroactively find explanations for
`how we got into this mess'. In this paper we address the problem of programmatically
finding such explanations in the context of software-defined networking.

%Software bug diagnosis, a highly expensive and time consuming process, would benefit
%greatly from technology for answering such questions. A 2006 survey found
%that developers at Microsoft spend 49\% of their
%time troubleshooting bugs~\cite{msoft_concurrency}. The same study found that 70\% of the reported concurrency bugs
%take days to months to fix, and 74\% of
%respondents considered reproducibility to be hard or very hard.

Based on anecdotal evidence from colleagues and acquaintances in the industry, it seems clear that
developers of software-defined networks spend much of their time
troubleshooting bugs. This should be no surprise, since software developers in
general spend roughly half (49\% according to one
study~\cite{msoft_concurrency}) of their time troubleshooting, and spend
considerable time on bugs that are difficult to trigger
(the same study found that 70\% of the reported concurrency bugs
take days to months to fix).
More fundamentally though, modern SDN control platforms are highly complex,
distributing state between replicated
servers~\cite{floodlight},
providing isolation and resource arbitration between multiple
tenants~\cite{Casado:2010:VNF:1921151.1921162}, and
globally optimizing network utilization~\cite{urs}.
Most of this complexity comes from
fundamentally difficult distributed systems challenges such as asynchrony and
partial failure. Even Google's Urs H$\ddot{\mathrm{o}}$elzle, a leading
networking and distributed systems technologist,
attests that~\cite{urs} ``[coordination between replicated controllers] is going to
cause some angst, and justifiably, in the industry.''
%As SDN control software
%is used to manage larger and more sophisticated networks,
%we expect its complexity to increase.
% Scott: needs more SDN war stories from VMW, BSN

The troubleshooting process is hindered by the large number of hardware failures,
policy changes, host migrations, and other inputs to SDN control software.
As one data point, Microsoft Research
reports 8.5 network error events per minute per
datacenter~\cite{Greenberg:2009:VSF:1592568.1592576}.
Troubleshooters find little immediate use from traces containing many inputs
prior to a fault,
since they are forced to manually filter extraneous inputs
before they can start fruitfully investigating the root cause.
It is no surprise that when asked to describe their
ideal tool, most network admins said ``automated troubleshooting''~\cite{Zeng:Survey}.

%And despite the prevalence of failures, maintaining uptime is of critical importance; according to one
%survey, ``meeting customer expectations about reliability'' is the second highest
%priority for networking professionals~\cite{market_report}.
% Software bugs cost $59 billion in 2002: bit.ly/V2Z7Al

Before continuing, we should clarify what we mean by `troubleshooting' and `bugs' in the SDN context.
SDN networks are designed to support high-level policies, such as inter-tenant
isolation (\ie~ACL enforcement). A bug, in this context, creates situations
where the network violates one or more of these high-level policies; that is, even though the control plane
has been told to implement a particular policy, the resulting configuration (\ie~flow entries in the switches)
does not do so properly. We call this an {\em invalid} configuration.
We presume that the control plane functions properly in most circumstances, so
that these policy violations are rare.
% and are typically triggered when the control plane takes an unusual codepath.
Bugs may be triggered by uncommon sequences of inputs, such as a simultaneous link failure or controller reboot.
The act of troubleshooting involves identifying which set of inputs triggered the bug.
Debugging is then the act of tracking down the error in the code itself, given a
set of triggering inputs.
The smaller the set of triggering inputs, the easier debugging will be.

Our focus here is on troubleshooting. When we observe an invalid
configuration,
which is prima facie evidence for a bug, our goal is
to automatically filter out inputs to the SDN software (\eg~link failures)
that are not relevant to triggering the bug, leaving a small sequence of inputs
that is directly responsible.
This would go a long way towards achieving ``automated troubleshooting.''

\eat{ % Not enough space...
If you consider a software-defined network as a distributed state machine,
with individual processes sending messages between themselves, one straightforward approach is
to account for potential causality: if an external input does not induce any messages before
the occurrence of the invalid configuration, it
cannot possibly have affected the outcome~\cite{Lamport:1978:TCO:359545.359563}.
Unfortunately, pruning only those inputs without a potential causal relation to
the invalid configuration does not
significantly reduce the number of inputs.
}

Our approach is to prune inputs from the original run, replay the remaining
inputs to the control software using simulated network devices, and
check whether the network re-enters the invalid configuration.
%In particular, we generalize delta
%debugging~\cite{Zeller:2002:SIF:506201.506206}---an algorithm for
%minimizing test cases that are inserted at a single
%point in time to a single program---to a distributed environment, where inputs
%are spread across time and involve multiple machines. \scott{I think you give
%too much credit to delta debugging}
The main difficulty in pruning historical inputs is
coping with divergent histories. Traditional replay
techniques~\cite{Dunlap:2002:REI:844128.844148,Geels:2006:RDD:1267359.1267386}
reproduce errors reliably by precisely recording the low-level I/O operations of
software under test. Pruning inputs, however, may cause the execution to
subtly change (\eg~the sequence numbers of packets may all differ), and some
state changes that occurred in the original
run may not occur. Without the exact same low-level I/O operations,
deterministic replay techniques cannot proceed in a sensible manner.

We deal with divergent histories by recording and replaying at the application layer,
where we have access to the syntax and semantics of messages passed
throughout the distributed system. In this way we can recognize functionally
equivalent messages and maintain causal dependencies throughout replay despite
altered histories.

Painstaking manual analysis of logs is the
{\em de facto} method of troubleshooting production SDN control software
today. As far as we know, our work
is the first to programmatically isolate fault-inducing inputs to a distributed
system. Record and replay techniques such as
OFRewind~\cite{ofrewind} and liblog~\cite{Geels:2006:RDD:1267359.1267386}
allow you to manually step through the original execution and verify whether a
set of inputs triggered a bug,
but the original run is often so large that the
the set of potentially triggering inputs verges on unmanageable.
Tracing tools such as ndb~\cite{handigol2012debugger} provide
a historical view into dataplane (mis)behavior. In contrast, our technique provides
information about precisely what caused the network to
enter an invalid configuration in the first place.

We have applied our approach to three open source SDN control
platforms: Floodlight~\cite{bigswitch}, POX~\cite{pox}, and NOX~\cite{nox}.
We present three case studies of bugs we have found in each of
these controllers, and illustrate how the minimized sequences our system found
were valuable for finding the root causes of the bugs.

In the next section we formally define the problem of isolating fault-inducing
inputs to distributed systems. We then outline our approach in
\S\ref{sec:approach}, and the
system we built to realize our approach in \S\ref{sec:systems_challenges}.
We end by describing preliminary case studies that illustrate how minimized
input traces help software developers fix bugs.
