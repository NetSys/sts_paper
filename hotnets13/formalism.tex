
We represent the forwarding state of the network
at a particular time as a configuration $c$, which contains all the forwarding
entries in the network
as well as the liveness of the various network elements.
The network control plane is a system that takes a sequence of
external network events $E = e_1,e_2,\dots,e_m$ (such as link failures) as input,
and produces a sequence of network configurations
$C = c_1,c_2,\dots,c_n$.

An invariant is a predicate $P$ over forwarding state (a safety
condition, such as having no loops or blackholes). We say that a configuration
$c$ violates the invariant if $P(c)$ does not
hold, denoted as $\overline{P}(c)$.

In our formalism a log $L$ of a system execution is a triplet of external events $E_L$,
timing information $T_L$ of the external events, and the resulting sequence of forwarding
configurations $C_L$.
A replay of log $L$ involves replaying the external events along with a
particular timing $T$,
which need not be identical to the original timings $T_L$ captured in the log.
We denote a replay attempt by $replay(E_L,T)$.
The output of $replay$ is a sequence of forwarding state configurations
$C_R = \hat{c}_1,\hat{c}_2,\dots,\hat{c}_n$. In the ideal case $replay(E_L,T_L)$ reproduces the same
sequence of network configurations as occurred in the original run
(\ie~$\forall_i. c_i = \hat{c}_i$), but as we discuss later in the paper there are reasons why
this does not always hold.

If the original log $(E_L, T_L, C_L)$ violated predicate $P$
(\ie~$\exists_{c_i \in C_L}. \overline{P}(c_i)$)
then we say $replay(E_L,T) = C_R$ reproduces that invariant violation if
$\exists_{\hat{c}_i \in C_R}. \overline{P}(\hat{c}_i)$.

The goal of our work is, when given a log $(E_L, T_L, C_L)$ that violates a predicate $P$, to find a small sequence of events that reproduces that
invariant violation.  More formally, we define a minimal causal sequence (MCS)
to be a subsequence $E_M$
of $E_L$ and a timing $T_M$ such
that $replay(E_M,T_M)$ reproduces the invariant violation, but for all proper
subsequences $E_N$ of $E_M$
there is no timing $T$ s.t. $replay(E_N,T)$ reproduces the violation.
That is, an MCS is a sequence and timing of external events that reproduces the violation,
where one cannot find a subsequence of the MCS that reproduces the violation.
Note that an MCS is not necessarily {\em globally} minimal, in that there could be smaller
sequences that reproduce this violation, but are not subsequence of this MCS.
\colin{Problem with this definition: our approach does not provably find
MCSes, since we don't explore all timings!}

Given a log $(E_L, T_L, C_L)$ exhibiting an invariant violation,
our goal is to identify its MCS. We could do this through brute force, trying
all subsequence and timings, but this is clearly impractical given the large size of logs and the
infinitely large set of possible timings. Thus, we must approximate MCSes by finding good heuristics
for which events to eliminate and, more importantly, which timings to test.

In the next section we describe our heuristics for finding MCSes, and then describe the system we built in \S\ref{sec:systems_challenges} to implement these heuristics.
The key component of this system is a network simulator that can execute the $replay()$ function.
Because we do not have access to operational logs, all of our focus is on
discovering bugs in control software using this
simulator and then finding the relevant MCSes that trigger these bugs.
