We represent the forwarding state of the network
at a particular time as a configuration $C$.
An invariant is a predicate $P$ over forwarding state (a safety
condition). We say that forwarding
state $C$ violates an invariant if $P(C)$ does not
hold, also denoted as $\overline{P}(C)$.

Log $L$ consists of external events $E_L$ and observed internal events $I_L$,
along with a timing $T_L$ of the observed events.
We assume that all changes to forwarding state can be inferred
from internal events (\eg~OpenFlow message sends), so
that at all times we have knowledge of the relevant forwarding state.
Therefore we can determine from $L$ whether $P$ failed to hold at any point.
% It might make the notation more concise if we just include the sequence of
% forwarding configurations as part of L. We have access to this information
% in practice, since we're driving the fuzzing.

A replay of a log means replaying the external events along with a particular timing (or interleaving) $T$.
We denote this by $replay(E_L,T)$.
The output of $replay$ is a sequence of forwarding state configurations
$C_1,C_2,\dots,C_n$.
We say that a replay reproduces an invariant violation if
$\exists_{C_i \in replay(E,T)}\:s.t.\:\overline{P}(C_i)$.
In the ideal case $replay(E_L,T_L)$ reproduces the orignal invariant violation
asociated with $L$ (as long as there is no nondeterminism). \\

\noindent~Troubleshooting process:

\begin{outline}
\1 Start with log $L$ where $P$ did not hold at some point. Could be operational or fuzzed log.

\1 Definition: an MCS is a subsequence $E_M$ of $E_L$ and a timing $T_M$ such
that $replay(E_M,T_M)$ reproduces the invariant violation, but for all proper
subsets $E_L$ of $E_M$
there is no timing $T$ s.t. $replay(E_L,T)$ reproduces the violation.

\1 Assumption 1: naturally occurring external logs $E_L$ are large.

\1 Assumption 2: the MCS of most natural logs resulting in violations are much smaller than the original log.  That is, bugs are causally sparse.
\end{outline}

\noindent~Technical challenge: Find approximations to MCSs.

\begin{outline}
\1 Dumb approach: explore all timings $T$ for each subset $E$.

\1 Slightly smarter approach: use delta debugging, causality, and equivalence to limit set of timings $T$ you need to look at.
\end{outline}
