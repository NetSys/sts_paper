\subsection{Definitions}

We represent the forwarding state of the network
at a particular time as a configuration $C$ (this state contains all the forwarding entries, as well as the liveness of the various network elements).
The network control plane is a system that takes a sequence of
external network events $E$ (such as link failures) that occur at times $T$
as input and produces a sequence of network configurations
$C_1,C_2,\dots,C_n$.

An invariant is a predicate $P$ over forwarding state (a safety
condition, such a having no loops or complete reachability). We say that a configuration
$C$ violates an invariant if $P(C)$ does not
hold, also denoted as $\overline{P}(C)$.

One can obtain logs of the control plane in action by instrumenting operational networks or running QA tests.  In our formalism a Log $L$ consists of external events $E_L$
along with timing information $T_L$ of the external events. {\bf perhaps make a log the triplet of E, T, C}
A replay of log $L$ involves replaying the external events along with a particular timing $T$ which need not be identical to the original timings in the captured log.
We denote this by $replay(E_L,T)$.  
The output of $replay$ is a sequence of forwarding state configurations
$\hat{C}_1,\hat{C}_2,\dots,\hat{C}_n$. In the ideal case $replay(E_L,T_L)$ reproduces the same sequence of network configurations as occurred in the original run (i.e., $C_i = \hat{C}_i$), but as we discuss later in the paper there are reasons why this does not always hold.

If the original network run produced a violation (i.e., $\overline{P}(C_i)$ for some$i$), then we say replay reproduces that invariant violation if
$\exists_{\hat{C}_i \in replay(E,T)}\:s.t.\:\overline{P}(\hat{C}_i)$.

Definition: an MCS is a subsequence $E_M$ of $E_L$ and a timing $T_M$ such
that $replay(E_M,T_M)$ reproduces the invariant violation, but for all proper
subsets $E_L$ of $E_M$
there is no timing $T$ s.t. $replay(E_L,T)$ reproduces the violation.
\\

\subsection{Assumptions}

Assumption 1: naturally occurring  logs $E_L$ are large.

Assumption 2: the MCS of most natural logs resulting in violations are much smaller than the original log.  That is, bugs are causally sparse.

Troubleshooting is the act of finding small logs that reproduce problems, so that developers can debug.

\subsection{Technical Challenge: Approximating MCSes}

{\bf clean this up, and use it to outline what we do in the next section....}

\begin{outline}
\1 Dumb approach: explore all timings $T$ for each subset $E$.

\1 Slightly smarter approach: use delta debugging, causality, and equivalence to limit set of timings $T$ you need to look at.  And hints for timing, etc....

\1 Practical difficulties: nondeterminism, etc.
\end{outline}
