\label{subsec:non_goals}

Having detailed the specifics of our approach and illustrated some use cases, we now
clarify the scope of our technique's use.

\noindent{\bf Partial Visibility.} Our event scheduling algorithm assumes that
it has visibility into the occurrence of all relevant internal events. In
practice many relevant internal state changes are already marked by logging
statements, but developers may need to add additional
logging statements to ensure reliable replay.

\noindent{\bf Non-determinism Within Individual Controllers.} Our tool is not designed to reproduce bugs
involving non-determinism within a single controller (\eg~race-conditions between threads);
we focus on coarser granularity errors (\eg~incorrect failover logic). The upshot of
this is that our technique is not able to minimize all possible failures.
Nonetheless, the worst case for us is that the developer ends up with what they started:
an unpruned log.

\eat{That said, if the developer is willing to instrument their system to
provide finer granularity log messages (\cf~\cite{Geels:2006:RDD:1267359.1267386}),
our approach readily supports deterministic replay.}

\noindent{\bf Troubleshooting vs.\ Debugging.} Our technique is a troubleshooting tool, not a debugger;
by this we mean that our approach helps identify and localize inputs that
trigger erroneous behavior, but it does not directly identify which
line(s) of code cause the error.

\noindent{\bf Bugs Outside the Control Software.} Our goal is not to find the root
cause of individual component failures in the system (\eg~misbehaving routers,
link failures). Instead, we focus on
how the distributed system as a whole reacts to the occurrence of such inputs.
If there is a bug in your switch, you will need to contact your hardware vendor;
if you have a bug in your policy specification, you will need to take a closer look at what you specified.

\noindent{\bf Globally vs.\ Locally Minimal Input Sequences.}
Our approach is not guaranteed to find the globally minimal
causal sequence from an input trace, since this requires $O(2^N)$ computation in the worst case.
The delta debugging algorithm we employ does provably find a
locally minimal causal sequence~\cite{Zeller:1999:YMP:318773.318946},
meaning that if any input from the sequence is pruned, no invariant violation
occurs. \colin{Need to mention monotonicity, consistency, unambiguity here}

\noindent{\bf Correctness vs.\ Performance.}
We are primarily focused on correctness bugs, not performance bugs.

\noindent{\bf Complexity}
The overall runtime of delta debugging with replays is
$O(nlog\,n)$ in the number of replayed inputs in the best case,
and $O(n^2)$ in the worst case. We are currently exploring techniques such as
parallelization and snapshotting for improving the overall runtime.

\colin{Add: relevence of bugs found by fuzzing.}

\eat{
\noindent{\bf Proactive vs.\ Reactive Configuration.} We focus primarily on
\emph{proactive} configuration, where controllers react to policy and topology changes, but
not necessarily individual packets or flows events in the
dataplane.\footnote{Production controllers typically adopt this model for
performance reasons.}
The main challenge in extending our approach to reactive controllers is
achieving efficient simulation of dataplane traffic.
\andi{Could cut this. We actually find reactive bugs}
}

