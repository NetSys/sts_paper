SDN is widely heralded as the ``future of networking'', because it makes it
much easier for operators to manage their networks. SDN does this, however, by
pushing complexity into SDN control software itself. Just as sophisticated
compilers are hard to write, but make programming easy, SDN platforms make
network management easier for operators, but only by forcing the developers of
SDN platforms to confront the challenges of asynchrony, partial failure, and
other notoriously hard problems that are inherent to all distributed systems.
%Thus, people will be troubleshooting and debugging SDN control software for many
%years to come, until it becomes as stable as compilers are now.

Current techniques for troubleshooting SDN control software are quite primitive; they
essentially involve manual inspection of logs in the hope of identifying the
relevant inputs. In this paper we developed a technique for automatically
identifying a minimal sequence of inputs responsible for triggering a given
bug. We believe our technique will be especially valuable for troubleshooting
distributed controllers running complex applications, which are just now
starting to make their way into the public domain.

\colin{Should maybe place more emphasis on this work "fosters discussion of
broad research agendas for the networking community as a whole" and
"identifies
fundamental open questions"}
