We have applied \projectname~to the routing applications\footnote{At the time
we conducted these experiments, routing was among the most complex
SDN applications publicly available. We are currently investigating
distributed controllers with more complex applications such as network virtualization,
which have recently become publicly available. We believe bugs involving
distribution are particularly amenable to techniques such as ours.} of three open
source SDN control platforms:
POX~\cite{pox}, NOX~\cite{nox}, and Floodlight~\cite{bigswitch}. We describe
bugs we found in each of these controllers to illustrate
how the minimized traces our technique produced were valuable in
understanding the bugs' root causes.

\subsection{POX In-Flight Blackhole}
After roughly 20 runs of randomly generated inputs,
we detected a persistent blackhole while
POX was bootstrapping its
discovery of link and host locations in a small 2-switch, 2-host network.
There were 29 inputs in the initial trace, and \projectname~returned an 11 input
MCS.

We provided the MCS to the lead developer of POX. Primarily using the
console output, we were able to trace through the code and identify the problem
within 7 minutes, and were able to find a fix for the race condition within 40
minutes. By matching the console output with the code, he found that the crucial
triggering events were two
in-flight packets (set in motion by prior traffic injection events):
POX first incorrectly learned a host location as a result of the first in-flight
packet showing up immediately after POX discovered that that port belonged to
a switch-switch link -- apparently the code had not accounted for the
possibility of in-flight packets directly following link discovery -- and
then as a result the
second in-flight packet
POX failed to return out of a nested conditional that would have
otherwise prevented the blackholed routing entries from being installed.

Although the initial trace was fairly small to begin with, knowing that every
input in the MCS was necessary for triggering the race condition allowed us to
avoid the
possibility of being mislead by extraneous diagnostic information while
tracing through the many potentially faulty code paths.

\subsection{NOX Discovery Loop}

We tested NOX on a four-node mesh, and discovered a
routing loop between three switches
within
roughly 20 runs of randomly generated inputs.

Our initial input size was 68 inputs, and
\projectname~returned an 18 input MCS.\footnote{We had difficulty replaying
the final MCS. The reason seems clear: this bug depends on the
order NOX discovers links, which in turns depends on hashes computed from random memory
addresses its discovery modules LLDP selection algorithm!}
Our approach to debugging was to
reconstruct from the MCS how NOX should have installed routes, then compare
how NOX actually installed routes.

The order in which NOX discovered links was crucial: at the point NOX
installed the 3-loop, it had only discovered one link towards the destination.
Therefore all other switches routed through the one known neighbor switch.
This comprised 2 of the 3 links involved in the link.

The destination host only sent one packet, which caused NOX to initially learn
its correct location. After NOX flooded the packet though, it became confused
about its location. One flooded packet arrived at
another switch that was currently not known to be attached to anything, so NOX
incorrectly concluded that the host had migrated. Other flooded packets were
dropped as a result of link failures in the network and randomly generated
network loss. The loop was then installed when the source injected another
packet.

This case took us roughly 10 hours to debug. We are confident that without the
minimized trace, it would have taken much
longer to trace through the subtle sequence of events that were necessary for
setting up the network in precisely the right conditions.

\subsection{Floodlight Forwarding Loop}

We applied \projectname~to Floodlight's routing application.
In about 30 minutes, our fuzzing uncovered a
117 input sequence that caused a persistent 3-node forwarding loop.
\projectname~reduced the sequence to 13 input events in 324 replays and 8.5 hours.

We repeatedly replayed the 13 event MCS, while successively adding
instrumentation and increasing the log level each run. After about 15 replay
attempts, we found that the problem was caused by interference of end-host
traffic with ongoing link discovery packets. In our experiment, Floodlight had
not discovered an inter-switch link due to dropped LLDP packets, causing an
end-host to flap between perceived attachment points.

While this behavior cannot strictly be considered a bug in Floodlight,
the case-study nevertheless highlights the benefit of
\projectname~over traditional fuzzing techniques: by leveraging repeated replays
of a significantly reduced MCS, we were able to diagnose the root cause--a
complex interaction between the LinkDiscovery, Forwarding, and DeviceManager
modules.
