Software-defined networking (SDN) restructures the network control plane
by introducing a `network operating system` (NOS). The NOS provides
a logically centralized view of the network on top of which control
applications can operate. This enables a more structured approach to
networking, facilitates concise specfication of intended behavior,
and decouples the control logic from the distribution of control-plane
state in the physical network.  However, the additional complexity
introduced by the NOS can also increase the difficulties of debugging
network problems: Correlating the high level specification
of intended behavior and its physical manifestation through multiple
layers of indirection is challenging. Additionally, bugs may manifest not only
in the control application, but in any of the layers of abstraction
that translate control decisions to changes in the physical network:
virtualization logic, distribution logic, network devices, or the mappings
between any of these layers. In this paper we present \projectname{},
a cross-layer debugger for software-defined networks. \projectname{}
interposes on the layers of the SDN stack, both within a single control
server and between distributed controllers, to enable isolated invariant
checking of individual components. Using bug reports from a production
SDN deployment, we demonstrate that \projectname{} can detect and isolate
many real problems in large-scale production networks today.
