\documentclass{sig-alternate-10pt}
%\documentclass[letterpaper,11pt]{article}
\usepackage{url}
%\usepackage{usenix,epsfig,endnotes}
%\usepackage{fullpage} 
%\setlength{\textwidth}{6.75in}
%\setlength{\oddsidemargin}{-.125in}
\usepackage{graphicx}

\usepackage{natbib}
%\usepackage{subfigure}
%\usepackage{ifpdf}
%\usepackage{multicol}
%\usepackage{amsmath, amssymb, amsthm}
%\usepackage{rotating}
%\onehalfspacing
%\newcommand{\tbd}[1]{[{\bf{#1}}]}
\newcommand{\tbd}[1]{}
\newcommand{\ie}{{\it i.e.}}
\newcommand{\eg}{{\it e.g.}}
\newcommand{\etc}{{\it etc.}}
\newcommand{\eat}[1]{}

\usepackage[usenames,dvipsnames]{color}
\newcommand{\justine}[1]{{\color{ForestGreen}\bf JS: {#1}}}
\newcommand{\andi}[1]{{\color{blue}\bf AW: {#1}}}
\newcommand{\colin}[1]{{\color{Red}\bf CS: {#1}}}

%\setlength\topmargin{0in}
%\usepackage{verbatim}
%\usepackage[compact]{titlesec}
%\usepackage[small]{caption}
\usepackage{times}

\title{CLINT: Cross-layer Debugging for the Software-Defined Networking Stack\vspace{-25pt}}

\author{Colin Scott and Justine Sherry\thanks{In collaboration with Andreas
Wundsam, Teemu Koponen, and Scott Shenker}}

%        \begin{multicols}{2}{{\it Draft - Please do not distribute.}}\\
        
%\author{Paper \#69, 14 Pages}
\date{}
\begin{document}
    \maketitle
% Outline:
%
%1 Problem formulation
%  1.1 Why is the problem important/relevant?
%  1.2 What are the main challenges in solving the problem?
%
%2 Approach
%  2.1 What is your approach? Briefly describe it
%  2.2 How is you approach different/similar from/to related work? You
%      don't need to be exhaustive here.
%
%3 Deliverables/milestones
%   3.1 What do you plan to accomplish by the end of the semester? What are the
%   metrics of success?
%   3.2 Do you plan to work on the project past the end of the semester?, and
%   if yes, what do you hope to accomplish by the end?
%   3.3 Specify measurable milestones by Oct. 26th
%  
%4 Any special needs for the project?

\abstract{
{\it
    Software-defined networking (SDN) simplifies network management
    by presenting a global, abstract model of network state to 
    management applications, hiding the low-level details of 
    individual switch configurations. \andi{More important than abstract: logically centralized model, 
    \ldots isolating the control logic from the distribution of control-plane state in the network\ldots}
    However, by providing administrators with virtualized abstractions, debugging
    system errors becomes considerably more difficult, as bugs may be 
    present in the control application, the virtualization in the network operating system, 
    the physical switch implementations, or the mappings between any of these layers. \andi{Distance to the metal: Difficult to correlated failure
    occurrence and responsible fault through multiple layers of abstraction}.
    In this paper, we present CLINT, a multilayer debugger for software defined
    networks. CLINT takes advantage of clean interfaces between layers of the SDN stack by interposing between each layer to check for 
    violations of expected invariants from each component of the network controller.
    Using bug reports from production SDN deployments, we demonstrate that
    CLINT can detect and isolate many real problems that encumber networks today.
}
\justine{Am I right that CLINT doesn't yet handle the inter-controller stuff? I know you wrote a basic replicated controller, but not sure if you have any checks yet. Anyway, if so, we should mention that too...}
\colin{It would be really cool to run the inter-controller stuff by Monday,
mostly just because it makes the problem space way more interesting. We
should be able to just run your application on top of the replicated
controller and call it good.  Of course, we'll see
what happens between now and then.}
}

\section{Introduction}
    \label{sec:intro}
    The SDN platform's $raison\text{ }d'\hat{e}tre$ is to 
hide complexity from control applications. To this end, modern platforms perform
replication, resource arbitration, failure recovery, and network 
virtualization on the control application's behalf. 

While these measures are effective in simplifying control applications,
they do not remove any complexity from the overall system. Rather, they merely move the complexity
from control applications into the underlying SDN platform.

As in any software system, additional complexity increases the probability of
bugs. And unfortunately for the network operator, finding bugs in the platform requires
access to precisely the details hidden from the control application.
When operators encounters erratic behavior in their network, the error's root
cause may lie in their own policy specification, or in the SDN platform
itself. To deal with the latter case, they must trace through
multiple layers of abstraction: virtualization logic, distribution logic, and
network devices.

As it stands, the SDN platform provides meager support for troubleshooting.
The predominant troubleshooting method is log analysis: manually
specifying log statements at relevant points throughout the system,
collecting; gathering; and ordering distributed log files; and analyzing the
results {\it post-hoc} when a error is encountered in production. Besides its
apparent tediousness, this approach is lacking in several ways: logs events
are enormous in number, impossible to aggregate into a single serial
execution of the system, and often at the wrong level of granularity to be of
use.

Recent work has contributed much-needed improvements to the highest (control
application) and lowest (dataplane forwarding tables) levels of abstraction, 
but no principled troubleshooting mechanism exists yet for the SDN platform.
NICE applies concolic execution and model checking to SDN control
applications, thereby automating the testing process and catching bugs before
they are deployed~\cite{nice}. Aneater~\cite{anteater} and HSA~\cite{hsa}
introduce mechanisms for checking static invariants in the dataplane.

It would be unthinkable to introduce a new programming language without a
debugger. Similarly, we think it highly undesirable to deploy SDN-based
networking without a viable troubleshooting paradigm. 

Correctness of the SDN platform can be stated concisely: high-level policies
should correspond with low-level configuration. We observe that the structure
of the SDN platform, graphs at every layer, enables a straightforward
algorithm to check this invariant. Our algorithm, which we term correspondence checking,
enumerates all inconsistencies at any point in time and isolates the
root cause of an inconsistency to a particular component of the system.

In eventually-consistent systems such as software-defined
networks however, transient inconsistencies between network policy and actual network
behavior are an inevitable state-of-affairs.
In such an environment, it does not suffice for troubleshooting tools to
simply enumerate inconsistencies; they should also aid the developer
in identifying which are related to serious problems, and which are
harmlessly ephemeral. To this end we present \simulator.
\Simulator allows troubleshooters 
to sift out pernicious inconsistencies by tracking the life cycle of problems 
both forward and backward in time.

We have implemented prototypes
of correspondence checking and \simulator. Our code is publicly available
at~\cite{github}.

The rest of this paper is organized as follows. In \S\ref{sec:overview},
we present an overview of the SDN stack and its failure modes.
In \S\ref{sec:approach} we present correspondence checking and
\simulator in detail. In \S\ref{sec:evaluation} we present
two use-cases and a preliminary performance evaluation
Finally, in \S\ref{sec:related_work} we discuss related work,
and in \S\ref{sec:conclusion} we conclude.


\section{Nicira Bug Reports}

Teemu is a bamf. Let's hope his bug reports come in by Monday.

\section{Approach}

Key insights. We argue that invariants are the right way to think about this
problem. Delta over Anteater: need to detect dynamic configuration bugs.
OpenFlow networks are typically quite dynamic compared to traditional static
(campus) networks. Also need cross-layer debugger (correspondance checks) to isolate where the bug is,
and verify that

\section{Architecture}

Details that aren't ultimately that important.

\begin{enumerate}
\item Fuzzer
\item Invariant Checks
\item Correspondance Checks
\end{enumerate}

\section{Evaluation}

Three forms of evaluation:

\begin{enumerate}
\item Does our system capture the bugs described by Nicira?
\item Did our system find bugs in third-party applications?
\item Is the overhead of our system reasonable? For now, just a graph that
shows how long it takes to run Anteater in the fuzzer loop. Graphs look
pretty.
\end{enumerate}

\section{Discussion and Future Work}
\label{sec:future_work}
    We plan to continue developing \projectname{} with the eventual aim of publication; we plan to 
submit to SIGCOMM 2012. 
    Our plans for future work include the following:

    {\bf Detailed Bug Report Analysis.} Pending CTO permission, we've recently
     been granted access to raw bug reports from Nicira users running Onix in production. 
     We'd like to use this data to answer questions about commonly encountered SDN bugs: 
     Of the bug classifications developed in S\ref{sec:bugs}, which are the most common?    
     When administrators trace down the root cause of the bugs, where in the SDN stack are
     bugs most prevalent?

    {\bf Build and Evaluate Correspondence Checker.} Our existing implementation does not
    include the cross-layer correspondence checker described in \S\ref{sec:architecture}.

    {\bf Design, Build, and Evaluate Consistency Debugger.} We built a distributed controller for
    POX, but have yet to design and build a consistency debugger to detect and resolve bugs
    resulting from replication.~\footnote{We're interested in combining this with techniques to make
    programming consistency better by treating the whole network as a distributed system, \eg{} adding
    vector clocks to switches and deploying updates as transactions.}

    {\bf Improved input generator.} Our current implementation of the input generator is
    a random ``fuzzer.'' The fuzzer randomly explores the state space of possible flows, but
    we believe that more sophisticated techniques (high-fideity tracing, symbolic execution) can help
    us target input flows to more efficiently explore the space of possible bug-generating inputs. 

    {\bf Run on a real, rather than simulated network.} Our evaluation ran on a real POX implementation,
    but the physical network controlled by POX was a simulated, rather than real network.

    {\bf Distinguish between transient and persistent errors.}  Transient errors are very different than persistent errors. Both are
    important problems. But you attack those two problems very differently. Can we
    automatically distinguish the two? That is, can we infer whether a bug is
    triggered by a certain order of events in the system, versus a systematic bug
    with the NOS state machine?

    {\bf Reactive trobleshooting techniques.} \projectname{} passively monitors state of the 
    components of the SDN, but it could benefit from reactive techniques like issuing traceroutes
    when a bug is first detected; this supplemental information can aid in the debugging process.

    {\bf Better tools for administrators.} We plan to develop two techniques to help administrators use
    the debugger. First, we plan to allow them to specify their own, custom invariants (deployment-specific policies)
     included with the standard invariants (\eg{} loops, dead ends).
    Second, enabling and interactive debugging process, allowing administrators to step by step monitor
    packets traversing the network and inspect FIB entries and network state at each hop.



\section{Related work}
\label{sec:related_work}
This work extends a growing literature on troubleshooting tools for
Software-Defined Networks.
    
The work most closely related to ours is NICE~\cite{nice}. NICE combines concolic execution
and model checking to automate the process of testing NOX applications. This enables one to catch bugs before
they are deployed.  

Our approach and NICE complement each other in several ways.  First, NICE's systematic exploration of failure orderings 
is potentially of great use for finding corner-case errors, which we could then add to our regression suite. NICE may also be applied directly to the code-base of the SDN platform, but in the case that only a subset
of all possible code-paths in the SDN platform can be model-checked due to state-space explosion; 
our mechanisms allows users to troubleshoot errors 
{\it post-hoc} after they are observed in production, so we can find bugs that might be missed due to truncating the state-space exploration.
In complement to NICE, correspondence checking helps developers isolate the
specific component of the SDN platform responsible for an error, without needing to specify invariants.

Focusing on the physical network, Anteater~\cite{anteater} and HSA~\cite{hsa}
are alternative approaches to statically checking invariants in the
configuration of switches and routers. Both take take as input a snapshot of
the FIB of each network device. To check invariants, Anteater generates a set of constraint functions and feeds them through a SAT
solver, while HSA defines an algebra for virtual packets and
their transformation through the network. We leverage the HSA work in \projectname{}, and our simulator allows us to detect policy-violations not just in a given set of tables but what tables are produced by a wide range of scenarios. \

Also focusing on the physical network, OFRewind~\cite{ofrewind} develops
record and replay techniques for the control plane of OpenFlow networks.
Unlike \simulator, OFRewind focuses specifically on OpenFlow
interactions, while we focus on more course-grained replay of
failures and topology changes. Running replay within a simulator also allows
us to manually modify the execution of the system, rather than playing a
static recording. 

Another line of work aims to prevent bugs from being introduced in the first
place. Frenetic~\cite{frenetic} presents a language-based approach to building
robust SDN applications. By providing a specialized programming model,
 Frenetic helps developers avoid writing common classes of
bugs, such as `composition errors' where installed flow entries override each other.
Reitblatt et al.~\cite{consistentupdates} developed a technique for ensuring
consistent routing updates, guaranteeing that all switches in the network either route
a given packet under the new configuration or under the old configuration,
but not both. These abstractions are valuable for preventing common, difficult errors
in platform logic.

Several other network simulators exist for testing SDN controllers. Mininet is a 
platform for emulating OpenFlow switches and hosts within a single
 VM~\cite{Lantz:2010:NLR:1868447.1868466}. The ns-series of network simulators
provides a general framework for testing new protocols, topologies,
and traffic mixes~\cite{ns3}. We found that these existing simulators did
not provide sufficient support for the corner-cases situations which are the
focus of our work, such as failures and VM migration.

Many of our ideas originate from the literature on troubleshooting general
distributed systems. WiDS checker introduced the notion of recording
production executions to be later replayed and verified in a controlled simulation.
Pip~\cite{pip} defines a DSL and collection of annotation tools to
reason about causal paths throughout the execution of the
distributed system. Finally, end-to-end tracing
frameworks such as X-Trace~\cite{Fonseca:2007:XPN:1973430.1973450} and 
Pinpoint~\cite{Chen02pinpoint:problem} provide a framework for tracing requests throughout 
a distributed system in order to infer correctness errors between layers and
across components. Our work solves a more constrained problem; we leverage
the structure of the SDN stack to enable a simple notion of platform
correctness. In addition, these systems assume that invariants should hold at
all times; we observe that in an eventually-consistent system such as SDN,
transient policy-violations are inevitable. We built \simulator{} to help troubleshooters
differentiate ephemeral from persistent errors. 

% If we manage to run multiple applications by Monday, we should cite papers
% on consistency and cross-layer debugging:
%X-Trace~\cite{xtrace}
% Vector Clocks
% Onix
% Virtualization definitely won't happen by Monday. But, papers include
% Martin's presto '10 paper 'Virtualizaing the Network Forwarding Plane'



\section{Conclusion}

We debugged all the things!

\bibliographystyle{abbrv}
\bibliography{bib}

%\input{appendix}

\end{document}
