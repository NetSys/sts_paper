\documentclass{sig-alternate-10pt}
%\documentclass[letterpaper,11pt]{article}
\usepackage{url}
%\usepackage{usenix,epsfig,endnotes}
%\usepackage{fullpage} 
%\setlength{\textwidth}{6.75in}
%\setlength{\oddsidemargin}{-.125in}
\usepackage{graphicx}

\usepackage{natbib}
%\usepackage{subfigure}
%\usepackage{ifpdf}
%\usepackage{multicol}
%\usepackage{amsmath, amssymb, amsthm}
%\usepackage{rotating}
%\onehalfspacing
%\newcommand{\tbd}[1]{[{\bf{#1}}]}
\newcommand{\tbd}[1]{}
\newcommand{\ie}{{\it i.e.}}
\newcommand{\eg}{{\it e.g.}}
\newcommand{\etc}{{\it etc.}}
\newcommand{\eat}[1]{}

\usepackage[usenames,dvipsnames]{color}
\newcommand{\justine}[1]{{\color{ForestGreen}\bf JS: {#1}}}

%\setlength\topmargin{0in}
%\usepackage{verbatim}
%\usepackage[compact]{titlesec}
%\usepackage[small]{caption}
\usepackage{times}

% TODO: split this file into multiple .tex's

\title{Debugging Software Defined Networks with CLINT\vspace{-25pt}}

\author{Colin Scott and Justine Sherry\thanks{In collaboration with Andi
Wundsam and Scott Shenker}}

%        \begin{multicols}{2}{{\it Draft - Please do not distribute.}}\\
        
%\author{Paper \#69, 14 Pages}
\date{}
\begin{document}
    \maketitle
% Outline:
%
%1 Problem formulation
%  1.1 Why is the problem important/relevant?
%  1.2 What are the main challenges in solving the problem?
%
%2 Approach
%  2.1 What is your approach? Briefly describe it
%  2.2 How is you approach different/similar from/to related work? You
%      don't need to be exhaustive here.
%
%3 Deliverables/milestones
%   3.1 What do you plan to accomplish by the end of the semester? What are the
%   metrics of success?
%   3.2 Do you plan to work on the project past the end of the semester?, and
%   if yes, what do you hope to accomplish by the end?
%   3.3 Specify measurable milestones by Oct. 26th
%  
%4 Any special needs for the project?

\abstract{
{\it
    Software-defined networking (SDN) simplifies network management
    by presenting a global, abstract model of network state to 
    management applications, hiding the low-level details of 
    individual switch configurations.
    However, by providing administrators with virtualized abstractions, debugging
    system errors becomes considerably more difficult, as bugs may be 
    present in the control application, the virtualization in the network operating system, 
    the physical switch implementations, or the mappings between any of these layers.
    In this paper, we present CLINT, a multilayer debugger for software defined
    networks. CLINT takes advantage of clean interfaces between layers of the SDN stack by i    nterposing between each layer to check for 
    violations of expected invariants from each component of the network controller.
    Using bug reports from production SDN deployments, we demonstrate that
    CLINT can detect and isolate many real problems that encumber networks today.
}
\justine{Am I right that CLINT doesn't yet handle the inter-controller stuff? I know you wrote a basic replicated controller, but not sure if you have any checks yet. Anyway, if so, we should mention that too...}
}

\section{Introduction}
    \label{sec:intro}
    The SDN platform's $raison\text{ }d'\hat{e}tre$ is to 
hide complexity from control applications. Modern controllers perform
replication, resource arbitration, failure recovery, and network 
virtualization on the control application's behalf. 

Despite the abstractions provided by the SDN programming model,
software-defined networks are no less complex than traditional networks. The architectural goal of SDN is
simply to push complexity from the control application onto the underlying platform.

SDN control platforms are prone to bugs as a result of their complexity. Bugs in the
platform present an architectural tension: troubleshooting requires
access to precisely the same details hidden by the platform's abstractions.
When an application developer 
encounters erratic behavior in the network, they must trace their
policy specification through multiple layers of abstraction
preceding changes in the physical network: virtualization logic,
distribution logic, and network devices. The error's root cause
may manifest in any of these layers, not just the control application.

As it stands, the SDN platform provides meager support for troubleshooting.
The predominant troubleshooting method is log analysis: manually
specifying log statements at relevent points throughout the system,
collecting, gathering, and ordering distributed log files, and analyzing the
results {\it post-hoc} when a error is encountered in production. Besides its
apparent tediousness, this approach is lacking in several ways: logs events
are enormous in number, impossible to aggregate into a single serial
execution of the system, and often at the wrong level of granularity to be of
use. \colin{</ why it's hard>}

Recent work has contributed much-needed improvements to this state of affairs. 
NICE applies concolic execution and model checking to SDN control
applications, thereby automating testing and catching bugs before
they are deployed~\cite{nice}. Aneater~\cite{anteater} and HSA~\cite{hsa}
introduce mechanisms for checking static invariants in the dataplane.
Nevertheless, no troubleshooting mechanism exists yet for the SDN platform itself.

New operating system abstractions face an arduous path towards adoption
without sound troubleshooting mechanisms. Analogously, the success of the
SDN programming model depends heavily on the utility of its troubleshooting
paradigm. Our goal in this paper is to work towards a useful
troubleshooting mechanism for the SDN platform. \colin{</ why it's important>}

We observe that in eventually-consistent systems such as sofware-defined networks,
transient inconsistencies are an inevitable property of the system.
Consequently, the process of troubleshooting errors essentially boils down to
identifying relevant events amongst a clamor of inconsistencies and diagnostic
information.

We present two mechanisms designed to make it easier for operators and
developers to ``see through the noise'' of diagnostic information. The first,
cross-layer correspondence checking, leverages the structure of the SDN
architecture to enable a general and verifiable notion of platform
correctness. Correspondence checking allows troubleshooters to isolate the cause of 
an inconsistency to a particular layer without needing to define invariants or
instrument third-party code. Our second
mechanism, simulated replay analysis, allows troubleshooters 
to differentiating ephemeral from persisent inconsistencies by steering the
execution of the system forward and backward in time, filtering out extraneous
external events, and inducing uncommon events such as failures. \colin{</ what we did>}

The rest of this paper is organized as follows. In \S\ref{sec:overview},
we present an overview of the SDN stack and its failure modes.
In \S\ref{sec:approach} we present correspondence checking and simulated
replay analysis in detail. In \S\ref{sec:evaluation} we present
two use-cases of our techniques, as well as a preliminary evaluation
of their runtime. Finally, in \S\ref{sec:related_work} we discuss related work,
and in \S\ref{sec:conclusion} we conclude.


\section{Nicira Bug Reports}

Teemu is a bamf.

\section{Approach}

Key insights.

\section{Implementation}

Details that aren't ultimately that important.

\begin{enumerate}
\item Fuzzer
\item Invariant Checks
\item Correspondance Checks
\end{enumerate}

\section{Evaluation}

Two forms of evalution:

\begin{enumerate}
\item Does our system capture the bugs described by Nicira?
\item Did our system find bugs in third-party applications?
\end{enumerate}

\section{Discussion and Future Work}
    \label{sec:future_work}
    The scheduling heuristics we developed in the previous section have several
shortcomings. Most importantly, treating the software as a blackbox disallows us
from showing formal properties of the MCSes we produce. That is, we cannot
explain why our outputs are good approximations to truly minimal sequences, nor can we
explain how the heuristics helped. Second,
although we do not believe that the techniques are
specific to SDN control software, they currently pertain only to that specific
domain. Here we outline plans to address both of the these shortcomings.
At a high level, we plan to (i) start with an infeasible but provably correct
approach, and
(ii) find practical approximations to this approach, many of which will
involve leveraging empirical properties of (iii) different classes of distributed software
systems (beyond just SDN control software).

\noindent{\bf Showing Formal Properties of MCSes.} If we were to repeatedly
$replay$ a
fixed subsequence of external events $E_S$ to a blackbox distributed system,
it is possible that it would not to produce the same output on each
execution, even if we were to apply the heuristics outlined in
\S\ref{sec:past_work}. The issue is that we do not have full visibility into internal
events nor control over the internal scheduling decisions of the distributed
system. Our approach in~\cite{sts2014} was to
replay each subsequence multiple times, but this does not provide any
guarantees on the minimality of the MCSes we produce, since it's possible that
a subsequence $E_S$ that we thought did not reproduced the invariant violation
could have done so if the
distributed system had interleaved its internal events in some other way.

Suppose we want to show how close our outputs are to minimal. One approach would be to use a model checker to
{\em certify} whether each subsequence $E_S$ chosen by delta debugging does or
does not reproduce the original violation, thereby producing a provably
minimal MCS. The model checker could accomplish this
by systematically exploring all possible interleavings of internal events
given the fixed sequence of external events $E_S$.\footnote{Note that the
internal events are not fixed. That is, the schedule chosen by the model
checker affects which internal events are subsequently triggered by the software under
test.}

\noindent{\bf Working Around Impossibility.} In some cases, it is not possible for the model checker to certify a
given subsequence $E_S$. For example, suppose that the distributed system does
not terminate (or more specifically, suppose its state space is infinite
and non-recurring). In that case, model checking is not guaranteed to terminate.
Crucially, if the
model checker does not find a way to trigger the original violation in finite time,
that does not imply that the violation cannot be triggered, and we therefore lose our guarantees on minimality.

In distributed systems, non-termination means that an algorithm is not guaranteed to stop
sending messages. The formal term for this property is `non-quiescence`. In~\cite{aguilera1997heartbeat} Aguilera et al. prove that
all failure detectors---an important algorithmic component used by many distributed systems---are
non-quiescent.

As a fallback to non-quiescence, we can resort to
bounded model checking, where the model checker only explores the event
interleavings for a given subsequence $E_S$ up to a certain number of
execution steps. This approaches weakens the soundness (minimality) guarantees provided by
our approach in favor of the ability to operate on real distributed
systems implementations.

In some cases, we can also leverage the following observation to work around
non-quiescence:
given a failure detector component, it is possible to implement many other
distributed algorithms in a quiescent manner~\cite{aguilera1997heartbeat}. Building on this observation,
one might mark (non-quiescent) failure detector algorithms as `trusted-components`,
such that the model checker takes control of when the failure detector reports
failures and recoveries to the application, and systematically explores when
these failure/recovery reports are triggered in order to certify the remaining
(quiescent) components of the distributed system.

\noindent{\bf Making Certification Practical.} Although we only use the
model checker to find specific invariant violations rather than asking it to
find all invariant violations, it may nonetheless need to explore an
intractable number of event interleavings. This would make it impractical for
use on real distributed software.

We believe that we can ameliorate computational intractability by
leveraging the prior knowledge contained in the original failing test case.
That is, by making the $replay$ function stateful, we can develop heuristics that lead the
model checker to quickly find interleavings that trigger the original violation, so that
it is only forced to enumerate all interleavings for subsequence $E_S$ that do
not trigger the original violation.

As a concrete example, if we lead
the model checker to first explore interleavings that have small edit
distances from the original execution, we hypothesize that its probability of
finding the same invariant violation within the first few explored scheduled
will increase dramatically versus randomly choosing schedules to explore.
If the model checker can find violation-triggering schedules in $O(N)$ time rather than $O(2^N)$ time
(where $N$ denotes the number of events to schedule),
the asymptotic complexity of the overall minimization process can be reduced.
We will evaluate these heuristics empirically, by measuring the number of
interleavings that need to be explored for event traces where we know
MCS {\em a priori}. Often, we will develop these heuristics by examining the
behavior of specific distributed systems, as we describe next.

\noindent{\bf Applying to Other Distributed Systems.} In~\cite{sts2014} we
only applied our techniques to SDN control software. We do not however believe
that our techniques are specific to SDN. We have already begun applying our
techniques to other kinds of distributed systems, including distributed
databases and consensus protocols. One contribution of our work will be to
tailor our minimization strategies so that they
leverage the properties of those systems (e.g. atomic transactions in the
domain of distributed databases) or their empirical behavior (e.g. a tendency
to trigger violations along certain schedules) in
order to make the minimization process more practical.

% Left-behind ideas / lines of work:

% 1. exploring the "interposition tradeoff"? Iterate through the levels
% programmatically -- probably the case that different bug types necessitate
% different kinds of interposition. Also, look into modularizing model checking,
% a la JunFeng's work on demeter.

% 2. using Synoptic to generate model of the software.

% 3. Are Panda's email chains on modelling computational structure relevant?

% 4. Black-box delta debuggin on Jepsen.

% 5. Showing how to apply the techniques to production systems rather than QA
% tests.

% 6. Taint tracking / provenance

% 7. Dataflow analysis for filtering inputs. Possibly: dataflow analysis on
% network , but not controllers.

% 8. Optimizing delta debugging

% 9. Distinguishing between persistent and transient violations.

% 10. Using delta debugging to isolate differences, not just minimize whole test


\section{Related work}
\label{sec:related_work}
%-- Program Slicing --

The delta debugging algorithm~\cite{Zeller:2002:SIF:506201.506206} seeks to solve
a problem that is exactly analogous to ours on a single machine: given input that causes a test case
to fail, what is the minimum subset of the input that still produces the failure?
We apply the same reasoning to a distributed system.

%-- Deterministic Replay (OFRewind) --

Deterministic replay techniques such as OFRewind~\cite{ofrewind}
are designed to allow developers to interactively prune
the inputs that lead up to errant behavior. We present an algorithm that
automates this process.

%-- Model checking (NICE): --

NICE~\cite{nice} combines model checking with concolic execution
to enumerate all possible code paths taken by control software (NOX)
and identify concrete inputs (\eg{} control message orderings) that cause
the network to enter invalid configurations. Unlike NICE, by analyzing
bugs {\em post-hoc} from live runs of the system our approach applies
to large software systems without suffering from state explosion.

%-- Invariant Checking? --

\eat{
Invariant checking tools such as Anteater~\cite{anteater} and HSA~\cite{hsa}
detect problems in the dataplane. We leverage invariant checking tools
to distinguish inputs that are necessary for reproducing a given invariant violation.
}

%-- Root cause analysis? --

Root cause analysis techniques~\cite{577079} seek to identify the minimum set of failed
components (\eg{} link failures) needed to explain a collection of alarms. Rather than
focusing on individual component failures, we seek to minimize inputs that affect the behavior
of the overall distributed system.

%-- Distributed Systems debuggers --

Pip~\cite{pip} is a framework for instrumenting general-purpose distributed systems
with code to record, display, and check invariants on causal paths throughout
live executions. \Simulator{} observes the causal behavior of the
distributed system in a simulated environment, enabling us to iteratively prune extraneous input events.

%-- Simulators? --
%
%Several other network simulators exist for testing SDN controllers. Mininet is a
%platform for emulating OpenFlow switches and hosts within a single
% VM~\cite{Lantz:2010:NLR:1868447.1868466}. The ns-series of network simulators
%provides a general framework for testing new protocols, topologies,
%and traffic mixes~\cite{ns3}. We found that these existing simulators did
%not provide sufficient support for the corner-cases situations which are the
%focus of our work, such as failures and VM migration.

%-- Distributed Systems --
%
%Many of our ideas originate from the literature on troubleshooting general
%distributed systems. WiDS checker introduced the notion of recording
%production executions to be later replayed and verified in a controlled simulation.
% Finally, end-to-end tracing
%frameworks such as X-Trace~\cite{Fonseca:2007:XPN:1973430.1973450} and
%Pinpoint~\cite{Chen02pinpoint:problem} provide a framework for tracing requests throughout
%a distributed system in order to infer correctness errors between layers and
%across components. Our work solves a more constrained problem; we leverage
%the structure of the SDN stack to enable a simple notion of platform
%correctness. In addition, these systems assume that invariants should hold at
%all times; we observe that in an eventually-consistent system such as SDN,
%transient policy-violations are inevitable. We built \simulator{} to help troubleshooters
%differentiate ephemeral from persistent errors.

% If we manage to run multiple applications by Monday, we should cite papers
% on consistency and cross-layer debugging:
%X-Trace~\cite{xtrace}
% Vector Clocks
% Onix
% Virtualization definitely won't happen by Monday. But, papers include
% Martin's presto '10 paper 'Virtualizaing the Network Forwarding Plane'



\section{Conclusion}

We debugged all the things!

\bibliographystyle{abbrv}
\bibliography{bib}

%\input{appendix}

\end{document}
