    We plan to continue developing \projectname{} with the eventual aim of publication; we plan to 
submit to SIGCOMM 2012. 
    Our plans for future work include the following:

    {\bf Detailed Bug Report Analysis.} Pending CTO permission, we've recently
     been granted access to raw bug reports from Nicira users running Onix in production. 
     We'd like to use this data to answer questions about commonly encountered SDN bugs: 
     Of the bug classifications developed in S\ref{sec:bugs}, which are the most common?    
     When administrators trace down the root cause of the bugs, where in the SDN stack are
     bugs most prevalent?

    {\bf Build and Evaluate Correspondence Checker.} Our existing implementation does not
    include the cross-layer correspondence checker described in \S\ref{sec:architecture}.

    {\bf Design, Build, and Evaluate Consistency Debugger.} We built a distributed controller for
    POX, but have yet to design and build a consistency debugger to detect and resolve bugs
    resulting from replication.~\footnote{We're interested in combining this with techniques to make
    programming consistency better by treating the whole network as a distributed system, \eg{} adding
    vector clocks to switches and deploying updates as transactions.}

    {\bf Improved input generator.} Our current implementation of the input generator is
    a random ``fuzzer.'' The fuzzer randomly explores the state space of possible flows, but
    we believe that more sophisticated techniques (high-fideity tracing, symbolic execution) can help
    us target input flows to more efficiently explore the space of possible bug-generating inputs. 

    {\bf Run on a real, rather than simulated network.} Our evaluation ran on a real POX implementation,
    but the physical network controlled by POX was a simulated, rather than real network.

    {\bf Distinguish between transient and persistent errors.}  Transient errors are very different than persistent errors. Both are
    important problems. But you attack those two problems very differently. Can we
    automatically distinguish the two? That is, can we infer whether a bug is
    triggered by a certain order of events in the system, versus a systematic bug
    with the NOS state machine?

    {\bf Reactive trobleshooting techniques.} \projectname{} passively monitors state of the 
    components of the SDN, but it could benefit from reactive techniques like issuing traceroutes
    when a bug is first detected; this supplemental information can aid in the debugging process.

    {\bf Better tools for administrators.} We plan to develop two techniques to help administrators use
    the debugger. First, we plan to allow them to specify their own, custom invariants (deployment-specific policies)
     included with the standard invariants (\eg{} loops, dead ends).
    Second, enabling and interactive debugging process, allowing administrators to step by step monitor
    packets traversing the network and inspect FIB entries and network state at each hop.

