
This project is only in its beginnings. There are a number of research
avenues we would to explore in the near future:

\begin{itemize}
\item Analysis of bug reports would be really nice!
\item Right now we just simluate the network. Would be cool to emulate it.
Perhaps even run the debugger in a live network.
\item Virtualization bugs are really hard to deal with. We haven't yet evaluated
our correspondence checks on a real virtualized application.
\item Consistency errors in multi-server deployments are extremeley
pernicious. Let's focus our efforts on these. Initial idea: treat network
devices as a first class entity in the distrbuted system. Make sure they
support transactions, vector clocks, \etc.
\item Explore reactive troubleshooting techniques, \eg launching traceroutes
when a problem occurs in a live network. Also, X-Trace might be complementary
to static invariant checks. 
\item Make it easier for application developers to specifiy invariants.
\colin{Actually, JRex and Brighten are much better at that sort of thing than
we are.}
\item I think it would be really cool to build a fully interactive debugger
for a live system (think gdb). Rough idea: empty the flow entries of all switches so that
all packets go the controller. for each PacketIn event, allow the developer to
step through the code, examine variables, change variables, change code on the
fly, and then push the packet back down the next hop.
\item Right now the input generator is just a fuzzer. Let's explore more
sophisticated techniques, \eg high-fidelity tracing, symbolic execution, \etc.
\item Transient errors are very different than persistent errors. Both are
important problems. But you attack those two problems very differently. Can we
automatically distinguish the two? That is, can we infer whether a bug is
triggered by a certain order of events in the system, versus a systematic bug
with the NOS state machine?
\end{itemize}
