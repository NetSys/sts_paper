
In the previous section we described common error conditions observed in
software-defined networks. In this section we present two techniques for
detecting their occurence (\ref{sec:detection}) and one insight useful for isolating their cause
(\ref{sec:isolation}).

\subsection{Detection}
\label{sec:detection}

Many of the error conditions described in the previous section can be detected
by analyzing a static snapshot of the network state. For example, Anteater \cite{anteater}
showed that all of the generic errors shown in Figure
\ref{fig:generic_errors} can be detected in traditional networks by invoking a
SAT solver on constraint formulas derived from routers' forwarding tables.

There are two differences between traditional networks and the SDN environment
which imply that static analysis alone is insufficient. The most important
difference is that OpenFlow switches typcially do not have default forwarding
rules; flow entries are often installed on-demand by the controller
whenever unmatched packets arrive at the switch. Although loops (i) and
routing inconsitencies (iv) can still be detected in sparse forwarding tables,
network partitions \colin{Routing partitions, not physical partitions} (ii) and dead ends (iii) cannot. A related difference is that
SDN controllers tend to make frequent changes to network devices, implying that the
configuration space is far too large to be verified with a single analysis of
the network state.

To enable invariant checks for the SDN environment, we have developed a
technique we call {\it dynamic invariant checking}. The basic idea is to
explore the entire configuration space of the control application, invoking
invariant checks at well-chosen intervals. In this way, dead ends (iii) can be
detected by waiting for the controller to install flow entries for all
switches along the path for a particular flow. In a similar way, network
partitions (ii) can be detected by generating flows for all pairs of nodes in
the network. \colin{Maybe mention that the configuration space should be tractable,
since the virtualized NOM is intentionally simple?}

\colin{Justine had a very interesting question: are there failure modes in
{\it installed} OpenFlow entries that would not be detectable by
Anteater? Her particular example turned out to fall under routing inconsistency,
but I'm interested in pursuing this line of reasoning later on}

Checking invariants in the dataplane does not suffice to detect all failure modes. In
particular, even if the physical network passes invariant checks, it may not behave as
the control application intends it to. Consider for example, a bug in the
virtualization layer of the NOS which causes ACL rules to be placed
incorrectly along the virtual path between hosts in a datacenter.

To capture this failure mode, we have developed a technique we call `cross-layer
correlation`. Essentially, this involves translating the
virtualized representation of the network state to physical flow entries
independently from the system's translation logic. If
there is not a one-to-one correspondance between the derived network
state and the true network state, we warn the user of a possible
correspondence error.

\colin{We could talk a bit about system-wide analysis just so that we have
three techniques. If we do, the notion of order-independence is particularly
important.}

\subsection{Isolation}
\label{sec:isolation}

To aide developers in finding the root cause of a detected error, we note that
the modularity of the SDN stack works to our advantage. Because the interfaces
between the layers are well-defined, and the representation of the 
of the network state is known a priori, each layer can effectively be tested in
isolation. Conceptually, the debugger generates a set of hypotheses about
where the cause of the error may reside. It then successively prunes these
hypotheses by running invariant checks on each layer's representation of the
network.
