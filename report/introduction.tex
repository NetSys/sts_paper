Software-defined networking (SDN) simplifies network management by presenting a
global, abstract model of network state to management applications, hiding
the low-level details of individual switch configurations. However, in some ways this abstraction does
its job too well: by hiding the details of actual network behavior, debugging
system errors becomes considerably more difficult.
For our class project, we plan to build a debugger for software-defined networks to
simplify the detection, isolation, and repair of errors at all layers of
the network management stack.

%We begin by providing an overview of the software-defined network (SDN)
%architecture. 
Unlike in a traditional network, where each switch and router is independently responsible for
its own forwarding decisions, failure recovery, and access control, all management
decisions in a SDN are made by a logically-centralized controller. When a
switch observes a packet for which it does not have a forwarding rule,
it forwards the packet to a controller running a ``network operating system''
(NOS). The NOS, which maintains the global model of the network, updates its representation
of the switch's state, and then notifies a network management application of the event.
Finally, the management application decides how to handle the packet, and informs the NOS
and the switch of the new policy.

This modularity gives rise to three classes of system errors: \\
\noindent {\bf Semantic mismatches.} 
Each layer of of the architecture maintains a different model of the network state: the
switches maintain rules in TCAMs, the NOS represents the switches
in a graph-like data structure, and the network application operates on 
a virtualized version of this graph. It should be
the case that there is always a one-to-one mapping between the state at all
layers, but this does not always hold in practice.

\noindent {\bf Inconsistency between switches and controllers}.
Software-defined networks are extremely dynamic distributed systems.
For example, switches may suddenly disconnect and reconnect to a controller, causing inconsistent state
within the NOS.

\noindent {\bf Inconsistency between controllers}. In production-quality SDNs,
the NOS is distributed amongst many servers to achieve fault-tolerance
and scalability. Like in any system with distributed control, controllers may
act on inconsistent state in case of race-conditions or failures.

An SDN debugger must be general enough to address all of the above problems, and provide programmers
with the ability to (i) detect that a problem exists, (ii) identify the layer / component / line of code
responsible for the problem, and (iii) reproduce the execution that triggered the error.

In general, distributed systems debugging techniques lie somewhere on a spectrum between
{\it static checking} and {\it interactive replay}. Our first goal is to find
the point on this spectrum most appropriate for SDN. 

\noindent {\bf Static-checkers} \cite{anteater} excel at {\it detecting} that problems
     exist. However, SDNs are far too dynamic for pure static checking; one would
     essentially need to take snapshots at every state change. Moreover,
     knowing {\it that} a problem exists does not necessarily help isolate its
     cause.

\noindent {\bf Tracing frameworks} \cite{x-trace} excel at pinpointing the cause of an
error. However, the space of possible traces is intractable, so it can be very
difficult to produce the input that causes the error to arise in the first
place.

\noindent {\bf Interactive debuggers} facilitate intuition and shorten the
      debugging process. However, true interactivity is extremely difficult to
      obtain in a distributed system. 
	%That said, there may be hope with SDN:
      %the controller can simply refrain from installing flow rules in the switches
      %such that {\it all } packets are sent to the
      %centralized controller. Nevertheless, an interactive debugger would still
      %have difficulty detecting problems and reproducing problematic
      %executions. Moreover, 
      Moreover, while interactive debuggers are
      well-suited for sequential
      computations, networks are inherently event-based.

Although distributed systems debugging is well explored,
 software-defined networks differ in a number of ways.

        \justine{How does this come across in CLINT?} \colin{Right now, it
        doesn't. However, when we start dealing with consistency bugs, we will
        need to treat the switches as a first class entity in the distributed
        system. In particular, we will need to make sure that they support
        vector clocks to be able to infer a global ordering of events.}
	First, distributed system designers model inter-node communication as a black box, whose failure modes are an independent class of failures from the system itself.
	Thus, distributed systems makes guarantees that hold only ``when the network recovers from partition.''
	In SDN, however, the failed system {\it is} the network: loss of
    communication and system failure are one and the same. This implies an
    entirely different class of failure-modes.

	Second, SDN provides significant domain-specific context to inform the debugging process.
	The  SDN architecture provides well-defined interfaces whose correct interactions can be verified; \eg{} that the state of a switch and the state of its representation in the NOS should be consistent.
	Further, networking itself requires specific properties of the the network graph, the state of the switches, \etc{}; for example, the network graph should always be acyclic.
	These domain-specific invariants do not need to be specified by the programmer, but instead should be ``baked-in'' to the SDN debugging process.

        CLINT does\ldots

