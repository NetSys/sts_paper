Unfortunately, treating the software as a blackbox disallows us from
providing any guarantees on the minimality of the MCSes we produce.
In this section we flesh out plans for research on
(i) providing provable guarantees on minimality, (ii) developing optimizations
to make the provable scheduling techniques practical, and (iii) showing
that our techniques can be applied to general distributed systems.

\noindent{\bf Providing Provable Guarantees.} If we were to repeatedly
$replay$ a
fixed subsequence of external events $E_S$ to a blackbox distributed system,
it is possible that it would not to produce the same output on each
execution, even if we were to apply the heuristics outlined in
\S\ref{sec:past_work}. The issue is that we do not have full visibility into internal
events nor control over the internal scheduling decisions of the distributed
system. Our approach in~\cite{sts2014} was to
replay each subsequence multiple times, but this does not provide any
guarantees on the minimality of the MCSes we produce, since it's possible that
a subsequence $E_S$ that we thought did not reproduced the invariant violation
 could have done so if the
distributed system had interleaved its internal events in some other way.

In order to provide provable guarantees, we plan to use a model checker to
{\em certify} whether each subsequence $E_S$ chosen by delta debugging does or
does not reproduce the original violation. The model checker can accomplish this
by systematically exploring all possible interleavings of internal events
given the fixed sequence of external events $E_S$.

In some cases however, it is not possible for the model checker to certify a
given subsequence $E_S$. For example, suppose that the distributed system does
not terminate (or more specifically, suppose its state space is infinite
and non-recurring). In that case, model checking is not guaranteed to terminate.
Crucially, if the
model checker does not find a way to trigger the original violation in finite time,
that does not imply that the violation cannot be triggered, and we therefore lose our guarantees on minimality.

In distributed systems, non-termination means that an algorithm is not guaranteed to stop
sending messages. The formal term for this property is `non-quiescence`. In~\cite{aguilera1997heartbeat} Aguilera et al. prove that
all failure detectors---an important algorithmic component of distributed systems---are
non-quiescent.

Luckily, given a failure detector component, it is possible to implement many other
distributed algorithms in a quiescent manner. Building on this observation,
one of our proposals for coping with non-quiescence is to mark failure detector algorithms as `trusted-components`,
and prove modified soundness and completeness properties of model checkers that control rather than check the failure detector component.
Another option for coping with non-quiescence is to weaken our definition of soundness,
i.e. explore enough states for us to be satisfied, even if it is not all states.

\noindent{\bf Optimizing Model Checking.} Although we only use the
model checker to find specific invariant violations rather than asking it to
find all invariant violations, it may nonetheless need to explore an
intractable number of event interleavings. This would make it impractical for
use on real distributed software.

We believe that we can ameliorate computational intractability by
leveraging the prior knowledge contained in the original failing test case.
That is, by making the $replay$ function stateful, we can develop heuristics that lead the
model checker to quickly find interleavings that trigger the original violation, so that
it is only forced to enumerate all interleavings for subsequence $E_S$ that do
not trigger the original violation.

As a concrete example, if we lead
the model checker to first explore interleavings that have small edit
distances from the original execution, we hypothesize that it will find
violations quickly, thereby reducing the asymptotic complexity of model
checking. We will evaluate these heuristics empirically, by measuring the number of
interleavings that need to be explored for event traces where we know
MCS {\em a priori}.

\noindent{\bf Applying to Other Distributed Systems.} In~\cite{sts2014} we
only applied our techniques to SDN control software. We do not however believe
that our techniques are specific to SDN. We have already begun applying our
techniques to other kinds of distributed systems, including distributed
databases and consensus protocols. One contribution of our work will be to
enumerate the properties of those systems (e.g. atomic transactions in the
domain of distributed databases) that make them more or less amenable to
applying our minimization techniques.

% Left-behind ideas:

% 1. exploring the "interposition tradeoff"? Iterate through the levels
% programmatically -- probably the case that different bug types necessitate
% different kinds of interposition. Also, look into modularizing model checking,
% a la JunFeng's work on demeter.

% 2. using Synoptic to generate model of the software.

% 3. Are Panda's email chains on modelling computational structure relevant?

% 4. Black-box delta debuggin on Jepsen.

% 5. Showing how to apply the techniques to production systems rather than QA
% tests.
