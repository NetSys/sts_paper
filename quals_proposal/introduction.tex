Developers of distributed systems face notoriously difficult challenges, such as concurrency, asynchrony, and
partial failure. A crucial consequence of these challenges is that software
bugs are pervasive in distributed systems.

In an effort to catch these bugs
before they are released into production, many commercial vendors employ
teams of QA engineers. The QA engineers exercise automated test scenarios that involve
sequences of external events---such as failures or message delays---on large testbeds.
If they detect an invariant violation, they hand the resulting trace to a developer for analysis.

When faced with symptoms of a failing QA test (\eg~inconsistent system state)
software developers first need to identify which of the potentially tens of
thousands of events\footnote{Consider that an hour long QA test
emulating event rates observed in production could contain 8.5 network error
events per
minute~\cite{Greenberg:2009:VSF:1592568.1592576} and 500 VM migrations per
hour~\cite{Soundararajan:2010:CBS:1899928.1899941},
for a total of $8.5 \cdot 60 + 500 \approx 1000$ externally triggered events, not counting
internal events.}
in the QA test are relevant to triggering
this apparent bug before they can begin to isolate and fix it.
This act of ``troubleshooting'' (which precedes the act of debugging the
code) is highly time-consuming, as developers spend hours poring
over multigigabyte execution traces. This should come as no surprise since software developers in general spend roughly half (49\% according to one
study~\cite{msoft_concurrency}) of their time troubleshooting and debugging, and spend
considerable time troubleshooting bugs that are difficult to trigger (the same study found
that 70\% of the reported concurrency bugs take days to months to fix).
Consequently, when asked to describe their
ideal tool, most network operators said ``automated
troubleshooting''~\cite{Zeng:Survey}.

Our aim is to reduce effort spent on troubleshooting distributed systems.
Our thesis is the following:
\begin{quote}
{\it
It is possible to
automatically identify minimal sequences of inputs
responsible for triggering
distributed systems test case failures.
}
\end{quote}
By eliminating events from event traces that are not related to the
invariant violation at the end of the trace, we produce a ``minimal
causal sequence'' (MCS) of triggering events. After the MCS has been produced, the developer embarks
on the debugging process. We claim that the greatly reduced size of the
trace makes it easier for the developer to figure out which code path contains
the underlying bug, allowing them to focus their effort on
the task of fixing the problematic code itself. After the bug has been fixed, the MCS
can serve as a test case to prevent regression,
and can help identify redundant bug reports where the MCSes are the same.

Our goal of minimizing traces is in the spirit of
delta debugging~\cite{Zeller:1999:YMP:318773.318946}, but our problem is
complicated by the distributed nature of control software:
our input is not a single file fed to a single point of execution, but an ongoing
sequence of events involving
multiple actors. We therefore need to carefully
control the interleaving of events in the face of asynchrony, concurrency and non-determinism in
order to reproduce bugs throughout the minimization process.

In the first part of this dissertation proposal (\S\ref{sec:past_work}), we present research that has
already been completed on minimizing traces without making assumptions about the language
or instrumentation of the software~\cite{sts2014}. We evaluated this on one
particular type of distributed system: software-defined networking
(SDN) control software.

%We have built a troubleshooting system that,
%as far as we know, is the first to meet these challenges
%(as we discuss further in \S\ref{sec:related_work}). %(\cf~\S\ref{sec:related_work}).
%Our troubleshooting system, which we call {\projectname} ({\projectmeaning}),
%consists of 23,000 lines of Python, and is designed so that organizations can
%implement the technology within their existing QA infrastructure (discussed in
%\S\ref{sec:architecture}); over the last year we have worked with a
%commercial SDN company to integrate \projectname. We evaluate {\projectname} in
%two ways. First and most significantly, we use {\projectname} to troubleshoot
%seven previously unknown bugs---involving concurrent events,
%faulty failover logic, broken state machines,
%and deadlock in a distributed database---that we found
%by fuzz testing five controllers (Floodlight~\cite{floodlight}, NOX~\cite{nox},
%POX~\cite{pox}, Pyretic~\cite{frenetic}, ONOS~\cite{ONOS})
%written in three different languages (Java, C++, Python).
%Second, we demonstrate the
%boundaries of where \projectname~works well by
%finding MCSes for previously known and synthetic bugs that span a range of bug
%types. In our evaluation, we quantitatively show that \projectname~is able to
%minimize (non-synthetic) bug traces by up to 98\%, and we anecdotally found that reducing
%traces to MCSes made it easy to understand their root causes.

%Unfortunately, treating the software as a blackbox disallows us from
%providing any guarantees on the minimality of the MCSes we produce.
The remainder of this dissertation proposal (\S\ref{sec:future_work}) fleshes out plans for
broadening the applicability of our techniques to other kinds of distributed systems besides SDN control
software, and providing a stronger theoretical foundation for mechanics of the techniques.
In particular, we plan to (i) start with an infeasible but provably correct
approach, and (ii) find practical approximations to this approach, many of which will
involve leveraging empirical properties of (iii) different classes of distributed software
systems (beyond just SDN control software).

We start by providing a formal problem definition (\S\ref{sec:formalism}) and
discussing related work (\S\ref{sec:related_work}). We end by
presenting a timeline for the completion of the proposed research (\S\ref{sec:timeline}).
