\noindent{\bf Techniques for Selecting Subsequences.} Checking
random subsequences of $E_L$ would be one viable but inefficient
approach to achieving our first task. A better approach is
%divide-and-conquer search technique from the software engineering community:
the delta debugging algorithm~\cite{Zeller:1999:YMP:318773.318946,Zeller:2002:SIF:506201.506206}, a
divide-and-conquer algorithm for
isolating fault-inducing inputs. We use delta
debugging\footnote{Domain-specific
minimization algorithms also
exist~\cite{regehr2012test,claessen2000quickcheck,whitaker2004configuration},
but we focus on the general case.} to iteratively select subsequences of $E_L$ and $replay$ each
subsequence with some timing $T$.\\[0.5ex]
%If the bug persists for a given subsequence, delta debugging ignores the
%other inputs, and proceeds with the search for an MCS within this subsequence.
%In what follows, we use the term {\em delta debugging} to refer to our algorithm for finding relevant subsequences.
%The delta debugging algorithm we implement is shown in Figure~\ref{fig:ddmin}.
%
\noindent{\bf Techniques for Selecting Timings.} Simply
exploring subsequences $E_S$ of $E_L$ is insufficient for finding
MCSes: the timing of when we inject the external events during $replay$ is crucial for
reproducing violations.

The most natural approach to scheduling
external events is to maintain the original wall-clock timing intervals
between them.
If this is able to find all minimization opportunities,
\ie~reproduce the violation for all
subsequences that are a supersequence of some MCS, we say that the inputs are
isolated. The original applications of delta
debugging~\cite{Zeller:1999:YMP:318773.318946,Zeller:2002:SIF:506201.506206,regehr2012test,burger2011minimizing} make this assumption (where a
single input is fed to a single program), as well as QuickCheck's input ``shrinking''~\cite{claessen2000quickcheck}
when applied to blackbox systems like synchronous
telecommunications protocols~\cite{arts2006testing}.
\eat{\ie~if we
define $t_i$ as the timestamp of the $i^{th}$ input from the original run and $t'_i$ as the
replay clock value when it injects that same input
(which may or may not be the $i$'th input in the subsequence), then we might
just set $t'_i = t_i$.
\eat{
\begin{align*}
t'_0 = t_0 \\
t'_i = t'_{i-1} + |t_{i} - t_{i-1}|
\end{align*}
}
}

We tried this approach, but were rarely
able to reproduce invariant violations. As our case studies
demonstrate~\cite{sts2014}, this is largely due
to the concurrent, asynchronous nature of distributed systems; consider that the network
can reorder or delay messages, or that processes may
respond to multiple inputs simultaneously.
Inputs injected according to wall-clock time are not guaranteed to
coincide correctly with the current state of the processes.
%\eat{: when we replay only a
%subsequence of the original inputs, the reaction of the control software
%can change, such that it behaves differently or takes a different amount of
%time to respond to the remaining inputs events.
%In practice we have observed that simply maintaining relative timings can
%result in injecting the remaining inputs too early or late.}

We must therefore consider the distributed system's internal events. To deterministically
reproduce bugs, we would need visibility into every I/O request and response (\eg~clock
values or socket reads), as well as all thread scheduling
decisions for each process. This information is the starting point for
techniques that seek  to minimize thread interleavings leading up to race conditions.
These approaches involve iteratively feeding a single input (the thread
schedule) to a single entity (a deterministic scheduler)~\cite{choi2002isolating,claessen2009finding,jalbert2010trace}, or
statically analyzing feasible thread schedules~\cite{huang2011efficient}.

A crucial constraint of these approaches is that they must keep the inputs
fixed; that is, behavior must depend uniquely on the thread
schedule. Otherwise, the nodes may take a divergent code path. If this
occurs some processes might issue a previously unobserved I/O request, and the replayer will not
have a recorded response; worse yet, a divergent process might deschedule
itself at a different point than it did originally, so that the remainder of
the recorded thread schedule is unusable to the replayer.

Because they keep the inputs fixed, these approaches strive for a
subtly different goal than
ours: minimizing thread context switches rather than input events.
At best, these approaches can indirectly minimize input events by truncating
individual thread executions. That is, they can cause threads to exit early
(thereby shortening the execution trace), but they cannot remove extraneous events from
the middle of the trace.

% (\ie~causing them to exit early).

%\noindent{\bf Program Flow Analysis.} Other than delta debugging,
%the closest work to ours involves reasoning about
%control- and dataflow dependencies
%(which may be recorded at runtime~\cite{Lee:2011:TGR:1993498.1993528},
%or dynamically inferred~\cite{, tallam2007enabling})
%in order to reduce the length of deterministic replay executions.
%%decision processes of control software. % Our Peek() algorithm is a
%% contribution in how to infer dependencies without access to software internals
%Moreover, our application of functional equivalence to the space of
%possible inputs allows us to minimize
%inputs more aggressively, whereas they are forced to consider all controlflow
%and dataflow dependencies in the control software.

With additional information obtained by program flow
analysis~\cite{Lee:2011:TGR:1993498.1993528,tallam2007enabling,huang2012lean}
however, the inputs no longer need to be fixed. The internal events considered by these program flow reduction
techniques are individual instructions executed by the
programs (obtained by instrumenting the language runtime), in addition to I/O responses and the thread schedule.
With this information they can compute
program flow dependencies, and thereby remove input events from anywhere in the trace as long as they
can prove that doing so cannot possibly cause the faulty execution path to diverge.

While program flow reduction is able to minimize inputs,
these techniques are not able to explore alternate code paths that still
trigger the invariant violation. They are also overly conservative in
removing inputs (\eg~EFF takes the transitive closure of all possible
dependencies~\cite{Lee:2011:TGR:1993498.1993528}) causing them to miss opportunities to
remove dependencies that actually semantically commute.\\[0.5ex]
%
\noindent{\bf Our Contribution.} For the first portion of this dissertation proposal (\S\ref{sec:past_work}) we
discuss heuristics we have developed for interleaving internal and external
events that we have empirically shown to work
well for minimizing event traces. Our approach there is
to allow processes to proceed along divergent paths
rather than recording all low-level I/O and thread scheduling decisions.
This has several advantages. Unlike
the other approaches, we can find shorter alternate code paths that still
trigger the invariant violation. Previous {\em best-effort} execution minimization
techniques~\cite{clause2007technique,tucek2007triage} also allow alternate
code paths, but do not systematically
consider concurrency and asynchrony.\footnote{PRES
explores alternate code paths in best-effort replay of multithreaded
executions, but does not minimize executions~\cite{park2009pres}.}
We also avoid the performance overhead of recording all I/O
requests and later replaying them (\eg~EFF incurs \textasciitilde10x slowdown during
replay~\cite{Lee:2011:TGR:1993498.1993528}). Lastly,
we avoid the extensive effort required to instrument the control software's language runtime,
needed by the other approaches to implement a deterministic thread scheduler, interpose on syscalls,
or perform program flow analysis. By avoiding assumptions about the language of the control software,
we were able to easily apply our system to five different control platforms
written in three different languages.

Unfortunately, these heuristics do not lend themselves to solid theoretical
explanations for why they work well. In the latter portion of the
proposal (\S\ref{sec:future_work}) we outline a plan for
developing scheduling strategies that are based on sound principles (starting
with visibility into the structure of the software under test).
As far as we know, the problem statement and approach we pose there---applying model checkers
to {\em certify} whether each subsequence chosen by delta debugging reproduces
the original bug---has not appeared in the literature before.\\[0.5ex]
%
\noindent{\bf Broadly Related Work.} We end this section by discussing
techniques in the general area of troubleshooting. We characterize the other troubleshooting approaches
as (i) instrumentation (tracing),
(ii) bug detection (invariant checking),
(iii) replay,
(iv) root cause analysis (of device failures), (v) log comprehension
(visualization), (vi) program slicing, and (vii) automated debugging (fault
localization).\\[0.5ex]
%
\noindent{\bf Instrumentation.} Unstructured
log files collected at each node are the most common form of diagnostic information. The goal of
tracing frameworks~\cite{pip,fonseca2007x,Chen02pinpoint:problem,ndb14,barham2004using}
is to produce structured logs that can be easily analyzed, such as DAGs tracking
requests passing through the distributed system. These tools help developers to understand
how, when, and where the system broke. In contrast, we focus on making it
easier for developers to understand what lead the software
to violate an invariant in the first place.\\[0.5ex]
%
\noindent{\bf Bug Detection.} With instrumentation available, it becomes possible
to check expectations about the
system's state (either offline~\cite{Liu07widschecker} or online~\cite{d3s}), or about the paths requests take through
the system~\cite{pip}. Within the networking community, this research is
primarily focused on verifying routing tables~\cite{hsa,hsa_realtime,anteater,khurshid2012veriflow}
or forwarding behavior~\cite{Zeng:2012:ATP:2413176.2413205,libra}.
We use bug detection techniques (invariant checking) to guide delta debugging's minimization
process.\\[0.5ex]
%
It is also possible to infer
performance anomalies by building probabilistic models from
collections of traces~\cite{barham2004using,Chen02pinpoint:problem}.
Our goal is to produce exact minimal causal sequences, and we are primarily focused on
correctness instead of performance.\\[0.5ex]
%
Model checkers~\cite{musuvathi2008finding,nice} seek to
proactively find safety and liveness violations by analyzing all possible code paths.
After identifying a bug with model checking, finding a minimal code path leading to it is
straightforward. However, the testing systems we aim to improve do not employ
formal methods such as model checking, in part because model checking usually suffers from exponential
state explosion when run on large systems.\footnote{For example, NICE~\cite{nice} took 30 hours to
model check a network with two switches, two hosts, the NOX MAC-learning
control program (98 LoC), and five concurrent
messages between the hosts.} Nonetheless, in \S\ref{sec:future_work} we propose the
use of model checkers to provide provably
minimal MCSes. Crucially, we assume that we are already given a known
execution leading to an invariant violation, rather than attempting to find all
possible invariant violations without prior knowledge of faulty executions. We
believe that the prior knowledge contained in failing test cases can mitigate exponential
state explosion, and as far as we know we are the first to use model checkers
to minimize existing executions.\\[0.5ex]
%In future work we plan to explore the possibilities of model checking techniques with our %Rather than exploring all
%possibilities, we discover bugs through testing of selected scenarios, which
%makes our problem tractable.
%which we believe fits more naturally into software engineers' chosen
%way of developing and testing distributed systems.
%\colin{TODO: discuss how finding the critical transition is related to finding the MCS}
%
\noindent{\bf Replay.} Crucial diagnostic information is often missing from traces.
Record and replay
techniques~\cite{Geels:2006:RDD:1267359.1267386,lin2013defined,Zamfir:2010:EST:1755913.1755946,Yuan:2010:SED:1736020.1736038}
instead allow users to step through (deterministic) executions and interactively examine the
state of the system in exchange for performance overhead.
%Static analysis and symbolic execution can also be applied post-hoc %, without incurring runtime performance overhead,
%to find code paths that lead up to software
%crashes~\cite{Yuan:2010:SED:1736020.1736038} or thread schedules that reproduce
%race conditions~\cite{Zamfir:2010:EST:1755913.1755946}.
%Within SDN, OFRewind~\cite{ofrewind} provides
%record and replay of OpenFlow channels between controllers and switches.
Manually examining long system executions can be tedious, and our goal is to
minimize such executions so that developers find it easier to identify the
problematic code through replay or other means.\\[0.5ex]
%Rx~\cite{qin2005rx} is a technique for improving availability: upon
%encountering a crash, it starts from a previous checkpoint, fuzzes
%the environment (\eg~random number generator seeds) to avoid triggering the same bug,
%and restarts the program. Our
%approach perturbs the inputs rather than the environment
%prior to a failure.
%
\noindent{\bf Root Cause Analysis.} Without perfect instrumentation,
it is often not possible to know exactly what events are occurring (\eg~which
components have failed) in a
distributed system. Root cause analysis~\cite{yemini1996,Kandula:2009:DDE:1592568.1592597}
seeks to reconstruct those unknown events from limited monitoring data.
Here we know exactly which events occurred, but
seek to identify a minimal sequence of events.\\[0.5ex]
%
\noindent{\bf Log Comprehension.} Model inference techniques attempt to summarize log files
in order to make them more easily understandable by
humans~\cite{synoptic,csight,biermann1972synthesis,lorenzoli2008automatic,lou2010mining}.
Model inference does not modify the logs, although
it can be applied to our minimized logs to further facilitate the
troubleshooting process.\\[0.5ex]
%
\noindent{\bf Program Slicing.} It is worth mentioning another goal outside the purview of distributed systems, but
closely in line with ours: program slicing~\cite{weiser1981program} is a
technique for finding the
minimal subset of a program that could possibly affect the result of a particular line of code.
%This can be combined with delta debugging to automatically generate minimal unit tests~\cite{burger2011minimizing}.
Our goal is to slice the temporal dimension of an execution rather than the
code dimension.\\[0.5ex]
%
\noindent{\bf Automated Debugging.} Similar techniques go so far as to try to
automatically locate the exact line(s) of code or state transitions that are responsible for a
bug, using statistical data~\cite{zhangzhang}, test coverage
data~\cite{coverage_localization,xuan14}, constraint solving~\cite{jose11},
fault injection~\cite{zhang13} and
experimentally driven executions of the failing program~\cite{zeller2005,comparative_causality}.
These approaches aid the debugging process rather than test case
simplification,
and are therefore complementary.
