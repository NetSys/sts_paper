SDN is widely heralded as the ``future of networking'', because it makes it
much easier for operators to manage their networks. SDN does this, however, by
pushing complexity into SDN control software itself. Just as sophisticated
compilers are hard to write, but make programming easy, SDN platforms make
network management easier for operators, but only by forcing the developers of
SDN platforms to confront the challenges of asynchrony, partial failure, and
other notoriously hard problems that are inherent to all distributed systems.
%Thus, people will be troubleshooting and debugging SDN control software for many
%years to come, until it becomes as stable as compilers are now.

Current techniques for troubleshooting SDN control software are quite primitive; they
essentially involve manual inspection of logs in the hope of identifying the
relevant inputs. In this paper we developed a technique for automatically
identifying a minimal sequence of inputs responsible for triggering a given
bug. We believe our technique will be especially valuable for troubleshooting
distributed controllers running complex applications, which are just now
becoming available to the public and the broader research community.
%We have applied this system to three open source SDN platforms.
%Of the five bugs we encountered in a five day investigation,
%our technique reduced the size of the trace to 2 inputs in the best
%case and 18 inputs in the worst case.

\eat{
SDN is widely heralded as the ``future of networking'', and its purpose is to make
it easy to manage networks. Achieving this end forces platform developers to directly
confront asynchrony, partial failure, and other problems that are inherent to all distributed
systems and notoriously difficult to get right.

In this paper we developed a technique for automatically
identifying a minimal sequence of inputs responsible for triggering a given bug.
We have applied this system to three open source SDN platforms, and
were able to find or reproduce bugs in all the platforms we investigated.
%Now we just need to provide a mechanism for the next question: would that date
%have panned out if I hadn't spilt the wine?
}

% Two ideas that have been put on the backburner:
% - distinguishing persistent violations from transient
% - using correspondence checking to localize the layer where the bug first
%   manifests

%We chose SDN as our domain becuase X,Y,Z. We envision a new paradigm where
%domain knowledge is applied to debugging in all systems.

% ------------------------------------------- %
%             OLD TEXT
\eat{
It does so by moving control plane functionality out of
network devices, and into a tightly-coupled cluster of servers that provide a simple
programmatic interface through which policies can be specified. As we have
learned in this work, the challenges of maintaining virtualized and distributed
views in a failure-prone environment are
notably different from the challenges encountered in traditional,
fully distributed control planes.

}
