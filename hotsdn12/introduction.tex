The SDN platform's $raison\text{ }d'\hat{e}tre$ is to 
hide complexity from control applications. Modern platforms perform
replication, resource arbitration, failure recovery, and network 
virtualization on the control application's behalf. 

Despite the abstractions provided by the SDN programming model,
software-defined networks are no less complex than traditional networks. The architectural goal of SDN is
simply to push complexity from the control application onto the underlying platform.

SDN platform software is prone to bugs as a result of this complexity. Bugs in the
platform present an architectural tension: troubleshooting requires
access to precisely the same details hidden by the platform's abstractions.
When an application developer 
encounters erratic behavior in the network, they must trace their
policy specification through multiple layers of abstraction
preceding changes in the physical network: virtualization logic,
distribution logic, and network devices. Moreover, the error's root cause
may manifest in any of these layers, not just the control application.

As it stands, the SDN platform provides meager support for troubleshooting.
The predominant troubleshooting method is log analysis: manually
specifying log statements at relevant points throughout the system,
collecting, gathering, and ordering distributed log files, and analyzing the
results {\it post-hoc} when a error is encountered in production. Besides its
apparent tediousness, this approach is lacking in several ways: logs events
are enormous in number, impossible to aggregate into a single serial
execution of the system, and often at the wrong level of granularity to be of
use. \colin{</ why it's hard>}

Recent work has contributed much-needed improvements to this state of affairs. 
NICE applies concolic execution and model checking to SDN control
applications, thereby automating the testing process and catching bugs before
they are deployed~\cite{nice}. Aneater~\cite{anteater} and HSA~\cite{hsa}
introduce mechanisms for checking static invariants in the dataplane.
Nonetheless, no principled troubleshooting mechanism exists yet for the SDN platform itself.

New operating system abstractions face an arduous path towards adoption
without sound troubleshooting mechanisms. Analogously, the success of the
SDN programming model depends heavily on the utility of its troubleshooting
paradigm. Our goal in this paper is to work towards a useful
troubleshooting mechanism for the SDN platform. \colin{</ why it's important>}

We observe that in eventually-consistent systems such as software-defined networks,
transient inconsistencies are an inevitable state-of-affairs.
In such an environment, the process of troubleshooting errors essentially boils down to
identifying relevant events among a clamor of diagnostic information.

We present two mechanisms designed to make it easier for operators and
developers to ``see through the noise'' of diagnostic information. The first,
cross-layer correspondence checking, leverages the structure of the SDN
architecture to enable a general and verifiable notion of platform
correctness. Correspondence checking allows troubleshooters to isolate the cause of 
an inconsistency to a particular layer without needing to define invariants or
instrument third-party code. Our second
mechanism, simulated replay analysis, allows troubleshooters 
to differentiate ephemeral from persistent inconsistencies by steering the
execution of the system forward and backward in time, filtering out extraneous
external events, and inducing uncommon events. \colin{</ what we did>}

The rest of this paper is organized as follows. In \S\ref{sec:overview},
we present an overview of the SDN stack and its failure modes.
In \S\ref{sec:approach} we present correspondence checking and simulated
replay analysis in detail. In \S\ref{sec:evaluation} we present
two use-cases of our techniques, as well as a preliminary evaluation
of their runtime. Finally, in \S\ref{sec:related_work} we discuss related work,
and in \S\ref{sec:conclusion} we conclude.
