The SDN platform's $raison\text{ }d'\hat{e}tre$ is to 
hide complexity from control applications. To this end, modern platforms perform
replication, resource arbitration, failure recovery, and network 
virtualization on the control application's behalf. 

While these measures are effective in simplifying control applications,
they do not remove any complexity from the overall system. Rather, they merely move the complexity
from control applications into the underlying SDN platform.

As in any software system, additional complexity increases the probability of
bugs in the platform. And unfortunately for the network operator, finding bugs in the platform requires
access to precisely the details hidden from the control application.
When operators encounters erratic behavior in their network, the error's root
cause may lie in their own policy specification, or in the SDN platform
itself. To deal with the latter case, they must trace through
multiple layers of abstraction: virtualization logic, distribution logic, and
network devices.

As it stands, the SDN platform provides meager support for troubleshooting.
Besides {\it ad-hoc} measurement tools
such as {\tt traceroute}, the predominant troubleshooting technique is
log analysis: manually specifying log statements at relevant points throughout the system;
collecting; gathering; and ordering distributed log files; and analyzing the
results {\it post-hoc} when a error is encountered in production. Besides its
apparent tediousness, this approach is lacking in several ways: logs events
are enormous in number, impossible to aggregate into a single serial
execution of the system, and often at the wrong level of granularity to be of
use.

Recent work has contributed much-needed improvements to the highest (control
application) and lowest (dataplane forwarding tables) levels of abstraction. 
NICE applies concolic execution and model checking to SDN control
applications, thereby automating the testing process and catching bugs before
they are deployed~\cite{nice}. Aneater~\cite{anteater} and HSA~\cite{hsa}
introduce mechanisms for checking static invariants in the dataplane.
Nonetheless, no principled troubleshooting mechanism exists yet for the SDN
platform itself.

It would be unthinkable to introduce a new operating system or programming
language without troubleshooting tools. Similarly, we think it highly undesirable to deploy SDN-based
networking without a viable troubleshooting paradigm. 

Correctness of the SDN platform can be stated concisely: ``high-level policies
should correspond with low-level configuration''. We observe that the structure
of the SDN platform, graphs at every layer, enables a straightforward
algorithm to check this invariant. Our algorithm, which we term
`correspondence checking',
enumerates all inconsistencies at any point in time, and isolates the
root cause of an inconsistency to a particular component of the system.

Correspondence checking does not suffice on its own; SDN
favors availability over consistency, and transient inconsistencies
between policy and configuration
are inevitable in any large eventually-consistent system.
In such an environment, it does not suffice for troubleshooting tools to
simply enumerate inconsistencies; they should also aid the developer
in identifying which are related to serious problems, and which are
harmlessly ephemeral. To this end we present \simulator.
\Simulator{} allows troubleshooters 
to filter out harmless inconsistencies by tracking the life cycle of problems 
both forward and backward in time.

We have implemented prototypes
of correspondence checking and \simulator{}. Our code is publicly available
at~\cite{github}.

The rest of this paper is organized as follows. In \S\ref{sec:overview},
we present an overview of the SDN stack and its failure modes.
In \S\ref{sec:approach} we present correspondence checking and
\simulator{} in detail. In \S\ref{sec:evaluation} we present
two use-cases and a preliminary performance evaluation
Finally, in \S\ref{sec:related_work} we discuss related work,
and in \S\ref{sec:conclusion} we conclude.
