\begin{itemize}
\item Goal of SDN is to hide complexity from operator. (not removing
complexity though!) 
\item $\rightarrow$ SDN platform is complex (distributed system! replicated controllers,
virtualization, multiple-tenants, failures, churn...)
\item $\rightarrow$ platform bugs are common
\item Troubleshooting requires access to details
\item $\rightarrow$ troubleshooting in SDN (without a debugger) is hard.
\colin{</ why it's hard>}
\item Status quo is to painstakingly look at logs. Some work on NOX (NICE), static
analysis of dataplane (Anteater, HSA) but nothing for the general
(Nicira-scale) SDN platform.
    \begin{itemize}
    \item Logs are static (non-interactive, only capture a single execution)
    \item Often at wrong level of granularity
    \item Distributed across time and space
    \end{itemize}
\item We're taking the ``painstakingly'' out of the picture
\item A new programming language abstraction wouldn't fly without a troubleshooting
mechanism!
\item $\rightarrow$ analogously, adoption of SDN programming model depends on usability of its
debugging paradigm  \colin{</ why it's important>}
\item We observe that in an eventually consistent system like SDN,
troubleshooting essentially boils down to ``seeing through the noise'' of
inconsistencies
\item We present techniques (that go beyond logging) to make it easier for
developers to ``see through the noise''
    \begin{itemize}
    \item Isolation to a particular layer:
        \begin{itemize}
        \item We leverage the structure of SDN (graphs all the way up) to
        enable a general notion of platform correctness: correspondence.
        \item $\rightarrow$ enables isolation without forcing the developer to
        instrument layers or define invariants
        \end{itemize}
    \item Isolation to particular events:
        \begin{itemize}
        \item Simulator enables ``time travel''
        \item $\rightarrow$ easy way to differentiate ephemeral from persistent inconsistencies
        \item Simulator enables fine-grained control over events
        \item $\rightarrow$ filter out extraneous external events -- focus only on relevant
        events
        \end{itemize}
    \end{itemize}
\end{itemize}

The rest of this paper is organized as follows. In \S\ref{sec:overview},
we present an overview of the SDN stack and its failure modes.
In \S\ref{sec:approach} we present correspondence checking, our approach to
troubleshooting safety violations in SDN. In \S\ref{sec:evaluation} we present
two use-cases of correspondence checking, as well as a preliminary evaluation
of its runtime. Finally, in \S\ref{sec:related_work} we discuss related work,
and in \S\ref{sec:conclusion} we conclude.
