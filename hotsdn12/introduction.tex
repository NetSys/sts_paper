The SDN platform's $raison\text{ }d'\hat{e}tre$ is to 
hide complexity from control applications. To this end, modern platforms perform
replication, resource arbitration, failure recovery, and network 
virtualization on the control application's behalf. 

While these measures are effective in simplifying control applications,
they do not remove any complexity from the overall system. Rather, they merely move the complexity
from control applications into the underlying SDN platform.

As in any software system, additional complexity increases the probability of
bugs. Moreover, network operators lack an automated mechanism to
distinguish bugs in their own policy specification from bugs in the underlying
platform. When they encounter erratic behavior in their network,
they are forced to trace through
multiple layers of abstraction: policy-specification, virtualization logic, distribution logic, and
network devices.

Besides {\it ad-hoc} measurement tools,
the predominant troubleshooting mechanism in SDN is
log analysis: manually specifying log statements at relevant points throughout the system;
collecting; gathering; and ordering distributed log files; and analyzing the
results {\it post-hoc} when a error is encountered in production. Log analysis
is an important component of the troubleshooting process, but it is deficient
as a stand-alone solution: logs events
are enormous in number, impossible to aggregate into a single serial
execution of the system, and often at the wrong level of granularity.

Recent work has contributed troubleshooting techniques focused on the highest (control
application) and lowest (dataplane forwarding tables) layers of the SDN stack.
NICE applies concolic execution and model checking to SDN control
applications, thereby automating the testing process and catching bugs before
they are deployed~\cite{nice}. Aneater~\cite{anteater} and HSA~\cite{hsa}
introduce mechanisms for checking static invariants in the dataplane.
Nonetheless, no automated troubleshooting mechanism exists for problems that span
multiple layers of the SDN stack, especially those caused by bugs in the platform
itself.

%The utility of new programming models is heavily dependent on
%the usability of their troubleshooting mechanisms.
%Similarly, we think it highly undesirable to deploy SDN-based
%networking without a viable troubleshooting paradigm. 

Verifying correctness of the SDN platform is equivalent to showing that
high-level policies specified by control applications correspond with 
low-level network configuration. We show that the structure of the
SDN platform (graphs at every layer) enables a straightforward
algorithm to check this invariant. Our algorithm, which we term
`correspondence checking',
enumerates all policy-violations at any point in time, and isolates the
root cause to a particular component of the system.

Correspondence checking does not suffice on its own; like any distributed
system, transient policy-violations due to failures and delays are 
common, especially at large scale. We present \simulator{} to
augment the diagnostic information provided by correspondence checking.
\Simulator{} allows troubleshooters 
to differentiate harmless from persistent policy-violations by tracking the life cycle of problems 
both forward and backward in time.

We demonstrate that these mechanisms combined can be used to quickly
identify the root cause of difficult errors, including isolation breaches,
faulty failover logic, and consistency problems between replicated
controllers. We have implemented prototypes
of correspondence checking and \simulator{}, and made the code publicly
available~\cite{github}.

The rest of this paper is organized as follows. In \S\ref{sec:overview},
we present an overview of the SDN stack and its failure modes.
In \S\ref{sec:approach} we present correspondence checking and
\simulator{} in detail. In \S\ref{sec:evaluation} we present
two use-cases and a preliminary performance evaluation
Finally, in \S\ref{sec:related_work} we discuss related work,
and in \S\ref{sec:conclusion} we conclude.
