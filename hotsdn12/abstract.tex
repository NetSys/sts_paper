The predominant technique for troubleshooting software-defined networks
is log analysis, which is generally tedious and error-prone. We observe that
in eventually-consistent systems such as networks, the process of troubleshooting
essentially boils down to identifying relevant events
amongst a clamor of inconsistencies and interacting components. We present two
mechanisms, cross-layer correspondence checking and simulated replay analysis, 
designed to make it easier for troubleshooters to ``see through the noise''.
