The predominant technique for troubleshooting bugs in software-defined networks,
log analysis, is tedious and error-prone. We observe that
in eventually-consistent systems such as SDN,
transient inconsistencies are inevitable. Beyond identifying inconsistencies,
troubleshooting mechanisms should therefore differentiate
pernicious from harmless ephemeral inconsistencies. We present two
mechanisms, cross-layer correspondence checking and simulated replay analysis, 
designed to make it easier for troubleshooters to ``see through the noise'' of
diagnostic information.

\colin{Need to add prototype, and rephrase inconsistency}
