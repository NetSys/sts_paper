The predominant technique for troubleshooting bugs in software-defined networks,
log analysis, is tedious and error-prone. We argue that a more principled
approach should be built around identifying inconsistencies between lower-level network configuration
and higher-level policies dictated by control applications. In
eventually-consistent systems such as SDN however,
transient inconsistencies between policy and configuration are inevitable.
Troubleshooting mechanisms should therefore provide a mechanism to differentiate
pernicious persistent from harmless ephemeral inconsistencies. We present two
mechanisms based on these principles, cross-layer correspondence checking and \simulator,
designed to make it easier for troubleshooters to ``see through the fog'' of
diagnostic information.
