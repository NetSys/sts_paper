In this section we present two mechanisms to facilitate the troubleshooting
process. Correspondence checking allows troubleshooters to isolate
the cause of policy-violations to a particular layer. \Simulator{}
allows troubleshooters to isolate relevant events throughout the system
execution. We provide the details of these techniques below. 

\subsection{Correspondence Checking}

The sole task of the application layer is to specify network
policies. The platform translates these high-level policies
into changes in the physical network.

Platform correctness can be expressed as a simple invariant:
the state of the application layer should correspond to the state of the
physical network. We check this invariant by applying the virtual packet
algebra pioneered in headerspace analysis~\cite{hsa}. 

Formally, each layer of the SDN stack can be represented as a graph,
$G = (V, E)$. Packets are series of bits, $h \in \{0,1\}^L = H$,
where $L$ is the maximum number of bits in the header. Upon receiving a packet,
forwarding elements apply a transformation function, potentially modifying
packets before forwarding them on\footnote{We assume unicast forwarding for
clarity}:
\begin{align*}
T: (H \times E) \rightarrow (H \times E_{\emptyset})
\end{align*}

We use $`\Psi`$ to denote the collection of all transfer functions present in
the network at a particular point in time. In this model, network traversal is simply a composition of transformation
functions. For example, if a header $h$ enters the network through edge
$e$, its state after $k$ hops will be:
\begin{align*}
\Phi^k(h,e) = \Psi(\Psi(\dots \Psi(h,e)\dots))
\end{align*}

The externally visible behavior of the network can be expressed as the
transitive closure of $\Phi$:
\begin{align*}
\Omega: (H \times E_{access}) \rightarrow (H \times E_{\emptyset}) \\
\Omega(h,e) = \Phi^{\infty}(h,e)
\end{align*}
Here, $E_{access}$ represents access links adjacent to end-hosts.

In SDN, it should always be the case that 
$\Omega^{view} \sim \Omega^{physical}$. Informally, this means that
any packet injected at an access link in $G^{virtual}$ should arrive at
the same final location as the corresponding (encapsulated) packet injected at the
corresponding access link in $G^{physical}$. Note that hosts are represented
in all graphs, although though there may not be a one-to-one mapping between the
internal vertices of $G^{virtual}$ and $G^{physical}$.

To check correspondence in SDN, we begin by taking a causally consistent
snapshot~\cite{Chandy:1985:DSD:214451.214456} of the physical network. The routing
tables of forwarding elements can then be translated into transformation functions.
Finally, we feed a symbolic packet $x^L$ to each access link of the
network~\footnote{The rules for process wildcard bits $x^n$ are defined in
the HSA paper~\cite{hsa}}. The end result is a propagation graph representing all possible paths taken by a packet injected
at the access link.

The leaves of the propagation graph represent $\Omega$. We
verify correspondence in SDN by generating propagation graphs for all SDN layers,
and comparing the leaves.

\subsection{\SIMULATOR{}}

For many use-cases, isolating an error to a particular layer suffices.
For example, a network operator might verify that policy-violations manifest in
the underlying platform, and thereby hand off
troubleshooting responsibility to a platform maintainer.

To isolate and fix the root cause of a problem however, troubleshooters will often
need more information. In particular, correspondence checking only enumerates
policy-violations present in the network at a particular point in time;
troubleshooters need to explore the execution of the system over
time. Our approach is to record the execution of the production network and replay 
it in a simulated model of the system.

Even in small networks, the number of low-level events captured by
log-statements can be overwhelming. The main utility of \simulator{} is that
troubleshooters can easily differentiate pertinent from
irrelevant diagnostic information by exluding extraneous events and
tracking the lifetime of policy-violations over time. At any point in time
during the replay, the troubleshooter
can run enumerate all policy-violations in the 
network with correspondence checking.

Correspondence checking and \simulator{} serve to isolate the systems
components and events responsible for a given error. To debug the code
responsible for the error, add log statements and iteratively repeat the
execution of the system until the relevant code block has been identified fixed. Modeling the system additionally
yields complete control of the timing, ordering and production of events.
For example, troubleshooters can proactively induce failures or packet drops
to reproduce the failure-condition observed in practice.

\colin{Out of place?} Note that for a proactive
network, only periodic snapshots of routing tables, link state events, and
control server failure events are needed for high-fidelity replay.

\colin{Mention that model checking could help here?}

