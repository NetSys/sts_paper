\documentclass{sig-alternate-10pt}
%\documentclass[letterpaper,11pt]{article}
\usepackage{url}
%\usepackage{usenix,epsfig,endnotes}
%\usepackage{fullpage} 
\setlength{\textheight}{9.5in}
%\setlength{\textwidth}{6.75in}
%\setlength{\oddsidemargin}{-.125in}
\usepackage{graphicx}
%\usepackage{subfigure}
%\usepackage{ifpdf}
%\usepackage{multicol}
%\usepackage{amsmath, amssymb, amsthm}
%\usepackage{rotating}
%\onehalfspacing
%\newcommand{\tbd}[1]{[{\bf{#1}}]}
\newcommand{\tbd}[1]{}
\newcommand{\ie}{{\it i.e.}}
\newcommand{\eg}{{\it e.g.}}
\newcommand{\etc}{{\it etc.}}
\newcommand{\eat}[1]{}
\newcommand{\projectname}[1]{{\it pcc}}
%\renewcommand\bibname{}

%\setlength\topmargin{0in}
%\usepackage{verbatim}
%\usepackage[compact]{titlesec}
%\usepackage[small]{caption}
\usepackage{times}

\title{Debugging Software Defined Networks\vspace{-25pt}}

\author{Colin Scott, Andreas Wundsam and Scott Shenker}\vspace{-15pt}

%        \begin{multicols}{2}{{\it Draft - Please do not distribute.}}\\
        
%\author{Paper \#69, 14 Pages}
\date{}
\begin{document}
    \maketitle
    \thispagestyle{empty}
% Outline:
%
%1 Problem formulation
%  1.1 Why is the problem important/relevant?
%  1.2 What are the main challenges in solving the problem?
%
%2 Approach
%  2.1 What is your approach? Briefly describe it
%  2.2 How is you approach different/similar from/to related work? You
%      don't need to be exhaustive here.
%
%3 Deliverables/milestones
%   3.1 What do you plan to accomplish by the end of the semester? What are the
%   metrics of success?
%   3.2 Do you plan to work on the project past the end of the semester?, and
%   if yes, what do you hope to accomplish by the end?
%   3.3 Specify measurable milestones by Oct. 26th
%  
%4 Any special needs for the project?

The complexity of modern SDN controllers is converging on that of general
purpose Operating Systems; the SDN platform handles virtualization,
resource arbitration, failure recovery and state distribution on the control
application's behalf.

Although the SDN platform's  $raison\text{ }d'\hat{e}tre$ is to 
hide complexity from the control application, its abstractions can 
aggravate the task of debugging network problems. When an application developer
encounters erratic behavior in the network, they must trace their
specification of intended behavior through
multiple layers of abstraction. Additionally, bugs may
manifest not only in the control application, but in any of the layers of
abstraction that translate control decisions to changes in the physical network:
virtualization logic, distribution logic, network devices, or the mappings
between any of these layers.

In ongoing work, we are developing a cross-layer debugger for software-defined
networks. It aims to simplify the detection, isolation, and repair of errors at
all layers of the network management stack. Our system, \projectname{},
interposes on the layers of the SDN stack to enable isolated invariant checking
of individual components.

\section{System}
Describe the simulator in more detail:
\begin{enumerate}
\item Simulator
\begin{enumerate}
\item Controller-agnostic
\item Supports Distributed Controllers
\item Complete control over network events
\end{enumerate}
\item Correspondence Check
\item Additional Invariant Checks
\end{enumerate}

\section{DNRC deployment goals}

\begin{enumerate}
\item Right now: evaluation through software simulation, openvswitch,
trivial hw topologies
\item  need to evaluation on larger topologies and 3rd party apps with realistic
workload
\item  differences sw-switches / hardware
\begin{enumerate}
\item several layers of scheduling
\item control system performance
\item dataplane throughput
\end{enumerate}

\item validate model
\begin{enumerate}
\item compare failure mode
\item collect traces from production bugs
\end{enumerate}
\end{enumerate}

\section{Summary}

yallayalla.

\end{document}
