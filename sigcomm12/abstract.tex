Software-defined networking (SDN) enables a structured approach to
networking: the SDN platform shields control logic from the complexities of
state distribution, and may additionally provide a
virtualized view of the underlying network state to enable highly concise
specifications of intended behavior. In hiding complexity from control logic
however, the task of troubleshooting errors becomes
considerably more difficult. Besides bugs in the control logic, technically
correct applications can be affected by bugs in the platform itself. Moreover, application
developers have limited visibility into how their control decisions are translated through
multiple layers of indirection down to individual configuration changes
in network devices themselves, making the process of isolating the root cause
of errors painstaking if not impossible. In this paper we argue that by
checking for a simple invariant, {\em correspondence} between abstraction
layers, the troubleshooting process can be greatly improved. We present \projectname{},
a cross-layer debugger built around the notion of correspondence checking. We
demonstrate that \projectname{} is able to track the causes of virtualization
and consistency bugs similar those observed in production SDN deployments.
